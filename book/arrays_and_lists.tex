

\chapter{Arrays y Listas}
\label{arrays}

Este capítulo presenta algunos de los tipos integrados
de Perl más útiles, los arrays y las listas.


\section{Las Listas y los Arrays Son Secuencias}
\label{sequence}

Al igual que las cadenas de texto, las {\bf listas} y los {\bf arrays}
son secuencias de valores. En una cadena de texto, los valores son
caracteres; en una lista o en un array, los valores pueden ser
de cualquier tipo. Los valores en una lista o en un array se 
conocen  como {\bf elementos} o algunas veces como {\bf artículos}.
\index{list}
\index{array}
\index{type!list}
\index{type!array}
\index{element}
\index{sequence}
\index{item}

Existen varias diferencias importantes entre las listas y los arrays.
La principal de ellas es que las listas son colecciones ordenadas e inmutables
de elementos: no puedes cambiar el número de elementos en una lista y tampoco puedes
cambiar los elementos individuales tampoco. Los arrays, por el contrario,
son variables y son generalmente mutables: puedes agregar elementos 
a un array, o quizás removerlos. Y también puedes acceder los
elementos individuales de un array y modificarlos. Para que esto sea
posible, los arrays usualmente tienen un nombre (como cualquier otra variable)
aunque algunos arrays son anónimos, lo que significa que no tienen 
un nombre, pero aún tienen otras formas de acceso.

Una lista es también efímera (al menos que se ha asignada a una
variable o algo más): deja de existir tan pronto ha sido usada,
usualmente tan pronto el flujo de ejecución del programa procede
a la siguiente línea de código. Por el contrario, un array tiene
alguna forma de persistencia: tú puedes usarlo en otras partes del
programa si la variable que lo contiene aún está dentro del
ámbito.

Hay varias maneras de crear una lista nueva;
la forma más simple es enumerar sus valores, separados
por comas:
\index{comma operator}
\index{operator!comma}

\begin{lstlisting}
> 3, 4, 5
(3 4 5)
> say (3, 4, 5).WHAT;
(List)
say $_ for 1, 2, 3;
1
2
3
\end{lstlisting}
%

No necesitas usar paréntesis para crear una lista,
pero son usualmente útiles para delimitar los valores,
i.e., para estipular dónde comienza y dónde termina, y
en algunos casos, anular la precedencia.

Ya hemos usado listas en este libro. Si escribimos:

\begin{lstlisting}
> print "$_ " for 1, 3, 5, 9, "\n";
1 3 5 9
 >
> print "$_ " for 1..10;
1 2 3 4 5 6 7 8 9 10 >
\end{lstlisting}

estamos básicamente creando y usando una lista de números
enteros (desde el punto de vista de la jerarquía de tipos en
Perl; esta observación no es totalmente correcta para el segundo
ejemplo, dado que \verb|1..10| es del tipo \emph{Range} y
es transformado en el tipo \emph{Seq}. Sin embargo, esta 
aproximación es suficientemente buena para nuestros 
propósitos aquí).
\index{range!type}
\index{range!operator}

Los arrays son variables cuyo nombre comienzan con
el sigilo \verb|@|. Los arrays con nombres necesitan ser declarados
antes de ser usados, como cualquier otra variable que hemos
visto hasta ahora (excepto la variable tópico, \verb|$_|). 
Una de las formas más fácil de crear un array es asignar una
lista a una variable:
\index{topical variable}
\index{sigil}

\begin{lstlisting}
> my @dígitos_impares = 1, 3, 5, 7, 9;
[1 3 5 7 9]
> say @dígitos_impares.WHAT;
(Array)
> my @números_solo_dígito = 0..9;
[0 1 2 3 4 5 6 7 8 9]
\end{lstlisting}

En el REPL de Perl, un array se muestra entre corchetes
(\verb|[| y \verb|]|), mientras que las listas se muestran
entre paréntesis.
\index{REPL}
\index{square bracket operator}
\index{operator!square bracket}
\index{bracket!square}

Si los elementos no contienen caracteres de espacio, es bien útil
construir una lista (y asignarla a cualquier array si es necesario)
con el operador quote-word \verb|<...>|:
\index{quote-word operator}

\begin{lstlisting}
> my @días-semana = <lun mar mie jue vie>;
[lun mar mie jue vi]
> my @fin-semana = <sab dom>;
[sab dom]
\end{lstlisting}

La ventaja de este método es que no hay necesidad de separar los 
elementos con comas y tampoco hay necesidad de insertar comillas
cuando los elementos son cadenas de texto. Básicamente, el operador
quote-word descompone su contenido por espacios en blanco y devuelve
una lista de palabras, la cual puede ser entonces usada en un bucle
o asignada a un array como en el ejemplo anterior.

La mayoría de este capítulo será dedicado a los arrays. No obstante,
ten presente que muchas de las funciones y operadores de los arrays
que estudiaremos aquí también funcionan con listas (por lo menos
todas aquellas que no violan la propiedad de inmutabilidad de las
listas). 

Los elementos de un array (o una lista) no necesitan ser del mismo
tipo:

\begin{lstlisting}
> my @array-heterogeneo = 1, 2.3, pi, "str", (1, 2, 4);
[1 2.3 3.14159265358979 str (1 2 4)]
\end{lstlisting}

Aquí, el array está compuesto por un entero, 
un racional, un número con coma flotante (tipo {\tt Num}),
una cadena de texto, y  una lista de tres enteros.
Aunque esto podría no ser recomendado por la sanidad mental
del programador quien tendría que usar un array con elementos
tan heterogéneos, Perl no se queja: al final, depende de ti hacer 
sentido de tus datos.

El array anterior hasta contiene una lista de artículos. 
Si iteras sobre los elementos de este array, por ejemplo con
el bucle for, esta lista terminará como un elemento distinto;
este elemento no será ``aplanado`` como tres elementos del
array. Similarmente, {\tt elems} es un método que cuenta el
número de artículos de un array (o de una lista). Al usarlo
en el array anterior produce el siguiente resultado:

\begin{lstlisting}
> say @array-heterogeneo.elems;
5
\end{lstlisting}
%

Como puedes observar, la lista \verb|(1, 2, 4)| cuenta como
un solo elemento en el array.

Una lista dentro de otra es una lista {\bf anidada}.
\index{nested list}
\index{list!nested}

Un array que carece de elementos se conoce
como un array vacío; puedes crear un array vacío con
paréntesis vacíos, \verb|()|:

\begin{lstlisting}
> my @vacío = ();
[]
\end{lstlisting}
\index{empty list}
\index{list!empty}

Este código está realmente asignando una lista vacía al array.
Pero esta sintaxis no es normalmente necesaria para crear un
nuevo array vacío, dado que solo con declarar un array sin 
definirlo tiene el mismo efecto:

\begin{lstlisting}
> my @vacío;
[]
\end{lstlisting}

Así que el uso de paréntesis vacíos (i.e., en la asignación
de una lista vacía) sería necesario para restaurar un
array existente como un array vacío.

\section{Los Arrays Son Mutables}
\label{mutable}
\index{list!element}
\index{access}
\index{index}
\index{subscript}
\index{bracket operator}
\index{operator!bracket}
\index{bracket!square}

\index{index!starting at zero}
\index{coercion}
La sintaxis para acceder los elementos de un array o una lista
usa el operador de corchetes. La expresión dentro de los corchetes
especifica el índice o subíndice, el cual puede ser un 
entero literal (o algún valor que puede ser coaccionado en un 
entero), una variable que contiene un valor numérico,
una lista o un rango de valores numéricos, una expresión o una pieza
de código que devuelve un valor numérico, etc. Comparado con el
comienzo de una array o una lista (de la misma 
manera que los valores devueltos por la función {\tt index}),
los índices comienzan en la posición 0. Así que, el primer elemento
de un array tiene índice 0, el segundo tiene índice 1, etc. Por ejemplo:
\index{zero, index starting at}


\begin{lstlisting}
say <sab dom>[1];             # -> sab (accediendo un elemento de la lista)
my @días-semana = <lun mar mie jue vie>;   # asignando un array
say "El tercer día es @días-semana[2]";    # -> El tercer día es mie
\end{lstlisting}
%

También puedes usar rangos o listas de índices para acceder
partes (o \emph{rebanadas}) de un array o una lista:
\index{slice}
\index{slice!operator}
\index{operator!slice}
\index{index!slice}
\index{list!slice}
\index{slice!list}
\index{range}

\begin{lstlisting}
> my @dígitos-pares = 0, 2, 4, 6, 8;
[0 2 4 6 8]
> my @pequeños-dígitos_pares = @dígitos-pares[0..2];
[0 2 4]
> my @min-max-dígitos-pares = @dígitos-pares[0, 4]
[0 8]
\end{lstlisting}

Si necesitas rebanadas en el orden opuesto, puedes usar
la función {\tt reverse} para revertir el rango:

\begin{lstlisting}
> my @opuestos-pequeños-dígitos_pares = @dígitos-pares[reverse 0..2];
[4 2 0]
\end{lstlisting}

o también puedes revertir los datos que resultan de la expresión
de rebanadas:

\begin{lstlisting}
> my @opuestos-pequeños-dígitos_pares = reverse @dígitos-pares[0..2];
[4 2 0]
\end{lstlisting}

A diferencia de las listas, los arrays son mutables. 
Cuando el operador de corchetes aparece después de un array 
en el lado izquierdo de una asignación, el operador identifica el
elemento del array que será asignado:
\index{mutability}

\begin{lstlisting}
> my @dígitos-pares = 0, 2, 2, 6, 8;   # Oops, error en el segundo 2
[0 2 2 6 8]
> @dígitos-pares[2] = 4; # arreglando el tercer elemento
4
> say @dígitos-pares
[0 2 4 6 8]
\end{lstlisting}
%

El tercer elemento de {\tt \@dígitos-pares}, el cual es (por error) 2,
es ahora 4. Si el índice corresponde a un artículo que no existe aún
en el array, el array se expandaré para incluir el nuevo elemento:

\begin{lstlisting}
> my @impares = 1, 3, 5;
[1 3 5]
> @impares[3] = 7;
7
> say @impares;
[1 3 5 7]
\end{lstlisting}

\index{index!starting at zero}
\index{item assignment}
\index{assignment!item}
\index{reassignment}

\index{elems function or method}
\index{end method}
La función o método {\tt elems} devuelve el número
de elementos de un array. La función o método {\tt end}
devuelve el índice del último elemento de un array:

\begin{lstlisting}
my @nums = 1..5;      # -> [1 2 3 4 5]
say @nums.elems;      # -> 5
say elems @nums;      # -> 5
say @nums.end;        # -> 4
\end{lstlisting}

El método {\tt end} devuelve el resultado del método 
{\tt elems} menos uno porque, dado que los índices comienzan
en la posición 0, el índice del último elemento es igual
al número de elementos menos uno.
\index{zero, index starting at}

La función o método {\tt unique} devuelve una secuencia
de elementos únicos en la lista de entrada o el array (
i.e., el método devuelve la lista original si no hay valores
duplicados):
\index{unique function}
\index{function!unique}

\begin{lstlisting}
> say < a b d c a f d g>.unique;
(a b d c f g)
\end{lstlisting}

Si sabes que la entrada está ordenada (y que, por lo tanto,
los duplicados son adyacentes), en vez de usar la función
{\tt unique}, usa la función {\tt squish} debido a que termina
siendo más eficiente. La función {\tt squish} remueve elementos
duplicados y adyacentes.
\index{squish function}
\index{function!squish}

Para saber si dos arrays son idénticos (estructuralmente 
los mismos, con el mismo tipo y los mismos valores), usa 
el operador de equivalencia {\tt eqv}. Para saber si los arrays
contienen los elementos, usa el operador de coincidencia inteligente
\verb|~~|. Entre dos arrays o listas, el operador de equidad numérica
\verb|==| devolverá {\tt True} si los arrays tienen el mismo número 
de elementos y de lo contrario, {\tt False}. Esto se debe a que
\verb|==| coacciona sus argumentos a tipo numérico, así que compara
el número de elementos:
\index{eqv operator}
\index{coercion}
\index{equivalence operator}
\index{operator!eqv}
\index{smart match operator}
\index{operator!smart match}
\index{numeric equality operator}
\index{operator!numeric equality}

\begin{lstlisting}
> my @pares1 = 0, 2, 4, 6, 8;
[0 2 4 6 8]
> my @pares2 = reverse 8, 6, 4, 2, 0;
[0 2 4 6 8]
> say @pares1 eqv @pares2;        # mismos elementos, misma estrutura
True
> say <1 2 3 4 5> eqv 1..5;       # mismos elementos, estructura diferente
False
> say <1 2 3 4 5> ~~ 1..5;        # mismos elementos, True
True
> my @array = 1..5;               
[1 2 3 4 5]
>  say <1 2 3 4 5> ~~ @array;     # mismos elementos, True
True
>  say <1 2 3 4 6> ~~ @array;     # no los mismos elementos
False
> say <1 2 3 4 5> == <5 6 7 8 9>; # compara el número de elementos
True
\end{lstlisting}

La sentencia \verb'<1 2 3 4 5> eqv 1..5' devuelve \verb|False|, debido 
a que aunque contienen los mismos elementos, los argumentos 
son estructuralmente entidades diferentes (uno es una lista y 
la otra es un rango).

\section{Cómo Agregar o Remover Elementos de un Array}

Hemos visto que la asignación de un elemento a un índice 
que no existe expande el array. Existen otras formas de expandir
un array.

Perl tiene operadores para agregar elementos a un array, o remover
un elemento del mismo:
\index{pop function}
\index{push function}
\index{shift function}
\index{unshift function}

\begin{itemize}
\item {\tt shift}: elimina el primer elemento del array y lo devuelve;
\item {\tt pop}: elimina el último elemento del array y lo devuelve;
\item {\tt unshift}: agrega un elemento o una lista de elementos al principio 
del array;
\item {\tt push}: agrega un elemento o una lista de elementos al final 
del array;
\end{itemize}

Estos son algunos ejemplos sobre cada uno de ellos:

\begin{lstlisting}
> my @números = <2 4 6 7>;
[2 4 6 7]
> push @números, 8, 9;
[2 4 6 7 8 9]
> unshift @números, 0, 1;
[0 1 2 4 6 7 8 9]
> my $num = shift @números
0
> $num = pop @números
9
> say @números
[1 2 4 6 7 8]
\end{lstlisting}

Como podrías esperar por ahora, estas subrutinas también
tienen una sintaxis de invocación de método. Por ejemplo:
\index{invocation!method}
\index{method invocation}

\begin{lstlisting}
> my @números = <2 4 6 7>;
[2 4 6 7]
> @números.push(8, 9)
[2 4 6 7 8 9]
\end{lstlisting}

No obstante, debes tener presente que si usas las funciones
{\tt push} o {\tt unshift} con un array como argumento, 
obtendrás algo diferente a lo que podrías esperar:

\begin{lstlisting}
> my @números = <2 4 6 7>;
[2 4 6 7]
> my @agregar-array = 8, 10;
[8 10]
> @números.push(@agregar-array);
[2 4 6 7 [8 10]]
\end{lstlisting}

Como puedes observar, cuando \verb|@agregar-array| se agrega
como una entidad al array \verb|@números|, \verb|@agregar-array|
se convierte en un elemento nuevo del array original. Si lo que 
quieres es agregar los elementos de \verb|@agregar-array| 
al array original, puedes usar el método {\tt append} en vez 
de {\tt push}:

\begin{lstlisting}
> my @números = <2 4 6 7>;
[2 4 6 7]
> @número.append(@agregar-array);
[2 4 6 7 8 10]
\end{lstlisting}

O puedes usar el operador prefijo ``|``, el cual aplana
el array añadido en una lista de argumentos:

\begin{lstlisting}
> my @números = <2 4 6 7>;
[2 4 6 7]
> @números.push(|@agregar-array);
[2 4 6 7 8 10]
\end{lstlisting}

También existe el método {\tt prepend} que reemplaza 
a {\tt unshift} para agregar elementos individuales 
de un array al principio de otro array existente (
en vez de añadir el array como una sola identidad).

\section{Pilas y Colas}
\label{stacks_queues}

\index{stack}
\index{queue}
Las pilas (\emph{stacks} en inglés) y las colas ({\emph{queues} en inglés}) son 
estructuras de datos muy usadas en la ciencia de la computación.

\index{LIFO (last in / first out)}
\index{last in / first out (LIFO)}
Una pila es una estructura de datos de tipo LIFO (del inglés 
\emph{Last In / First Out}, ``último en entrar, primero en salir``).
Puedes pensar sobre una pila como un grupo de platos apilados. Cuando
pones un plato limpio en la pila, usualmente lo pones en la parte superior;
de igual manera cuando tomas un plato, lo toma de la parte superior.
Así que el primer plato que tomas fue el último en ser añadido. Una pila
en la ciencia de la computación implementa la misma idea: usas una pila
cuando la primera pieza de dato que necesitas de una estructura de datos fue
la última en ser añadida.

\index{FIFO (first in / first out)}
\index{first in / first out (FIFO)}
Por el contrario, una cola (o fila) es una estructura de datos de tipo FIFO (
del inglés \emph{First In / First Out}, ``primero en entrar, primero en
salir``). Esta es la idea de personas esperando en línea para pagar en el
supermercado. La primera persona en ser atendida es la primera
persona que entra en la fila.

Una pila puede ser implementada con un array y las funciones
{\tt push} y {\tt pop}, las cuales agregan un elemento (o varios)
al final de un array y remueven un elemento del final del array 
respectivamente. Esta es una implementación bien simple de una pila:
\index{pop function}
\index{push function} 	
\index{function!pop}
\index{function!push}
\label{stack_code}

\begin{lstlisting}
sub poner-en-pila (@pila, $new_elem) {
	push @pila, $new_elem;
}
sub tomar-desde-pila (@pila) {
    my $elem = pop @pila;
    return $elem;
}
my @una-pila = 1, 2, 3, 4, 5;
poner-en-pila @una-pila, 6;
say @una-pila;
say tomar-desde-pila @una-pila for 1..3;
\end{lstlisting}

Este ejemplo imprimirá esto:

\begin{lstlisting}
[1 2 3 4 5 6]
6
5
4
\end{lstlisting}

\index{stack}
Esta pila es simplista porque, por lo menos, una implementación
más robusta debería hacer algo sensible cuando tratas de 
{\tt tomar-desde-pila} cuando la pila está vacía. También sería
sabio agregar signaturas a las subrutinas. En adición, podrías querer
{\tt poner-en-pila} más de un elemento al mismo tiempo. Échale un
vistazo a la solución sobre el ejercicio sobre colas más abajo
(Subsección~\ref{sol_exercise_queue}) para descifrar cómo esta pila
podría ser mejorada.
\index{signature}
\index{subroutine signature}

Podrías obtener las mismas funciones usando las subrutinas
{\tt unshift} y {\tt shift} en vez de {\tt push} y {\tt pop}.
Los elementos será añadidos al principio del array y tomados
desde el inicio, pero todavía tendrás el comportamiento LIFO.
\index{shift function}
\index{unshift function}
\index{push function}
\index{LIFO}

\label{exercise_queue}
Como ejercicio, trata de implementar una cola de tipo FIFO en el
mismo modelo. Pista: probablemente quieres usar un array y las
funciones {\tt unshift} y {\tt pop} (o las funciones 
{\tt push} y {\tt shift}). Solución: \ref{sol_exercise_queue}.
\index{queue}
\index{shift function}
\index{unshift function}
\index{push function}
\index{pop function}
\index{FIFO}

\section{Otras Formas de Modificar un Array}
\label{modify_array}

Las funciones {\tt shift} and {\tt pop} eliminan el primer
y último elemento de un array respectivamente y devuelven
ese elemento. Es posible casi hacer la misma operación
con cualquier elemento de un array, usando el \emph{adverbio}
{\tt delete}:

\begin{lstlisting}
my @frutas = <manzana banana fresa mango piña naranja>;
my $removida = @frutas[2]:delete; 
say $removida;  # -> fresa
say @frutas;    # -> [<manzana banana (Any) mango piña naranja]
\end{lstlisting}

Nota que el tercer elemento (``fresa``) ha sido removido
y devuelto, pero el array no ha sido reorganizado; la operación
deja un tipo de ``hueco``, un artículo indefinido, en el medio
del array. La sintaxis de los dos puntos (``:``) que se usa aquí es 
el operador para un adverbio (discutimos adverbios en la 
Sección~\ref{regex} sobre regexes); por el tiempo presente, puedes
imaginarte este operador como un método especial que opera sobre un
elemento de una colección de elementos.

Hemos visto cómo rebanar un array para tomar varios elementos de
un array o una lista al mismo tiempo. La misma sintaxis de 
rebanar puede ser usada en el lado izquierdo de una asignación
para modificar algunos elementos de un array:
\index{slice!assignment}

\begin{lstlisting}
my @dígitos = <1 2 3 6 5 4 7 8 9>
@dígitos[2..4] = 4, 5, 6
say @dígitos;   # -> [1 2 4 5 6 4 7 8 9]
\end{lstlisting}

Por supuesto no puedes hacer esto con listas, debido a que
las listas son inmutables.

La función {\tt splice} puede ser considerada como la Navaja Suiza
de los arrays. Esta función puede añadir, remover, y devolver varios 
elementos de un array. La sintaxis general es como sigue:

\begin{lstlisting}
my @array_salida = splice @array_entrada, $inicio, $número_elems, @reemplazo;
\end{lstlisting}
%
Los argumentos para {\tt splice} son el array de entrada, el
índice del primer elemento sobre el cual se harán los cambios,
el número de elementos afectados por la operación, y una lista
de reemplazos para los elementos a ser removidos\footnote{
Nota que la función {\tt splice} en arrays tiene casi la misma
sintaxis que la función {\tt substr} en cadenas de texto. Esto
hará más fácil entender y recordar la sintaxis.}. Por ejemplo, 
para realizar la asignación de rebanada mostrada más arriba, es 
posible hacer esto:

\begin{lstlisting}
my @dígitos = <1 2 3 6 5 4 7 8 9>
my @dígitos_removidos = splice @dígitos, 3, 3, 4, 5, 6;
say @dígitos_removidos;     # -> [6 5 4]
say @dígitos;             # -> [1 2 3 4 5 6 7 8 9]
\end{lstlisting}
%
Aquí, la sentencia {\tt splice} removió tres elementos (6, 5, 4)
y los reemplazó con los reemplazos en los argumentos (4, 5, 6).
También devolvió los artículos removidos, los cuales
fueron almacenados en \verb|@dígitos_removidos|.
El número de reemplazos no necesita ser igual al número de elementos
removidos. En tal caso, el array crecerá o se reducirá. Por ejemplo,
si no se proveen reemplazos, entonces {\tt splice} removerá y devolverá
el número requerido de elementos y el array se reducirá por el
mismo número:

\begin{lstlisting}
my @dígitos = 1..9;
my @dígitos_removidos = splice @dígitos, 3, 2;
say @dígitos_removidos;     # -> [4 5]
say @dígitos;             # -> [1 2 3 6 7 8 9]
\end{lstlisting}
%

Por el contrario, si el número de elementos a removerse es
cero, no se removerá ningún elemento, un array vacío será
devuelto, y los elementos en la lista de reemplazos será
añadidos en el lugar correcto:

\begin{lstlisting}
my @dígitos = <1 2 3 6 4 7 8 9>;
my @dígitos_removidos = splice @dígitos, 3, 0, 42;
say @dígitos_removidos;     # -> []
say @dígitos;             # -> [1 2 3 42 6 4 7 8 9]
\end{lstlisting}
%

Asumiendo que la función {\tt shift} no existiera en Perl,
tú podrías escribir una subrutina {\tt mi-shift} para 
simularla:

\begin{lstlisting}
sub mi-shift (@array) {
    my @resultado = splice @array, 0, 1;
    return @resultado[0];
}
my @letras = 'a'..'j';
my $letra = my-shift @letras;
say $letra;             # -> a
say @letras;            # -> [b c d e f g h i j]
\end{lstlisting}

Podrías levantar ina excepción si el array que se
pasa a la función {\tt mi-shift} está vacío. Esto
podría hacerse al modificar la subrutina de la siguiente
manera:

\begin{lstlisting}
sub mi-shift (@array) {
    die "No se puede remover elemento de una array vacío" unless @array;
    my @resultado = splice @array, 0, 1;
    return @resultado[0];
}
\end{lstlisting}
%

o al añadir un restricción al array en la 
signatura de la subrutina:
\index{signature}
\index{subroutine signature}

\begin{lstlisting}
sub mi-shift (@array where @array > 0) {
    my @resultado = splice @array, 0, 1;
    return @resultado[0];
}    
\end{lstlisting}
%

La expresión \verb'@array > 0' evalúa a {\tt True} si el
número de elementos del array es más que 0, i.e., básicamente,
si no es un array vacío. Es equivalente a la expresión 
\verb|@array.elems > 0|.

\label{splice_exercise}
Como ejercicio, escribe subrutinas usando \verb|splice|
para simular las funciones integradas {\tt pop}, {\tt unshift}, 
{\tt push}, y {\tt delete}. Solución: \ref{sol_splice_exercise}.


\section{Cómo Recorrer una Lista}
\index{list!traversal}
\index{traversal!list}
\index{for loop}
\index{loop!for}
\index{statement!for}

La manera más común de recorrer los elementos de una lista
o un array es con el bucle {\tt for}. La sintaxis para un 
array es la misma que hemos usados en los capítulos
anteriores para las listas:

\begin{lstlisting}
my @colores-inglés = <red orange yellow green blue indigo violet>;
for @colores-inglés -> $color {
    say $color;
}
\end{lstlisting}
%
Esto funciona bien si solo necesitas leer los elementos de
la lista. Pero si quieres escribir o actualizar los elementos
de un array, tú necesitas un bloque de doble punta. Por ejemplo,
podrías usar la función {\tt tc} (``title case`` en inglés) para
capitalizar la primera letra de cada palabra de un array:
\index{tc function}

\begin{lstlisting}
my @colores-inglés = <red orange yellow green blue indigo violet>;
for @colores-inglés <-> $color {$color = tc $color};
say @colores-inglés;   # -> [Red Orange Yellow Green Blue Indigo Violet]
\end{lstlisting}
%
Aquí la variable \verb|$color| del bucle es un \emph{apodo} 
de lectura-escritura (``read-and-write`` en inglés)de 
los elementos del array, así que cualquier cambio 
que se hace a este apodo se refleja en el array. Esto funciona
bien con arrays, pero no funcionaría con listas, debido a que 
ellas son inmutables. Obtendrías el siguiente error al usarlo
con una lista:

\begin{lstlisting}
> for <red orange yellow> <-> $color { $color = tc $color}
Parameter '$color' expected a writable container, but got Str value...
\end{lstlisting}

También puedes usar la sintaxis de un bucle {\tt for}
con la variable tópico \verb|$_|. Por ejemplo, esto usa la 
función {\tt uc} (``upper case`` en inglés) para 
capitalizar cada palabra del array anterior:
\index{topical variable}
\index{uc function or method}

\begin{lstlisting}
for @colores-inglés { 
    $_ = $_.uc 
}
say @colores-inglés; # -> [RED ORANGE YELLOW GREEN BLUE INDIGO VIOLET]
\end{lstlisting}
%

Algunas veces quieres recorrer un array y necesitas saber
el índice de los elementos que estás visitando. Una forma
común de hacer esto es con el operador de rango \verb|..|
para iterar sobre los índices. Por ejemplo, para imprimir 
el índice y el valor de cada elemento de un array:

\begin{lstlisting}
for 0..@colores-inglés.end -> $idx { 
    say "$idx  @colores-inglés[$idx]"; 
}
\end{lstlisting}

Esto es útil, por ejemplo, para recorrer dos (o más) arrays
en paralelo:

\begin{lstlisting}
my @letras = 'a'..'e';
my @números = 1..5;
for 0..@letras.end -> $idx { 
    say "@letras[$idx] -> @números[$idx]"; 
}
\end{lstlisting}
%

Esto imprimirá:
\begin{lstlisting}
a -> 1
b -> 2
c -> 3
d -> 4
e -> 5
\end{lstlisting}

No necesitas especificar el rango de índices tú mismo, dado que
la función {\t keys} devolverá una lista de índices para un
array o una lista:
\index{keys function or method}

\begin{lstlisting}
for keys @colores-inglés -> $idx { 
    say "$idx  @colores-inglés[$idx]"; 
}
\end{lstlisting}

Otra forma de iterar sobre los índices y valores de un
array es con el uso de la función o método 
{\tt kv} (``keys values``-``llaves valores``) que
devuelve el índice y el valor de cada elemento del array:
\index{kv function or method}

\begin{lstlisting}
for @letras.kv -> $idx, $val { 
    say "$idx $val";
}
\end{lstlisting}

En un contexto de lista, \verb|@letters.kv| simplemente
devuelve un secuencia entrelazada de índices y valores:

\begin{lstlisting}
my @letras = 'a'..'e';
say @letras.kv;    # -> (0 a 1 b 2 c 3 d 4 e)
\end{lstlisting}

Es el bloque de doble punta con las variables de iteración
que hace posible procesar un índice y una valor 
a cada paso del bucle. Por supuesto, puedes tener 
más de dos variables de iteración si es necesario.


\section{Nuevas Construcciones de Repetición}
\index{loop}

Dado que el tema de este capítulo son los arrays y las 
listas, es probablemente el tiempo perfecto para estudiar
brevemente dos construcciones de repetición que hemos
dejado a un lado hasta ahora.

La primera usa la misma palabra clave {\tt for} usada
anteriormente, pero con una sintaxis diferente para la iteración
de variable(s):
\index{for loop}

\begin{lstlisting}
my @letras = 'a'..'e';
for @letras { 
    say $^a-letra; 
}
\end{lstlisting}

\index{placeholder}
\index{placeholder!parameter}
\index{twigil}
\index{sigil}
\index{self-declared parameter}
El símbolo \verb|^| en la variable \verb|$^a-letra| es conocido
como un \emph{twigil}, i.e., un tipo de sigilo secundario. 
Cuando hay un twigil, el primer símbolo (en este caso, el signo
\verb|$|) posee el mismo significado de los sigilos usuales (aquí, denota
una variable escalar) y el segundo (aquí, \verb|^|) extiende la descripción
de la variable y usualmente modifica su ámbito. En este caso
específico, el segundo carácter especifica que la variable
\verb|$^a-letter| es un \emph{parámetro marcador} o un 
\emph{parámetro auto-declarado de posición}. Es decir, es un parámetro
de posición del bloque actual que no necesita ser declarado en la 
signatura.
\index{signature}

Si el bloque usa más de un marcador, ellos son asociados con la
entrada de datos de acuerdo a su orden lexicográfico (alfabético):

\begin{lstlisting}
my @letras = 'a'..'e';
for @letras.kv { 
    say "$^a -> $^b"; 
}
\end{lstlisting}
%
Esto imprimirá:
\begin{lstlisting}
0 -> a
1 -> b
2 -> c
3 -> d
4 -> e
\end{lstlisting}
%

Como vimos anteriormente, la función {\tt kv} devuelve
una secuencia entrelazada de índices y valores. Dado
que \verb|$^a| viene antes que \verb|$^b| en el orden
alfabético, \verb|$^a| será atada al índice y \verb|$^b|
al valor de cada pareja de la entrada de datos.

Los marcadores pueden también ser usados en las
subrutinas:

\begin{lstlisting}
> sub dividir { $^primero / $^segundo }
sub dividir ($primero, $segundo) { #`(Sub|230787048) ... }
> dividir 6, 4
1.5
\end{lstlisting}
%

Estos marcadores no se usan muy a menudo para recorrer arrays,
pero veremos más adelante lo útiles
que son en casos donde no sería muy práctico tener que declarar los
parámetros.
\index{placeholder}

\index{loop!keyword}
\index{loop!statement}
\index{C-style loop}
\label{C-style loop}
La segunda construcción de repetición que quiero introducir
aquí usa la palabra clave {\tt loop} y es similar
al bucle for de stilo C (i.e., el bucle del lenguaje
de programación C). En este tipo de bucle, tú declaras
entre una pareja de paréntesis tres expresiones separadas
por puntos y comas: el valor inicial de la variable de iteración,
la condición por la cual el bucle debería terminar, y el cambio
que se hace a la variable de iteración en cada iteración:

\begin{lstlisting}
loop (my $i = 0; $i < 5; $i++) {
    say $i, " -> " ~ @letras[$i];
}
\end{lstlisting}
%
Para los bucles más comunes, el bucle {\tt for} visto anteriormente
es más fácil de escribir y usualmente más eficiente que esta
construcción. Esta construcción especial {\tt loop} debería 
ser probablemente usada solo cuando la salida de una condición
o el cambio que se hace a la variable de iteración es bien inusual
y sería algo difícil de expresar en un bucle {\tt for} regular. 
Como una excepción, la construcción {\tt loop} sin la especificación
de tres partes es bien común  y hasta idiomática para hacer un bucle
infinito:
\index{infinite loop}
\index{loop!infinite}
\index{idiomatic}

\begin{lstlisting}
loop {
    # haz algo
    # termina si ...
}
\end{lstlisting}
%


\section{Asociaciones, Filtros y Reducciones}
\label{map_filter}

Al recorrer los elementos de un array (o una lista), lo que hemos
hecho hasta ahora es procesar los elementos uno por uno con un 
bucle. Ahora estudiaremos formas de procesar todos los elementos
al mismo tiempo.

\subsection{Reducir una Lista a un Valor}

Para añadir todo los números en una lista, puedes usar 
el bucle {\tt for} de la siguiente manera:

\begin{lstlisting}
sub agrega-todos (@números) {
    my $total = 0;
    for @números -> $x {
        $total += $x;
    }
    return $total;
}
\end{lstlisting}
%
La variable {\tt \$total} es inicializada a0. Cada vez que el
bucle itera, la variable {\tt \$x} consigue uno de los elementos
de la lista y es agregada a {\tt \$total}. Mientra el bucle se ejecuta,
{\tt \$total} acumula la suma de los elementos; una variable que 
es usada de esta forma se conoce como un {\bf acumulador}.
\index{accumulator!sum}

Una operación que combina una secuencia de elementos en un
solo valor es usualmente llamada una operación de reducción porque
su efecto es {\bf reducir} todos los artículos a un solo elemento
(esto es conocido como ``pliegues`` (del inglés \emph{folding})
en otros lenguajes de programación). Estas ideas son derivadas de
lenguajes de programación funcionales tal como LISP (cuyo nombre
deriva de ``LISt Processing`` (procesamiento de listas)).
\index{reduce pattern}
\index{pattern!reduce}
\index{traversal}

Perl~6 tiene una función {\tt reduce}, la cual genera un solo
valor "combinado" de la lista de valores. Esto se lleva a cabo 
al aplicar de forma iterativa a cada artículo una
función que sabe cómo combinar dos valores. El uso de la función
{\tt reduce} para calcular la suma de los primeros 
diez números luce así:

\begin{lstlisting}
> my $suma = reduce { $^a + $^b }, 1..10;
55
\end{lstlisting}

?`Recuerdas la función \emph{factorial} de la Sección~\ref{for_loops}?
Esta función usaba un bucle {\tt for} para calcular el producto de los
primeros enteros hasta un cierto límite. Podría ser escrita de nuevo 
usando la función {\tt reduce}:
\index{factorial}
\index{factorial!using the reduce function}

\begin{lstlisting}
sub factorial (Int $num) { 
    return reduce { $^a * $^b }, 1..$num;
}
say factorial 10;   # -> 3628800
\end{lstlisting}
%
De hecho, el código para calcular el factorial es tan corto
con la función {\tt reduce} que se puede argumentar se ha vuelto
innecesario escribir una subrutina para eso. Podrías solo
poner el código ``en línea``:

\begin{lstlisting}
my $fact10 = reduce { $^a * $^b }, 1..10;     # -> 3628800
\end{lstlisting}
%

Podemos hacer muchísimas cosas poderosas con esto, pero regresaremos
a esto más tarde, debido a que requiere algunas características 
sintácticas que no hemos visto todavía.

\subsection{El Metaoperador de Reducción}

Perl~6 tiene un operador de reducción, o más bien 
un \emph{metaoperador} de reducción. Un operador usualmente
trabaja con variables o valores; un metaoperador actúa sobre 
otros operadores. Dada una lista y un operador, el metaoperador
\verb|[...]| aplica de forma iterativa el operador 
a todos los valores de la lista para producir un solo
valor.
\index{reduction operator}
\index{metaoperator}

Por ejemplo, el código siguiente también imprime la suma de todos
los elementos de una lista:

\begin{lstlisting}
say [+] 1, 2, 3, 4;           # ->  10
\end{lstlisting}

Esto básicamente toma los primeros dos valores, los agrega,
y agrega el resultado al siguiente valor, continuando 
hasta que haya recorrido la lista completa. Actualmente,
existe una forma de este operador, con una barra invertida 
delante del operador, el cual devuelve los resultados 
intermedios de la operación:

\begin{lstlisting}
say [\+] 1, 2, 3, 4;          # -> (1 3 6 10)
\end{lstlisting}

Este metaoperador puede usarse para transformar básicamente
cualquier operador asociativo infijo\footnote{Un operador 
\emph{infijo} es un operador que se coloca
entre dos operandos.} en un operador de lista 
que devuelve un solo valor.


La función factorial puede escribirse de nuevo
en la siguiente forma:
\index{factorial}
\index{factorial!using the reduction meta-operator}

\begin{lstlisting}
sub fact(Int $x){
    [*] 1..$x; 
}
my $factorial = fact(10);     # -> 3628800
\end{lstlisting}

El metaoperador de reducción puede también ser usado con
operadores de relación para chequear si los elementos de
un array o una lista están en el orden numérico o alfabético
correcto:

\begin{lstlisting}
say [<] 3, 5, 7;              # -> True
say [<] 3, 5, 7, 6;           # -> False
say [lt] <a c d f r t y>;     # -> True
\end{lstlisting}

\subsection{Asociando una Lista a Otra Lista}
\index{map}

Algunas veces tú quieres recorrer una lista mientras 
construyes otra. Por ejemplo, la siguiente función toma una
lista de cadenas de texto y devuelve una lista nueva que 
contiene las cadenas de texto en letras mayúsculas:

\begin{lstlisting}
sub mayúsculas(@palabras){
    my @resultado;
    push @resultado, $_.uc for @palabras;
    return @resultado;
}
my @palabras_minus = <one two three>;
my @palabras_mayus = mayúsculas(@palabras_minus); # -> [ONE TWO THREE]
\end{lstlisting}
%
La variable \verb|@resultado| es declarado como un array vacío;
cada vez que el bucle itera, agregamos el siguiente el 
siguiente elemento. Así que \verb|@resultado| es otro tipo de
acumulador.
\index{accumulator!list}

Una operación como \verb|mayúsculas| es algunas veces conocida
como un {\bf map} porque ``maps`` (asocia) una función (en este
caso el método {\tt uc}) a cada elemento en una secuencia.
\index{map pattern}
\index{pattern!map}
\index{filter pattern}
\index{pattern!filter}

Perl tiene una función {\tt map} que hace posible todo eso
en solo una sentencia:

\begin{lstlisting}
my @palabras_minus = <one two three>;
my @palabras_mayus = map { .uc }, @palabras_minus;      # -> [ONE TWO THREE]
\end{lstlisting}
%

Aquí la función {\tt map} aplica el método {\tt uc} a cada
elemento del array \verb|@palabras_minus| y los devuelve
en el array \verb|@palabras_mayus|. Más precisamente, 
la función {\tt map} asigna de forma iterativa cada elemento
del array \verb|@palabras_minus| a la variable tópico \verb|$_|,
aplica el bloque de código después de la palabra clave {\tt map}
a \verb|$_| para crear nuevos valores, y devuelve una lista de estos
valores.
\index{map function}
\index{topical variable}
\index{map}

Para generar una lista de números pares entre 1 y 10, podríamos
usar el operador de rango para generar números entre 1 y 5
y usar {\tt map} para multiplicarlos por dos:

\begin{lstlisting}
my @pares = map { $_ * 2 }, 1..5;  # -> [2 4 6 8 10]
\end{lstlisting}
%

En vez de usar la variable tópico \verb|$_|, podríamos también
usar la sintaxis del bloque puntiagudo con una variable de 
iteración explícita:

\begin{lstlisting}
my @pares = map -> $num { $num * 2 }, 1..5;  # -> [2 4 6 8 10]
\end{lstlisting}
%

o un bloque anónimo con una variable marcador:

\begin{lstlisting}
my @pares = map { $^num * 2 }, 1..5;  # -> [2 4 6 8 10]
\end{lstlisting}
%

En vez de un bloque de código, el primer argumento para {\tt map}
puede ser una referencia de código (una referencia de una
subrutina):

\begin{lstlisting}
sub doble-raíz-más-uno (Numeric $x) { 
    1 + 2 * sqrt $x;
}
my @resultados = map &doble-cuadrado-más-uno, 4, 16, 42;
say @resultados;     # -> [5 9 13.9614813968157]
\end{lstlisting}
%
\index{map}

El nombre de la subrutina necesita ser precedido con el 
sigilo {\tt \&} para clarificar que es un parámetro para 
{\tt map} y no una llamada directa de la subrutina.
\index{sigil}
\index{ampersand sigil}

Si el nombre del array a la izquierda y a la derecha de la 
asignación es el mismo, entonces la modificación es hecha
``en lugar,`` i.e., lo cual aparece ser como si el array original
es modificado en el proceso.

Esta es una función inmensamente poderosa y expresiva; regresaremos
a ella más tarde.

\subsection{Filtrando los Elementos de una Lista}

Otra operación común de las listas es seleccionar algunos elementos
de la lista y devolver una sublista. Por ejemplo, la siguiente
función toma una lista de cadenas de texto y devuelve una lista
que contiene solamente las cadenas de texto que contienen
una vocal:

\begin{lstlisting}
sub contiene-vocal(Str $cadena) {
    return True if $cadena ~~ /<[aeiouy]>/;
}
sub filtrar_palabras_con_vocales (@cadenas) {
    my @cadena-retenida;
    for @cadena -> $cad { 
        push @cadena-retenida, $cad if contiene-vocal $cad;
    }
    return @cadena-retenida;
}  
\end{lstlisting}
%

La subrutina {\tt contiene-vocal} devuelve {\tt True} si 
la cadena de texto contiene por lo menos una vocal (para 
nuestro propósito, consideramos que ``y`` es una vocal).

La subrutina \verb|filtrar_palabras_con_vocales| devolverá 
una lista de cadenas de texto que contienen por lo menos
una vocal.

Una operación como la que \verb|filtrar_palabras_con_vocales| lleva
a cabo es conocida como un {\bf filtro} (\emph{filter} en inglés) 
porque selecciona algunos de los elementos y filtra el resto.
\index{filter}

Perl tiene una función llamada {\tt grep} que hace lo mismo
en una sola sentencia:
\index{grep}

\begin{lstlisting}
my @filtradas = grep { /<[aeiouy]>/ }, @entrada;
\end{lstlisting}
%

El nombre de la función integrada {\tt grep} usada
para filtrar algunas entradas viene del mundo de Unix,
donde es una utilidad que filtra las líneas de una archivo
de texto que coinciden con un patrón dado. 

En el código de arriba, todas las cadenas de texto de 
\verb|@entrada| serán comparadas contra el bloque {\tt grep},
y aquellas que coinciden con el regex serán almacenadas
en el array \verb|@filtradas|. Al igual que {\tt map},
la función {\tt grep} asigna, de forma iterativa, cada elemento
del array \verb|@entrada| a la variable tópico \verb|$_|, aplica 
el bloque de código después de la palabra clave {\tt grep} a \verb|$_|, 
y devuelve una lista de los valores para los cuales el bloque
de código evalúa como verdadero. Aquí, el bloque de código es un 
simple regex que se aplica a la variable \verb|$_|.

Al igual que {\tt map}, podríamos haber usado una referencia
de función como el primer argumento para {\tt grep}:

\begin{lstlisting}
my @filtradas = grep &contiene-vocal, @entrada;
\end{lstlisting}
%

Para generar una lista de números pares entre 1 y 10, podríamos
usar el operador de rango para generar números entre 1 y 10 y
usar la función {\tt grep} para filtrar los números impares:

\begin{lstlisting}
my @pares = grep { $_ %% 2 }, 1..10;  # -> [2 4 6 8 10]
\end{lstlisting}
%

\label{exercise_squares}
Como ejercicio, escribe un programa usando la función
{\tt map} para producir una array que contenga el cuadrado
de los números de la lista de entrada y un programa que use 
{\tt grep} para mantener solo los números de la lista de entrada
que son cuadrados perfectos. Solución: \ref{sol_exercise_squares}.

Muchas de las operaciones de listas pueden ser expresadas como una
combinación de {\tt map}, {\tt grep}, y {\tt reduce}.
\index{map}
\index{grep}
\index{reduce}

\subsection{Funciones de Orden Superior y Programación Funcional}
\label{array_functional_programming}
\index{higher-order function}
\index{functional programming}

Además de sus utilidades inmediatas, las funciones {\tt map},
{\tt grep}, y {\tt reduce} que usamos aquí  hacen algo 
cualitativamente nuevo. Los argumentos de estas funciones 
no son solo datos: su primer argumento es un bloque de
código o una función. No solo estamos pasando los datos
que tendrán que usar, pero también pasamos el código
que procesará los datos.

\index{reduce function}
\index{map function} 
\index{grep function}

Las funciones {\tt reduce}, {\tt map}, y {\tt grep} son 
usualmente conocidas como funciones de orden superior porque
ellas no solo manipulan datos, sino que también otras funciones.
Se puede pensar de estas funciones como funciones 
genéricas abstractas---ellas realizan una operación puramente
técnica: procesan los elementos de una lista y aplican 
a cada de ellos un comportamiento definido en el bloque de
código o la función del primer parámetro.

Estas ideas están basadas en gran medida en la programación
funcional, un paradigma de programación que es muy diferente
a lo que hemos visto hasta ahora y que ha sido implementado
históricamente en lenguajes tales como Lisp, Caml, Ocaml,
Scheme, Erlang o Haskell. Perl~6 no es un lenguaje de 
programación funcional en el mismo sentido que estos lenguajes,
porque puede usar otros paradigmas de programación, pero 
ha incorporado la mayoría de las funciones útiles. Por lo tanto,
puedes usar el poder expresivo e inherente seguridad de este 
modelo de programación sin estar forzado a hacerlo y
cuando prefieres un modelo diferente. Y todo esto lo 
puedes hacer sin tener que aprender una sintaxis
totalmente nueva la cual puede lucir un poco asbtrusa 
o hasta torpe.

Esto es inmensamente útil y puede darte un increíble poder
expresivo para resolver ciertos tipos de problema.
Pero otros tipos de problemas podrían resolverse mejor
con el modelo tradicional procedimental o imperativo,
mientras que otros problemas se pueden beneficiar 
de una perspectiva orientada a objetos. Perl~6 te deja
elegir el modelo de programación que quieras usar, y hasta hace 
posible la combinación de varios paradigmas en el mismo
programa.

La programación funcional es tan importante en mi opinión
que he dedicado un capítulo completo de este libro a las
capacidades de la programación funcional en Perl 
(ver Capítulo~\ref{functional programming}). Antes de leer
ese capítulo, asegúrate de leer la Subsección~\ref{functional_queue}
en la sección de arrays y listas en el capítulo sobre las
soluciones de los ejercicios.

\section{Arrays de Tamaños Fijos y Tipados}
\index{shaped array}
\index{fixed-size array}
\index{typed array}

Por defecto, arrays pueden contener elementos de cualquier
tipo, incluyendo elementos de diferentes tipos, y se pueden
auto-extender según lo necesites. Perl se encargará de 
los detalles innecesarios por ti, para que no preocupes
por ellos. Esto es bien práctico pero viene con su costo:
algunas operaciones de array pueden ser inesperadamente lentas,
debido a que Perl puede tener que realizar un poco de limpieza
entre bastidores, tales como gestionar la memoria, copiar un array 
completo a la memoria, etc.  

En algunos casos, sin embargo, es posible saber de antemano
el tamaño de un array, y el tipo de sus elementos. Si Perl
tiene esta información, podría ser capaz de funcionar más
rápido y usar muchísima menos memoria. También esto podría 
ayudar a prevenir errores sutiles.

Para declarar el tipo de los elementos de un array, solo
necesitas especificarlo cuando declaras el array. Por ejemplo,
para declarar un array de números enteros:

\begin{lstlisting}
> my Int @números = 1..20;
[1 2 3 4 5 6 7 8 9 10 11 12 13 14 15 16 17 18 19 20]
> @números[7] = 3.5;     # ERROR
Type check failed in assignment to @números; expected Int but got Rat
  in block <unit> at <unknown file> line 1
\end{lstlisting}
%

Del mismo modo, puedes declara el tamaño de un array. Arrays 
con un tamaño fijo son conocidos como \emph{arrays moldeados} 
(del inglés ``shaped arrays``). Hay doce meses en un año, así que 
podrías decirle a Perl que tu array \verb|@meses| nunca tendrá 
más de doce elementos:
\index{shaped array}
\begin{lstlisting}
> my @meses[12] = 1..7;
[1 2 3 4 5 6 7 (Any) (Any) (Any) (Any) (Any)]
> say @meses.elems
12
> say @meses[3];
4
> say @meses[12];
Index 12 for dimension 1 out of range (must be 0..11)
\end{lstlisting}
%

Aquí, Perl destinó 12 ``espacios`` para el array, aunque los
último cinco están actualmente indefinidos. Perl no necesita
destinar memoria cuando tú definas el décimo artículo del array.
Y Perl te dejará saber tu equivocación cuando accidentalmente 
trates de acceder un elemento que está fuera de rango.
\index{out-of-range error}

La definición del tipo de elementos y el tamaño máximo del array
puede conducir a un mayor rendimiento en términos de la
velocidad de ejecución (al menos para algunas operaciones) y 
reducir significativamente el consumo de memoria por el 
programa, especialmente cuando se manipulan arrays grandes.

\section{Arrays Multidimensionales}
\index{multidimensional array}
\index{array!multidimensional}
\label{multidimensional_array}

Los array que hemos visto hasta ahora han sido 
de una sola dimensión. En algunos lenguajes, tales arrays
son conocidos como vectores. Pero los arrays también 
pueden ser multidimensionales (puedes llamarlos matrices).

Por ejemplo, tú podrías usar un array de dos dimensiones 
para almacenar una lista de empleados con sus respectivos
salarios:

\begin{lstlisting}
> my @empleados;
[]
> @empleados[0;0] = "Liz";
Liz
> @empleados[0;1] = 3000;
3000
> @empleados[1] = ["Bob"; 2500];
[Bob 2500]
> @empleados[2] = ["Jack"; 2000];
[Jack 2000]
> @empleados[3] = ["Betty"; 1800];
[Betty 1800]
> say @empleados[1;1];
2500
> say @empleados[2];
[Jack 2000]
> say @empleados;
[[Liz 3000] [Bob 2500] [Jack 2000] [Betty 1800]]
\end{lstlisting}

Es posible tener más de dos dimensiones. Por ejemplo, 
podríamos tener un array con tres dimensiones para almacenar
las temperaturas de un reactor químico, medidas en varios lugares
identificados por sus coordinadas x, y y z:

\begin{lstlisting}
my @temp;
@temp[1;3;4] = 80;
\end{lstlisting}

Sin embargo, para este tipo de datos, es mejor usar la estructura
de datos que cubriremos en el siguiente capítulo, los hashes.

Los arrays multidimensionales pueden también tener un tamaño fijo.
Por ejemplo, esta puede ser la declaración de un array de dos dimensiones
donde la primera dimensión es el mes del año y la segunda es el 
número de días del mes:

\begin{lstlisting}
my @fecha[12, 31];
\end{lstlisting}


\section{Ordenando Arrays o Listas}
\label{sorting}
\index{sorting!data}

El ordenamiento (\emph{sorting} en inglés) de datos es 
una operación muy común en la ciencia de la computación.
Perl tiene una función {\tt sort} que puede ordenar
un array o una lista y devolver el resultado ordenado:
\index{sort}
\index{sort!function or method}
\index{function!sort}

\begin{lstlisting}
say sort <4 6 2 9 1 5 11>;  # -> (1 2 4 5 6 9 11)
\end{lstlisting}

Existen varios tipos de ordenamiento. Los más comunes
son los ordenamientos numérico y lexicográfico (o alfabético).
Ellos difieren en la manera en que compara los elementos
individuales a ser ordenados.
\index{numeric sort}
\index{alphabetic sort}
\index{lexicographic sort}
\index{sort!numeric}
\index{sort!alphabetic}
\index{sort!lexicographic}

En el ordenamiento alfabético, tú debes comparar la primera
letra de las palabras a ser comparadas; una palabra 
que comienza con una ``a`` siempre vendrá antes que una palabra
que comience con una ``b`` (o cualquier otra letra) en
orden ascendente, sin importar el valor o número de los 
otros caracteres. Tú necesitas comparar el segundo carácter
de las dos palabras solo si el primer carácter de las palabras
es el mismo.

El ordenamiento numérico es muy diferente: el valor de
interés es el valor total del número. Por ejemplo, si estamos
ordenando números enteros, 11 es es mayor que 9 porque tiene 
más dígitos. Sin embargo, el ordenamiento alfabético de 9 y 11
considera a 11 menor que 9, porque el primer dígito es menor.

Por lo tanto, el ordenamiento alfabético o lexicográfico de la
lista de enteros anterior devuelve:

\begin{lstlisting}
(1 11 2 4 5 6 9)
\end{lstlisting}

La consecuencia es que, con muchos lenguajes de programación,
cuando quieres ordenar datos, tú necesitas especificar qué
tipo de ordenamiento quieres. Con datos consistentes (
cada artículo del mismo tipo), Perl~6 es lo suficientemente
astuto para encontrar qué tipo de ordenamiento es apto para
tus necesidades. Así que, por ejemplo, este código hará
el tipo de ordenamiento que esperas:

\begin{lstlisting}
say sort <ac a bc ab abc cb ca>; # ->(a ab abc ac bc ca cb)
\end{lstlisting}

Hasta con tipos de datos mezclados, {\tt sort} puede hacer 
un gran trabajo al proveer un resultado que puede ser lo que
esperas:

\begin{lstlisting}
say sort <1 b 11 5 cb 4 12 a ab abc 42 ac bc ca >;
         # -> (1 4 5 11 12 42 a ab abc ac b bc ca cb)
\end{lstlisting}

Existen otros casos donde el simple uso de la función
{\tt sort} fallará y no devolverá lo que probablemente
esperas:

\begin{lstlisting}
say sort <a ab abc A bc BAC AC>; # -> (A AC BAC a ab abc bc)
\end{lstlisting}
%

En este caso, {\tt sort} pone todas las cadenas de texto
que comienzan con una letra mayúscula antes que cualquier 
cadena de texto que comienza con una letra minúscula. Probablemente
no deseas esto. Esto luce peor si las cadenas de texto usan
caracteres ASCII extendidos:
\index{case!lower}
\index{case!upper}

\begin{lstlisting}
say sort <a ab àb abc Ñ A bc BAC AC>;
        # -> (A AC BAC a ab abc bc Ñ àb)
\end{lstlisting}
%

\index{sort!ASCIIbetical}
La razón detrás de esto es que, al ordenar cadenas de texto,
la función {\tt sort} usa la codificación numérica interna
de las letras. Esto fue llamado alguna vez orden 
"ASCIIbetical" (en contraste con orden alfabético),
pero el término es ahora muy limitado y algo obsoleto,
porque Perl~6 usa Unicode y no ASCII.

Claramente, hay casos donde más técnicas avanzadas 
de ordenamiento son necesarias.

\section{Más Técnicas Avanzadas de Ordenamiento}
\label{advanced_sort}
\index{sorting!advanced}

\index{sort}
\index{sort!code object}
\index{cmp operator}
\index{operator!cmp}
La rutina {\tt sort} típicamente toma dos argumentos,
un objeto de código y una lista de artículos a ser ordenados,
y devuelve una nueva lista ordenada. Si no se especifica un
objeto de código, como en el ejemplo que vimos más arriba,
el operador integrado de comparación {\tt cmp}
se usa para comparar los elementos. Si el objeto de código
es proveído, entonces se usa en la comparación,
la cual informa a {\tt sort} cuál de los elementos
debería estar primero en el orden final.

Existen tres operadores integrados de comparación que pueden
usarse para ordenar. Algunas veces son conocidos como los
comparadores de tres sentidos porque ellos comparan sus operandos
y devuelven un valor que informa si el primer operador debería
ser considerado menor que, igual a o mayor que el segundo operador
para el propósito de determinar el orden en que estos operadores
deben ser ordenados.  El operador {\tt leg} coacciona sus argumentos
a cadenas de texto y realiza una comparación lexicográfica. El operador
\verb|<=>| coacciona sus argumentos a números (Real) y realiza una
comparación numérica. El operador {\tt cmp} es el comparador ``inteligente``
de tres sentidos, el cual compara cadenas de texto con semántica de 
cadenas de texto y números con semántica de números.
\index{coercion}

La mayoría de nuestros simples ejemplos funcionó bien
con cadenas de texto y números porque ellos implícitamente
usaron el operador {\tt cmp} por defecto, el cual ``adivina``
el tipo de comparación a realizarse.

\index{three-way comparator}
\index{leg operator}
\index{cmp operator}
\index{operator!cmp}
\index{operator!leg}
% TODO: get these entries working in plastex
\ifplastex \else
\index{<=> operator@\texttt{<=>} operator}
\index{operator!<=> (numeric comparison)@\texttt{<=>} (numeric comparison)}
\fi


En otras palabras, esto:

\begin{lstlisting}
say sort <4 6 2 9 1 5 11>;  # -> (1 2 4 5 6 9 11)
\end{lstlisting}

es equivalente a esto:

\begin{lstlisting}
say sort { $^a cmp $^b }, <4 6 2 9 1 5 11>;
     # -> (1 2 4 5 6 9 11)
\end{lstlisting}

\index{placeholder}
El bloque de código usado aquí como el primer argumento
de la función {\tt sort} usa los parámetros marcadores
(parámetros auto-declarados de posición) los cuales
vimos anteriormente en este capítulo. El operador
{\tt cpm} recibe dos argumentos y le pasa a la
función {\tt sort}  información sobre cuál de los 
artículos debería aparecer primero en el orden 
resultante.
\index{cmp operator}
\index{operator!cmp}

\index{sort!reverse order}
Si quieres ordenar en el orden inverso, podrías solo
intercambiar el orden de los parámetros marcadores:

\begin{lstlisting}
say sort { $^b cmp $^a }, <4 6 2 9 1 5 11>;
     # -> (11 9 6 5 4 2 1)
\end{lstlisting}

Ten presente que discutimos este ejemplo solo con el
propósito de explicar algunas funciones de los 
parámetros marcadores. Para ordenar el array que hemos
presentado aquí en orden descendente, sería más fácil 
obtener el mismo resultado con el siguiente código:

\begin{lstlisting}
say reverse sort <4 6 2 9 1 5 11>;  # -> (11 9 6 5 4 2 1)
\end{lstlisting}
\index{reverse}

La razón por la cual {\tt sort} hace un trabajo tan grandioso
hasta con una mezcla de cadenas de texto y números es porque 
la función de comparación por defecto, {\tt cmp}, es muy astuta 
y puede adivinar si debería realizar una comparación de orden
lexicográfico u orden numérico al examinar sus argumentos.

Si el ordenamiento se complica para {\tt cmp}, o generalmente
cuando se requiere un ordenamiento específico o personalizado,
entonces debes crear tu propia subrutina ad-hoc de comparación.
\index{cmp operator}
\index{operator!cmp}

Por ejemplo, si tomamos de nuevo el ejemplo de las cadenas
de texto con letras mayúsculas y minúsculas, podemos lograr un
ordenamiento alfabético que no es sensible a la diferencia
entre letras mayúsculas y minúsculas:
\index{sort!case insensitive}

\begin{lstlisting}
say sort { $^a.lc cmp $^b.lc}, <a ab abc A bc BAC AC>;
     # -> (a A ab abc AC BAC bc)
\end{lstlisting}

or this way:
\begin{lstlisting}
say sort { $^a.lc leg $^b.lc}, <a ab abc A bc BAC AC>;
     # -> (a A ab abc AC BAC bc)
\end{lstlisting}

\index{lc function or method}
\index{function!lc}
\index{method!lc}
\index{lower case!lc function}
Cuando el bloque de código de comparación recibe sus dos 
argumentos, el método {\tt lc} los convierte en letras
minúsculas antes de realizar la comparación. Nota qu esto
no tiene ningún impacto en el caso de la salida de texto,
dado que la transformación a letra minúscula es local
al bloque de código de comparación y no tiene impacto
alguno en los datos manejados por {\tt sort}. Veremos
en un rato que hay una manera más fácil y eficiente de
realizar tal transformación antes de comparar los argumentos.

Si la especificación de la comparación es muy complicada,
podría ser necesario escribirla en una subrutina separada
y dejar que {\tt sort} llame a esa subrutina. Supón que 
tenemos una lista de cadenas de texto que están todas formadas
por dígitos iniciales, los cuales son seguidos por un grupo
de letras y posiblemente seguidos por otros caracteres irrelevantes,
y queremos ordenar las cadenas de texto de acuerdo al grupo de
letras que siguen los dígitos.

Comencemos con escribir la subrutina de comparación:
\index{comparison subroutine}
\index{sort!comparison subroutine}


\begin{lstlisting}
sub mi_comp ($cad1, $cad2) {
    my $cmp1 = $0 if $cad1 ~~ /\d+(\w+)/; 
    my $cmp2 = $0 if $cad2 ~~ /\d+(\w+)/; 
    return $cmp1 cmp $cmp2;
}
\end{lstlisting}

Nada complicada: la subrutina toma dos argumentos, usa
una regex para extraer el grupo de palabras de cada 
argumento, y devuelve el resultado de las cadenas 
extraídas a la función {\tt cmp}. En el mundo real,
se necesitaría hacer algo si la extracción falla, pero,
para nuestro propósito, asumiremos que esto no sucederá.
\index{regex}
\index{cmp operator}

El ordenamiento es ahora bastante directo, solo 
necesitamos pasar a la función {\tt sort} la
función anterior:

\begin{lstlisting}
say sort &mi_comp, < 22ac 34bd 56aa3 12c; 4abc( 1ca 45bc >;
     # -> (56aa3 4abc( 22ac 45bc 34bd 12c; 1ca)
\end{lstlisting}

Solo necesitamos preceder la subrutina de comparación
con el sigilo ``\&`` y funcionará bien: las cadenas de texto
son ordenadas de acuerdo a los grupos de letras que siguen los
dígitos al inicio de cada cadena de texto.
\index{sigil}
\index{ampersand sigil}

En todos los ejemplos anteriores, la subrutina de
de comparación aceptó dos parámetros, los artículos
a ser comparados. La función {\tt sort} puede funcionar
con un objeto de código que toma un solo parámetro. En 
tal caso, el objeto de código no es un bloque 
de código de comparación, sino que es un objeto de código
que implementa la transformación a ser aplicada a los
artículos antes de usar la función de comparación por
defecto {\tt cmp}. 
\index{sort!transformation subroutine}
\index{cmp operator}
\index{operator!cmp}

Por ejemplo, si tomamos una vez más el ejemplo de las cadenas
de texto mezcladas con letras mayúsculas y minúsculas, 
podemos lograr un ordenamiento alfabético que no es 
sensible a la diferencia entre letras mayúsculas y minúsculas
en una nueva forma:
\index{sort!case insensitive}

\begin{lstlisting}
say sort { $_.lc }, <a ab abc A bc BAC AC>;
     # -> (a A ab abc AC BAC bc)
\end{lstlisting}

Esto podría también escribirse con un parámetro marcador:
\index{placeholder}

\begin{lstlisting}
say sort { $^a.lc }, <a ab abc A bc BAC AC>;
     # -> (a A ab abc AC BAC bc)
\end{lstlisting}

En este ejemplo, debido a que el bloque de código
de comparación toma un solo argumento, su tarea es
transformar cada uno de los artículos a ser comparados
antes de aplicar la función {\tt cmp}  a los argumentos.
Esto no solo las cosas más simples, sino que también
es probablemente más eficaz, especialmente si 
el número de artículos a ser ordenados es grande y si
la subrutina de transformación es relativamente costosa
(en término de procesamiento): los valores transformados
son actualmente almacenados en \emph{caché} (i.e.,
almacenados en la memoria para uso repetido), de tal manera
que la transformación se hace solamente una vez por cada artículo,
a pesar de que la subrutina de comparación se llama muchas veces por
cada artículo en un ordenamiento.
\index{cache}

De igual manera, podríamos ordenar los números de acuerdo 
a sus valores absolutos:
\index{abs function or method}
\index{function!abs}
\index{method!abs}

\begin{lstlisting}
say sort {$_.abs}, <4 -2 5 3 -12 42 8 -7>; # -> (-2 3 4 5 -7 8 -12 42)
\end{lstlisting}

El ejemplo ``más complejo`` con dígitos y letras que requiere
una subrutina separada aplica la misma transformación
a ambos argumentos. Como ejercicio, escribe un simple programa
de ordenamiento que usa una subrutina de transformación y
el operador de comparación por defecto {\tt cmp} sobre
los artículos transformados. Solución: \ref{sol_sort_exercise}.
\label{sort_exercise}
\index{cmp operator}
\index{sort}

Sobra decir que los usos avanzados de la función {\tt sort}
presentada en esta sección son solo más ejemplos del
estilo de programación funcional. Las subrutinas de comparación
y las subrutinas de transformación pueden pasarse como 
argumentos a la función {\tt sort}, y generalmente, todas
las funciones, las subrutinas, y los bloques de códigos
usados aquí son funciones de orden superior o funciones
de primera clase.
\index{functional programming}

\section{Depuración de Programas}
\index{debugging}

El uso descuidado de arrays ( y cualquier otros objetos
mutables) puede conducir a largas horas de depuración. 
Estas son algunas de las dificultades y las maneras de
evitarlas:

\begin{enumerate}

\item Algunas de las funciones y métodos integrados de arrays
modifican sus argumentos mientras otras no.

Puede ser tentativo escribir código así:

\begin{lstlisting}
@array = splice @array, 1, 2, $nuevo-elem;   # INCORRECTO!
\end{lstlisting}
\index{splice function}

La función {\tt splice} devuelve los elementos
que ha eliminado del array, no el array en sí mismo,
el cual es modificado ``en el lugar``.

Antes de usar los métodos y operadores de arrays,
deberías leer la documentación cuidadosamente y quizás
probarlos en modo interactivo.

Al recorrer un array, por ejemplo con {\tt for}
o {\tt map}, la variable tópico \verb|$_| es
un apodo para los elementos sucesivos del array,
y no una copia de ellos. Esto significa que si
cambias a \verb|$_|, el cambio sera reflejado en
el array. Pueden haber casos donde este comportamiento
es deseable y otros casos donde no importa lo que suceda
con ellos (porque ya no necesitas el array original),
pero esta técnica es propensa a errores y quizás debería
evitarse (o por lo menos usarla con precaución).

\begin{lstlisting}
my @números = <1 2 3>;
push @dobles, $_*=2 for @números;  # INCORRECTO (probablemente)
say @números; # -> [2 4 6]
\end{lstlisting}

El error aquí es que la sentencia \verb|$_*=2| está
modificando a \verb|$_|, así que el array \verb|@números|
es también modificado, mientras que la intención era 
ciertamente poblar los nuevos números en el array
\verb|@dobles|, no modificar a \verb|@números|.

El mismo código aplica a una lista literal en vez de 
un array lo cual conduce a un error al tiempo de 
ejecución, debido a que una lista es inmutable:

\begin{lstlisting}
> push @dobles, $_*=2 for <1 2 3>; # INCORRECTO (definitivamente)
Cannot assign to an immutable value
\end{lstlisting}

La solución en este caso es bien fácil y consiste
de usar una expresión que no modifique a \verb|$_|
pero devuelva el valor deseado:

\begin{lstlisting}
push @dobles, $_ * 2 for @números; # OK
\end{lstlisting}

Lo mismo aplica a {\tt map}:

\begin{lstlisting}
my @números = <1 2 3>;
say map { ++$_}, @números;          # INCORRECTO (probablemente)
say @números; # -> [2 3 4]
\end{lstlisting}
%

Otra vez, el uso de una expresión que no modifique a \verb|$_|
pero que en cambio, devuelva el nuevo valor deseado arreglará
el problema:

\begin{lstlisting}
my @números = <1 2 3>;
say map { $_ + 1}, @números;        # -> (2 3 4)
say @números; # -> [1 2 3]
\end{lstlisting}
%

\end{enumerate}



\section{Glosario}

\begin{description}

\item[Lista] Una secuencia inmutable de valores.
\index{list}

\item[Array] Una variable que contiene una secuencia mutable
de valores.
\index{list}

\item[Elemento] Uno de los valores en una lista o array (
o cualquier otra secuencia), también llamado artículos.
\index{element}
\index{item}

\item[Array anidado] Un array que es un elemento de otro array.
\index{nested list}

\item[Acumulador] Una variable que se usa en un bucle 
para agregar o acumular un resultado.
\index{accumulator}

\item[Reducir] Un patrón de procesamiento que recorre
una secuencia y acumula los elementos en un solo
resultado.
\index{reduce pattern}
\index{pattern!reduce}

\item[Map] A processing pattern that traverses a 
sequence and performs an operation on each element. 
Also the name of a Perl built-in function that performs 
such a processing pattern.
\index{map pattern}
\index{pattern!map}

\item[Filtro] Un patrón de procesamiento que recorre una
lista y selecciona los elementos que satisfacen una condición.
{\tt grep} es una implementación de Perl de un filtro.
\index{filter pattern}
\index{pattern!filter}

\item[Apodo] Una circunstancia donde un identificador
se refiere directamente a alguna variable o valor, 
de tal manera que un cambio del identificador
se traduce en un cambio de la variable o valor. 
Esto significa tener dos nombres para el mismo valor,
contenedor u objeto.
\index{alias}

\end{description}


\section{Ejercicios}
\label{array_exercises}

\begin{exercise}

Escribe una subrutina llamada \verb|suma-anidada| que toma un
array de arrays de enteros y añade todos los elementos de los
arrays anidados. Por ejemplo:
\label{nested_sum}

\begin{lstlisting}
my @AoA = [[1, 2], [3], [4, 5, 6]];
say suma-anidada(@AoA);        # -> 21
\end{lstlisting}

Solución: \ref{sol_nested_sum}.

\end{exercise}

\begin{exercise}
\label{cumulative}
\index{cumulative sum}

Escribe una subrutina llamada \verb|suma-acumula| que toma una lista
de números y devuelve la suma acumulativa, es decir, una nueva lista
donde la suma del elemento $i$th es la suma de los primeros elementos
$i+1$ de la lista original. Por ejemplo:
\label{cumsum}

\begin{lstlisting}
my @nums = [1, 2, 3, 4];
say suma-acumula(@nums);           # -> [1, 3, 6, 10]
\end{lstlisting}

Solución: \ref{sol_cumsum}.

\end{exercise}

\begin{exercise}

Escribe una subrutina llamada \verb|medio| que toma una lista y devuelve
una lista nueva que contiene todos los elementos excepto el primero y el
último. Por ejemplo:
\label{middle}

\begin{lstlisting}
say medio(1, 2, 3, 4);      # -> (2, 3)
\end{lstlisting}

Solución: \ref{sol_middle}.

\end{exercise}

\begin{exercise}

Escribe una subrutina llamada \verb|corta| que toma un
array, lo modifica al remover el primer y último elemento y
devuelve algo no útil. Por ejemplo:
\label{chop}

\begin{lstlisting}
my @nums = 1, 2, 3, 4;
corta(@nums);
say @nums;                   # -> [2, 3]
\end{lstlisting}

Solución: \ref{sol_chop}.

\end{exercise}


\begin{exercise}
Escribe una subrutina llamada \verb|está-ordenada| que toma
una lista o un array de números como parámetro y devuelve {\tt True}
si la lista está ordenada en orden ascendente y {\tt False} 
de lo contrario. Por ejemplo:
\label{is_sorted}
\index{is-sorted}

\begin{lstlisting}
> está-ordenada (1, 2, 2);
True
> está-ordenada (1, 2, 1);
False
\end{lstlisting}

Solución: \ref{sol_is_sorted}.

\end{exercise}


\begin{exercise}

\label{is_anagram}
\index{anagram}
\index{is-anagram}
Una palabra es un anagrama de otra si las dos tienen las mismas letras, 
con el mismo número de apariciones, pero en un orden diferente.
Escribe una subrutina llamada \verb|es-anagrama| que toma dos
cadenas de texto y devuelve {\tt True} si son anagramas.

Solución: \ref{sol_is_anagram}.

\end{exercise}



\begin{exercise}
\label{has_duplicates}
\index{duplicate}
\index{has-duplicates}
\index{uniqueness}

Escribe una subrutina llamada \verb|tiene-duplicados| que toma
una lista o un array  y devuelve {\tt True} si hay algún elemento
que aparece más de una vez. La subrutina no debería
modificar la entrada original.

Solución: \ref{sol_has_duplicates}.

\end{exercise}


\begin{exercise}

Este ejercicio es acerca de la Paradoja del Cumpleaños.
Encuentra más información sobre esta paradoja aquí: 
\url{https://es.wikipedia.org/wiki/Paradoja_del_cumpleaños}.
\index{birthday paradox}
\label{birthdays}

Si hay 23 estudiantes en tu clase, cuál es la probabilidad
de que dos estudiantes cumplan años el mismo día? 
Puedes estimar esta probabilidad al generar muestras 
aleatorias de los 23 cumpleaños y chequear por 
cumpleaños duplicados. Pista: puedes generar cumpleaños 
aleatorios con la funciones {\tt rand} y {\tt int}.
\index{rand function}
\index{function!rand}

Solución: \ref{sol_birthdays}.

\end{exercise}



\begin{exercise}

\label{push_unshift}
Write a subroutine that reads the file \emph{words.txt} and builds
a list with one element per word.  Write two versions of
this function, one using the {\tt push} method and the
other using the idiom {\tt unshift}.  Which one takes
longer to run?  Why?
\index{push function}
\index{unshift function}

Solución: \ref{sol_push_unshift}.


\end{exercise}


\begin{exercise}
\label{bisection}
\index{membership!bisection search}
\index{bisection search}
\index{search!bisection}
\index{membership!binary search}
\index{binary search}
\index{search!binary}
\index{half-interval search}

Para chequear si una palabra está en nuestra lista 
estándar de palabras, podrías chequear cada elemento
por turno, pero sería lento porque buscaría a través 
de las palabras en orden.

Si las palabras están en orden alfabético (el cual es
el caso en nuestra lista de palabras), podemos 
acelerar las cosas con una \emph{búsqueda binaria},
la cual es similar a lo que haces cuando buscas una
palabra en el diccionario. Tú comienzas en alguna parte
en el medio del diccionario y chequea si la palabra que
estás buscando se encuentra antes que la palabra en
el medio de la lista. Si esto sucede, busca en la 
primera mitad de la lista del mismo modo. Por lo
contrario, busca en la segunda mitad de la lista.

De cualquier manera, tú reduces el tiempo de búsqueda
por la mitad. Si la lista de palabras contiene 113,809 
palabras, tomará alrededor de 17 pasos para encontrar la 
palabra o concluir que la palabra no se encuentra en 
la lista.

Escribe una función llamada \verb|búsqueda-binaria|
que toma una lista ordenada y un valor a encontrar en
la lista. La función debe devolver si el valor
se encuentra en la lista o no.
\index{bisect}

Solución: \ref{sol_bisection}

\end{exercise}

\begin{exercise}
\index{reverse word pair}
\label{reverse_pair}

Dos palabras son una ``pareja de inversas`` si cada una es la
inversa de la otra. Por ejemplo, ``depot`` y ``toped`` forman
una pareja de inversas.; otro ejemplo incluye ``reward`` y ``drawer``,
o ``desserts`` y ``stressed``.

Escribe un programa que encuentra todas las parejas de inversas en 
el archivo \emph{word.txt}.


Solución: \ref{sol_reverse_pair}.

\end{exercise}

\begin{exercise}
\index{interlocking words}
\label{interlock}

Dos palabras se ``entrelazan`` si al tomar letras alternadas de
ambas palabras se forma una nueva palabra. Por ejemplo, ``shoe``
y ``cold`` se entrelazan para formar la palabra ``schooled``.

Escribe un programa que encuentre todos los pares de palabras
que se entrelazan en el archivo \emph{word.txt}.
Pista: !`no enumeres todas la parejas, hay demasiadas!

Solución: \ref{sol_interlock}

Crédito: este ejercicio fue inspirado por un ejemplo en \url{http://puzzlers.org}.

\end{exercise}


