\chapter{The Way of the Program}

La meta de este libro es enseñarte a pensar como un científico de la 
computación. Esta manera de pensar combina algunas de las mejores características
de las matemáticas, la ingeniería, y las ciencias naturales. Al igual que los
matemáticos, los científicos de la computación usan lenguajes formales para denotar
ideas (específicamente computaciones). Al igual que los ingenieros, ellos diseñan cosas,
ensamblan componentes en sistemas y evalúan las compensaciones entre las alternativas.
Al igual que los científicos, ellos observan el comportamiento de sistemas complejos, 
formulan hipótesis, y prueban las predicciones. 
\index{problem solving}
\index{formal language}

La habilidad más importante de un científico de la computación
es la {\bf resolución de problemas}. La resolución de problemas se refiere
a la habilidad de formular problemas, pensar creativamente sobre las
soluciones, y expresar una solución clara y precisamente. Como sucede, 
el proceso de aprender a programar es una oportunidad excelente para practicar 
las habilidades de resolución de problemas. Por esta razón, este capítulo se titula,
 ``La Manera del Programa.``

En un nivel, aprenderás a programar, la cual es una habilidad útil por sí misma.
En otro nivel, aprenderás a usar la programación como un medio para un fin. 
A medida que avancemos, este fin se volverá más claro.
\index{programming}


\section{¿Qué es un Programa?}

Un {\bf programa} es una secuencia de instrucciones que especifica
como realizar una computación. La computación podría involucrar 
algo matemático, tal como resolver un sistema de ecuaciones o 
encontrar las raíces de un polinomio, pero también puede ser algo
simbólico, tal como buscar y reemplazar texto en un documento, o 
algo gráfico, como el procesamiento de una imagen o la reproducción 
de un video.
\index{program}

Los detalles lucen diferentes en lenguajes distintos, pero hay algunas
instrucciones básicas que aparecen en casi todos los lenguajes:

\begin{description}

\item[Entrada] Obtener datos desde el teclado, un archivo, la red, 
un sensor, un chip de GPS o algún otro dispositico.
\index{input}

\item[Salida] Mostrar información en la pantalla, guardarla en un archivo,
 enviarla a través de la red, actuar en un dispositico mecánico,  etc.
\index{output}

\item[Matemática] Realizar operaciones matemáticas básicas tales como la adición y
la multiplicación.

\item[Ejecución condicional] Revisar ciertas condiciones y ejecutar el código 
apropiadamente.
\index{conditional!execution}

\item[Repetición] Realizar alguna acción repetidamente, usualmente con alguna
forma de variación.
\index{repetition}

\end{description}

Lo creas o no, eso es todo. Cada programa que has usado, sin importar
que tan complicado sea, está compuesto de instrucciones que lucen exactamente
como estas. Por lo tanto, puedes imaginarte la programación como el 
proceso de fragmentar una tarea grande y compleja en piezas más pequeñas
las cuales se pueden ejecutar con una de estas operaciones básicas.
\index{instruction}

\index{abstraction}
\index{subtask}
Al usar o llamar estas piezas, es posible crear varios niveles
de \emph{abstración}. Probablemente te han dicho que las computadoras
solamente usan 0s y 1s al nivel más fundamental; pero usualmente no 
nos preocupamos acerca de esto. Cuando usamos un procesador de texto para
escribir una carta o un reporte, estamos interesados en archivos con textos
y algunas instrucciones de formato, y con comandos para cambiar el archivo o
imprimirlo; afortunadamente, no tenemos que preocuparnos sobre los 0s y 1s; el
procesador de texto nos ofrecemos una vista más general (archivos, comandos, etc,)
que oculta todos los detalles insignificantes para el usuario. 


Similarmente, cuando escribimos un programa, 
usualmente usamos y/o creamos varias capas de abstracción, 
para que, por ejemplo, una vez que hayamos creado una tarea pequeña 
que consulta una base de datos y guarda la información relevante
en la memoria, no tenemos que preocuparnos sobre los detalles técnicos
de la tarea. Podemos usarla como una caja negra la cual realizará 
la operación deseada para nosotros. La esencia de la programación es 
en gran parte este arte de crear estas capas succesivas de abstracción 
de tal manera que realizar tareas de niveles más altos se vuelva 
relativamente más fácil.
\index{subtask}
\index{black box}


\section{Ejecutando Perl~6}
\label{running_perl_6}

Uno de los retos de comenzar con Perl~6 es que podrías tener que instalar
Perl~6 y cualquier software relacionado en tu computadora. Si estás familiarizado
con tu sistema operativo, y especialmente te sientes cómodo con shell o
intérprete de comandos, no tendrás problemas instalando Perl~6. Para los 
principiantes, pueden ser un poco difícil aprender sobre la
administración de sistema y programación al mismo tiempo.

Para evitar ese problema, podrías comenzar ejecutando Perl~6 en
el navegador. Podrías usar un motor de búsquedas para encontrar tal
sitio. Por el momento, la forma más fácil es probablemente conectarse
al sitio \url{https://glot.io/new/perl6}, donde puedes escribir código
de Perl~6 en la ventana principal, ejecutarlo, y observar el resultado 
en la ventana de salida más abajo.
\index{Perl~6 in a browser}

Tarde o temprano, sin embargo, tendrás que instalar Perl~6 en tu computadora.

La manera más fácil de instalar Perl~6 en tu sistema es descargar 
Rakudo Star (una distribución de Perl~6 que contiene el {\bf compilador}
Rakudo de Perl~6, documentación y módulos útiles): sigue las instrucciones 
para tu sistema operativo en
\url{http://rakudo.org/how-to-get-rakudo/} y en 
\url{https://perl6.org/downloads/}. 

\index{Perl 6 version}
Al momento de escribir este libro, la especificación más reciente del 
lenguaje es Perl~6 version 6c (v6.c), y el lanzamiento más reciente disponible
para la descarga es Rakudo Star 2016.07; los ejemplos en este libro deberían
funcionar con esta versión. Puedes encontrar la versión instalada al ejecutar
el siguiente comando en el intérprete de comandos de tu sistema operativo:
\begin{verbatim}
$ perl6 -v
This is Rakudo version 2016.07.1 built on MoarVM version 2016.07
implementing Perl 6.c.
\end{verbatim}

No obstante, deberías descargar e instalar las versión más reciente que puedas
encontrar. La salida (advertencias, mensages de errores, etc.) que obtendrás de tu versión de Perl
podría en algunos casos diferir de la que se encuentra en este libro, pero estas diferencias
deberían ser esencialmente cosméticas.

Comparado con Perl~5, Perl~6 no es sólo una nueva versión de Perl.
Es como la nueva pequeña hermana de Perl~5. Su objectivo no es reemplazar
al Perl~5. Perl~6 es realmente un nuevo lenguaje de programación, con una sintaxis
que es similar a las versiones anteriores de Perl (tal como Perl~5), pero considerablemente
diferente. Al menos que se indique lo contrario, este libro es sobre Perl~6 solamente, no
acerca de Perl~5 y versiones anteriores del lenguaje de programación Perl. Desde aquí adelante,
cuando hablemos de \emph{Perl} sin ninguna cualificación, nos referimos a Perl~6.


El {\bf interpretador} de Perl~6 es un programa que lee y  
ejecuta código de Perl~6. Algunas veces se le llama REPL (por ``read, 
evaluate, print, loop''). Dependiendo de tu entorno, 
podría iniciar el interpretador al cliquear un ícono, o al 
escribir {\tt perl6} en el intérprete de comandos.

Cuando comience, deberías ver algo similar a esto:
\index{interpreter}
\index{REPL}

\begin{verbatim}
To exit type 'exit' or '^D'
(Possibly some information about Perl and related software)
> 
\end{verbatim}
%

La última línea con {\tt >} es un {\bf prompt} que indica 
que el REPL está listo para que entres código. Si escribes un 
línea de código y presiona Enter, el interpretador muestra el resultado: 
\index{prompt}

\begin{verbatim}
> 1 + 1
2
>
\end{verbatim}
%
Puedes escribir {\tt exit} en el prompt de REPL para salir del REPL.

Ahora estás listo para comenzar.
De aquí en adelante, asumimos que sabes como iniciar el REPL de Perl~6 y 
ejecutar código.


\section{El Primer Programa}
\label{hello}
\index{Hello, World}

Tradicionalmente, el primer programa que escribes en un nuevo lenguaje
se llama ``Hello, World'' porque todo lo que hace es mostrar las palabras
``Hello, World.''  En Perl~6, luce de la siguiente manera:

\begin{verbatim}
> say "Hello, World";
Hello, World
>
\end{verbatim}
%
\index{say function or method}
\index{function!say}
Este es un ejemplo de lo que usualmente se conoce como una {\bf sentencia de impresión},
aunque actualmente no imprime nada sobre el papel y ni siquiera usa la palabra clave
{\tt print} 
\footnote{Perl también tiene una función {\tt print},
pero la función integrada {\tt say} es usada aquí 
porque agrega un carácter de nueva línea a la salida.}
(palabras claves son palabras que tienen un significado especial
en el lenguaje y son usadas por el interpretador para reconocer la
estructura del programa).
La sentencia print muestra un resultado en la pantalla. En este caso, 
el resultado son las palabra {\tt Hello World}.
%
Las comillas inglesas en el programa indican el comienzo y final
del texto a ser mostrado; ellas no aparecen en el resultado.
\index{quotation mark}
\index{print statement}
\index{statement!print}

El punto y medio ``{\tt ;}'' al final de la línea indica
que este es el final de la sentencia actual. Aunque un punto y medio
no es técnicamente necesario al ejecutar código simple en el REPL, 
es usualmente necesario cuando se escribe un programa con varias líneas
de código, así que prodrías ir acostumbrándote a finalizar instrucciones
de código con un punto y medio.   
\index{semi-colon}

Otros lenguajes de programación podrían requerir paréntesis
alrededor de la oración que se quiere mostrar, pero esto es usualmente 
no necesario en Perl~6.

\section{Operadores Aritméticos}
\index{operator!arithmetic}
\index{arithmetic operator}

Después de ``Hello, World,'' el siguiente paso es aritmética. Perl~6 provee {\bf operadores}, los cuales son símbolos especiales que representan computaciones tales como adición y multiplicación.

Los operadores {\tt +}, {\tt -}, {\tt *}, y {\tt /} realizan adición,, sustracción, multiplicaión y división, como en los siguientes ejemplos en el REPL:

\begin{verbatim}
> 40 + 2
42
> 43 - 1
42
> 6 * 7
42
> 84 / 2
42
\end{verbatim}
%

Dado que usamos el REPL, no necesitamos una sentencia print explícita
en estos ejemplos, debido a que el REPL automáticamente imprime el resultado de las sentencias. En un programa real, necesitarías una sentencia print para mostrar el resultado, como veremos más adelante. Similarmente, si ejecutas sentencias de Perl en el navegador mencionado en la sección~\ref{running_perl_6}, necesitarás una sentencia print para mostrar el resultado de estas operaciones.
Por ejemplo:

\begin{verbatim}
say 40 + 2;   # -> 42
\end{verbatim}


Finalmente, el operador {\tt **} realiza potenciación; es decir que eleva un número a un exponente:

\begin{verbatim}
> 6**2 + 6
42
\end{verbatim}
%
En otros lenguages, el signo de intercalación (``\verb"^"'') o el acento circunflejo es usado para la potenciación, pero en Perl~6 se utiliza para otros propósitos.
%
\index{set}
\index{set!operator}
\index{operator!set}


\section{Values and Types}
\label{values_and_types}
\index{value}
\index{type}
\index{string}

A {\bf value} is one of the basic things a program works with, like a
letter or a number.  Some values we have seen so far are {\tt 2},
{\tt 42}, and \verb'"Hello, World"'.

These values belong to different {\bf types}:
{\tt 2} is an {\bf integer}, {\tt 40 + 2} is also an integer, 
{\tt 84/2} is a {\bf rational number},
and \verb"'Hello, World'" is a {\bf string}, so called 
because the characters it contains are strung together.
\index{integer}
\index{floating-point}

If you are not sure what type a value has, Perl can
tell you:

\begin{verbatim}
> say 42.WHAT;
(Int)
> say (40 + 2).WHAT;
(Int)
> say (84 / 2).WHAT;
(Rat)
> say (42.0).WHAT
(Rat)
> say ("Hello, World").WHAT;
(Str)
>
\end{verbatim}
%
In these instructions, {\tt .WHAT} is known as an 
introspection method, that is a kind of method which 
will tell you \emph{what} (of  which type) the preceding 
expression is. {\tt 42.WHAT} is an example of the dot 
syntax used for method invocation: it calls the {\tt .WHAT} 
built-in on the ``42'' expression (the invocant) and provides 
to the {\tt say} function the result of this invocation, 
which in this case is the type of the expression.
\index{WHAT}
\index{introspection}
\index{string!type}
\index{type!Str}
\index{Int type}
\index{type!Int}
\index{rational!type}
\index{type!Rat}
\index{invocant}
\index{invocation}

Not surprisingly, integers belong to the type {\tt Int},
strings belong to {\tt Str} and rational 
numbers belong to {\tt Rat}.  

Although {\tt 40 + 2} and {\tt 84 / 2} seem to yield the 
same result (42), the first expression returns an integer 
({\tt Int}) and the second a rational number ({\tt Rat}). 
The number 42.0 is also a rational.

The rational type is somewhat uncommon in most programming 
languages. Internally, these numbers are stored as two 
integers representing the numerator and the denominator 
(in their simplest terms). For example, the number 17.3 
might be stored as two integers, 173 and 10, meaning that 
Perl is really storing something meaning the $\frac{173}{10}$ 
fraction. Although this is usually not needed (except 
for introspection or debugging), you might access these 
two integers with the following methods:

\begin{verbatim}
> my $num = 17.3;
17.3
> say $num.WHAT;
(Rat)
> say $num.numerator, " ", $num.denominator; # say can print a list
173 10
> say $num.nude;      # "nude" stands for numerator-denominator
(173 10) 
\end{verbatim}
\index{numerator method}
\index{method!numerator}
\index{denominator method}
\index{method!denominator}
\index{nude method}
\index{method!nude}
%
This may seem anecdotal, but, for reasons which are 
beyond the scope of this book, this makes it possible for Perl~6 
to perform arithmetical operations on rational numbers with 
a much higher accuracy than most common programming languages. 
For example, if you try to perform the arithmetical operation
\verb'0.3 - 0.2 - 0.1', with most general purpose programming languages 
(and depending on your machine architecture), you 
might obtain a result such as -2.77555756156289e-17 (in Perl~5), 
-2.775558e-17 (in C under gcc), or -2.7755575615628914e-17 
(Java, Python~3, Ruby, TCL). Don't worry about these values if you 
don't understand them, let us just say that they  are 
extremely small, but they are not 0, whereas,  
obviously, the result should really be zero. In Perl~6, 
the result is 0 (even to the fiftieth decimal digit):
\begin{verbatim}
> my $result-should-be-zero = 0.3 - 0.2 - 0.1;
0
> printf "%.50f", $result-should-be-zero; # prints 50 decimal digits
0.00000000000000000000000000000000000000000000000000
\end{verbatim}
%
In Perl~6, you might even compare the result of the operation with 0:
\begin{verbatim}
> say $result-should-be-zero == 0;
True
\end{verbatim}
%
Don't do such a comparison with most common programming 
languages; you're very likely to get a wrong result.

What about values like \verb'"2"' and \verb'"42.0"'?
They look like numbers, but they are in quotation marks like
strings.
\index{quotation mark}

\begin{verbatim}
> say '2'.perl; # perl returns a Perlish representation of the invocant
"2"
> say "2".WHAT;
(Str)
> say '42'.WHAT;
(Str)
\end{verbatim}
%
\index{invocant}

They're strings because they are defined within quotes. Although 
Perl will often perform the necessary conversions for you, it 
is generally a good practice not to use quotation marks if your value 
is intended to be a number.

When you type a large integer, you might be tempted to use commas
between groups of digits, as in {\tt 1,234,567}.  This is not a
legal {\em integer} in Perl~6, but it is a legal expression:

\begin{verbatim}
> 1,234,567
(1 234 567)
>
\end{verbatim}
%
That's actually a list of three different integer numbers, and 
not what we expected at all! 

\begin{verbatim}
> say (1,234,567).WHAT
(List)
\end{verbatim}

Perl~6 interprets {\tt 1,234,567} as a comma-separated 
sequence of three integers.  As we will see later, 
the comma is a separator used for constructing lists.
\index{comma}

You can, however, separate groups of digits with the underscore character ``\verb"_"'' for better legibility and obtain a 
proper integer:
\index{underscore character}

\begin{verbatim}
> 1_234_567
1234567
> say 1_234_567.WHAT
(Int)
>
\end{verbatim}
%

\index{sequence}



\section{Formal and Natural Languages}
\index{formal language}
\index{natural language}
\index{language!formal}
\index{language!natural}

{\bf Natural languages} are the languages people speak,
such as English, Spanish, and French.  They were not designed
by people (although people try to impose some order on them);
they evolved naturally.

{\bf Formal languages} are languages that are designed by people for
specific applications.  For example, the notation that mathematicians
use is a formal language that is particularly good at denoting
relationships among numbers and symbols.  Chemists use a formal
language to represent the chemical structure of molecules.  And
most importantly:

\begin{quote}
{\bf Programming languages are formal languages that have been
designed to express computations.}
\end{quote}

Formal languages tend to have strict {\bf syntax} rules that
govern the structure of statements.
For example, in mathematics the statement
$3 + 3 = 6$ has correct syntax, but
not $3 + = 3 \$ 6$.  In chemistry
$H_2O$ is a syntactically correct formula, but $_2Zz$ is not.
\index{syntax}

Syntax rules come in two flavors, pertaining to {\bf tokens} and
{\bf structure}.  Tokens are the basic elements of the language, such as
words, numbers, and chemical elements.  One of the problems with
$3 += 3 \$ 6$ is that \( \$ \) is not a legal token in mathematics
(at least as far as I know).  Similarly, $_2Zz$ is not legal because
there is no chemical element with the abbreviation $Zz$.
\index{token}
\index{structure}

The second type of syntax rule, structure, pertains to the way tokens are
combined.  The equation $3 += 3$ is illegal in mathematics 
because even though $+$ and $=$ are legal tokens, you can't 
have one right after the other. Similarly, in a chemical formula, 
the subscript representing the number of atoms in a 
chemical compound comes after the element name, not before.

This is @ well-structured Engli\$h
sentence with invalid t*kens in it.  This sentence all valid 
tokens has, but invalid structure with.

When you read a sentence in English or a statement in a formal
language, you have to figure out the structure
(although in a natural language you do this subconsciously).  This
process is called {\bf parsing}.
\index{parse}

Although formal and natural languages have many features in
common---tokens, structure, and syntax---there are some
differences:
\index{ambiguity}
\index{redundancy}
\index{literalness}

\begin{description}

\item[Ambiguity] Natural languages are full of ambiguity, which
people deal with by using contextual clues and other information.
Formal languages are designed to be nearly or completely unambiguous,
which means that any statement has exactly one meaning. 

\item[Redundancy] In order to make up for ambiguity and reduce
misunderstandings, natural languages employ lots of
redundancy.  As a result, they are often verbose.  Formal languages
are less redundant and more concise.

\item[Literalness] Natural languages are full of idiom and metaphor.
If we say, ``The penny dropped,'' there is probably no penny and
nothing dropping (this idiom means that someone understood something
after a period of confusion).  Formal languages
mean exactly what they say.

\end{description}

Because we all grow up speaking natural languages, it is sometimes
hard to adjust to formal languages.  The difference between formal and
natural language is like the difference between poetry and prose, but
more so: \index{poetry} \index{prose}

\begin{description}

\item[Poetry] Words are used for their sounds as well as for
their meaning, and the whole poem together creates an effect or
emotional response.  Ambiguity is not only common but often
deliberate.

\item[Prose] The literal meaning of words is more important,
and the structure contributes more meaning.  Prose is more amenable to
analysis than poetry but still often ambiguous.

\item[Programs] The meaning of a computer program is unambiguous
and literal, and can be understood entirely by analysis of the
tokens and structure.

\end{description}

Formal languages are more dense
than natural languages, so it takes longer to read them.  Also, the
structure is important, so it is not always best to read
from top to bottom, left to right.  Instead, learn to parse the
program in your head, identifying the tokens and interpreting the
structure.  Finally, the details matter.  Small errors in
spelling and punctuation, which you can get away
with in natural languages, can make a big difference in a formal
language.


\section{Debugging}
\index{debugging}

Programmers make mistakes.  Programming errors are usually 
called {\bf bugs} and the process of tracking them down is called
{\bf debugging}.
\index{debugging}
\index{bug}

Programming, and especially debugging, sometimes brings out strong
emotions.  If you are struggling with a difficult bug, you might 
feel angry, despondent, or embarrassed.

There is evidence that people naturally respond to computers as if
they were people.  When they work well, we think
of them as teammates, and when they are obstinate or rude, we
respond to them the same way we respond to rude,
obstinate people\footnote{Reeves and Nass, {\it The Media
    Equation: How People Treat Computers, Television, and New Media
    Like Real People and Places}, (Center for the Study of Language and Information, 2003).)}.
\index{debugging!emotional response}
\index{emotional debugging}

Preparing for these reactions might help you deal with them.
One approach is to think of the computer as an employee with
certain strengths, like speed and precision, and
particular weaknesses, like lack of empathy and inability
to grasp the big picture.

Your job is to be a good manager: find ways to take advantage
of the strengths and mitigate the weaknesses.  And find ways
to use your emotions to engage with the problem,
without letting your reactions interfere with your ability
to work effectively.

Learning to debug can be frustrating, but it is a valuable skill
that is useful for many activities beyond programming.  At the
end of each chapter there is a section, like this one,
with our suggestions for debugging.  I hope they help!


\section{Glossary}

\begin{description}

\item[Problem solving]  The process of formulating a problem, finding
a solution, and expressing it.
\index{problem solving}

\item[Abstraction] A way of providing a high-level view 
of a task and hiding the underlying technical details so 
that this task becomes easy.

\item[Interpreter]  A program that reads another program and executes it
\index{interpret}

\item[Compiler]  A program that reads another program and 
transforms it into executable computer code; there used to be 
a strong difference between interpreted and compiled languages, 
but this distinction has become blurred over the last 
two decades or so.
\index{compiler}

\item[Prompt] Characters displayed by the interpreter to indicate
that it is ready to take input from the user.
\index{prompt}

\item[Program] A set of instructions that specifies a computation.
\index{program}

\item[Print statement]  An instruction that causes the Perl~6
interpreter to display a value on the screen.
\index{print statement}
\index{statement!print}

\item[Operator]  A special symbol that represents a simple computation like
addition, multiplication, or string concatenation.
\index{operator}

\item[Value]  One of the basic units of data, like a number or string, 
that a program manipulates.
\index{value}

\item[Type] A category of values.  The types we have seen so far
are integers (type {\tt Int}), rational numbers (type {\tt
Rat}), and strings (type {\tt Str}).
\index{type}

\item[Integer] A type that represents whole numbers.
\index{integer}

\item[Rational] A type that represents numbers with fractional
parts. Internally, Perl stores a rational as two integers 
representing respectively the numerator and the denominator 
of the fractional number.
\index{rational}

\item[String] A type that represents sequences of characters.
\index{string}

\item[Natural language]  Any one of the languages that people speak that
evolved naturally.
\index{natural language}

\item[Formal language]  Any one of the languages that people have designed
for specific purposes, such as representing mathematical ideas or
computer programs; all programming languages are formal languages.
\index{formal language}

\item[Token]  One of the basic elements of the syntactic structure of
a program, analogous to a word in a natural language.
\index{token}

\item[Syntax] The rules that govern the structure of a program.
\index{syntax}

\item[Parse] To examine a program and analyze the syntactic structure.
\index{parse}

\item[Bug] An error in a program.
\index{bug}

\item[Debugging] The process of finding and correcting bugs.
\index{debugging}

\end{description}


\section{Exercises}

\begin{exercise}

It is a good idea to read this book in front of a computer so you 
can try out the examples as you go.

Whenever you are experimenting with a new feature, you should try
to make mistakes.  For example, in the ``Hello, world!'' program,
what happens if you leave out one of the quotation marks?  What
if you leave out both?  What if you spell {\tt say} wrong?
\index{error message}

This kind of experiment helps you remember what you read; it also
helps when you are programming, because you get to know what the error
messages mean.  It is better to make mistakes now and on purpose than
later and accidentally.

Please note that most exercises in this book are provided with 
a solution in the appendix. However, the exercises in this chapter   
and in the next chapter are not intended to let you solve an 
actual problem but are designed to simply let you experiment 
with the Perl interpreter; there is no good solution, just try 
out what is proposed to get a feeling on how it works.

\begin{enumerate}

\item If you are trying to print a string, what happens if you
leave out one of the quotation marks, or both?

\item You can use a minus sign to make a negative number like
{\tt -2}.  What happens if you put a plus sign before a number?
What about {\tt 2++2}?

\item In math notation, leading zeros are OK, as in {\tt 02}.
What happens if you try this in Perl?

\item What happens if you have two values with no operator
between them, such as {\tt say 2 2;}?

\end{enumerate}

\end{exercise}



\begin{exercise}

Start the Perl~6 REPL interpreter and use it as a calculator.

\begin{enumerate}

\item How many seconds are there in 42 minutes, 42 seconds?

\item How many miles are there in 10 kilometers?  Hint: there are 1.61
  kilometers in a mile.

\item If you run a 10 kilometer race in 42 minutes, 42 seconds, what is
  your average pace (time per mile in minutes and seconds)?  What is
  your average speed in miles per hour?

\index{calculator}
\index{running pace}

\end{enumerate}

\end{exercise}


