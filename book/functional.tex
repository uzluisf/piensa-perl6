\chapter{Programación Funcional en Perl}
\label{functional programming}
\index{functional programming}

La programación funcional es un paradigma de programación que 
trata una computación como la evaluación de funciones matemáticas
y evita el cambio de estado y datos mutables. Es un paradigma de
programación declarativo, lo cual significa que la programación
se hace con expresiones o declaraciones en lugar de sentencias.
En el código funcional, el valor de la salida de una función 
depende solamente de los argumentos que son la entrada de la
función, así que llamar una función dos veces con el mismo argumento
producirá el mismo resultado cada vez. La eliminación de los efectos 
secundarios, i.e., cambios en el estado que no dependen en las
entradas de la función, hacen que entender y predecir el comportamiento
de un programa sea más fácil, lo cual es una de las principales 
motivaciones para el desarrollo de la programación funcional.


Perl no es un lenguaje funcional dado que también usa otros
paradigmas de programación que hemos visto en abundancia en
este libro. Sin embargo, Perl ofrece características y capacidades
extensivas de la programación funcional, algunas de las cuales
han sido introducidas en varias secciones de este libro y 
que revisaremos brevemente antes de entrar en más detalles.
\index{programming paradigm}

\section{Funciones de Orden Superior}
\index{higher-order function}

Desde el capítulo~\ref{funcchap} sobre funciones y subrutinas,
en la Sección~\ref{first_class} (p.~\pageref{first_class}),
hemos observado que funciones,subrutinas, y otros objetos de
código son \emph{objetos de primera clase} o 
\emph{ciudadanos de primera clase} en Perl, lo cual significa que
pueden pasarse como valores. Una función en Perl~6 es
un valor que puedes asignar a una variable o pasar como
un argumento a otra función o un valor de retorno desde otra
función.
\index{first-class object}
\index{object, first-class}
\index{first-class citizen}
\index{citizen, first-class}

\subsection{Breve Actualización: Funciones como Objetos de Primera Clase}
\label{fco-refresher}

Nuestro ejemplo inicial muy simple de una función de orden 
superior fue algo así:

\begin{verbatim}
sub do-twice($código) {
	$código();
	$código();
}
sub saludar {
    say "¡Hola Mundo!";
}
do-twice &saludar;
\end{verbatim}

en el cual la subrutina {\tt saludar} se pasa como un 
argumento a la subrutina {\tt do-twice}, con el efecto
de imprimir el saludo dos veces. Una función que se pasa
como un argumento a otra función es llamada usualmente
una \emph{función retrollamada} (del inglés \emph{callback function}).
\index{callback function}

\index{sigil}
El sigilo \verb|&| colocado antes del nombre de la subrutina {\tt saludar}
en la lista de argumentos (como también antes del parámetro 
{\tt code} en la signatura y en el cuerpo de la subrutina {\tt do-twice})
le dice a Perl que estás pasando una subrutina u otro
objeto código que puede ser llamado.
% TODO: get these entries working in plastex
\ifplastex \else
\index{\& sigil@\texttt{\&} sigil}
\fi

En la ciencia de la computación, una subrutina que puede tomar
otra subrutina como un argumento es algunas veces llamada
una \emph{función de orden superior}.
\index{higher-order function}
\index{function!higher-order}

Más ejemplos interesantes de funciones de orden superior
pueden encontrarse con las funciones {\tt reduce}, {\tt map}, y
{\tt grep} que estudiamos en la Sección~\ref{map_filter} 
(p.~\pageref{map_filter}), al igual que la función {\tt sort}
(Section~\ref{sorting} and Section~\ref{advanced_sort}).
\index{map}
\index{reduce}
\index{grep}

Consideremos por ejemplo la tarea de ordenar registros por fechas,
los cuales consisten de un identificador seguido por una fecha
en el formato DD-MM-YYYY, tal como ``id1;13-05-2015'' o ``id2;17-04-2015''.
Los registros necesitan un poco de transformación antes
que podamos compararlos con el propósito de encontrar el orden
cronológico en el cual deberían ser ordenados, así que podríamos
una función separada para la comparación:
\index{sort}

\begin{verbatim}
sub comparar ($reg1, $reg2) {
    my $cmp1 = join ",", reverse split /<[;-]>/, $reg1;
    my $cmp2 = join ",", reverse split /<[;-]>/, $reg2;
    return $cmp1 cmp $cmp2;
}   
\end{verbatim}

Cada registro modificado se construye al encadenar tres funciones.
Estas líneas deberían leerse de derecha a izquierda: primero, 
el valor de entrada se separa en cuatro artículos; estos artículos
son después invertidos y consecuentemente unidos. Así que el resultado
para ``id1;13-05-2015'' es ``2015,05,13,id1'', el cual es adaptado
para una comparación con el operador {\tt cmp}. Más tarde regresaremos
a esta forma de programación de tubería y otras maneras de realizar 
estas operaciones. 
\index{cmp operator}
\index{operator!cmp}
\index{pipeline programming}

Ahora podemos pasar la subrutina {\tt comparar} a la función
{\tt sort}:
\begin{verbatim}
.say for sort &comparar, <id1;13-05-2015 id2;17-04-2015 
                         id3;21-02-2016 id4;12-01-2015>;
\end{verbatim}

Esto muestra:
\begin{verbatim}
id4;12-01-2015
id2;17-04-2015
id1;13-05-2015
id3;21-02-2016
\end{verbatim}

Por favor nota que esto se provee como un ejemplo de funciones 
retrollamadas usadas con la función integrada {\tt sort}.
Veremos al final de la siguiente subsección una manera más simple
de lograr el mismo tipo de ordenamiento usando una función 
anónima.

\subsection{Subrutinas Anónimas y Lambdas}
\index{lambda}
\index{anonymous subroutine}

Hemos visto que una subrutina no necesita tener un nombre
y que puede ser \emph{anónima}. Por ejemplo, puede ser almacenada
en una variable escalar directamente:
\index{anonymous function}
\index{function!anonymous}

\begin{verbatim}
my $saludar = sub {
    say "Hola Mundo!";
};
do-twice $saludar;                 # imprime "Hola Mundo!" dos veces
\end{verbatim}

Ni siquiera necesitamos almacenar el código de la función
anónima en la variable \verb|$saludar|; de hecho, podemos
pasar dicho código directamente como un argumento a la 
subrutina {\tt do-twice}:

\begin{verbatim}

do-twice( sub {say "Hola Mundo!"} );
\end{verbatim}

Dado que nuestra subrutina anónima no toma ningún argumento
y tampoco devuelve un valor útil, podemos simplificar la
sintaxis más aún y pasar un bloque de código anónimo simple
a {\tt do-twice}:

\begin{verbatim}

do-twice {say "Hola Mundo!"};   # imprime "Hola Mundo!" dos veces
\end{verbatim}

Ya has visto varios ejemplos útiles de subrutinas anónimas en este
libro (ver Sección~\ref{map_filter} para más detalles):
\begin{itemize}
\item Con la función {\tt reduce} para computar la suma de 
los primeros 20 números enteros:
\index{reduce function}
\begin{verbatim}

my $suma = reduce { $^a + $^b }, 1..20; # -> 210
\end{verbatim}
\item Con la función {\tt map} para convertir la primera
letra de una lista de ciudades en mayúscula (usando la función
integrada {\tt tc}):
\index{tc function}
\index{map function}
\begin{verbatim}
> .say for map {.tc}, <londres parís roma washington madrid>;
Londres
París
Roma
Washington
Madrid
\end{verbatim}
\item Con la función {\tt grep} para generar una lista de números 
pares al filtrar los números impares:
\index{grep function}
\begin{verbatim}

my @pares = grep { $_ %% 2 }, 1..17; # -> [2 4 6 8 10 12 14 16]
\end{verbatim}
\end{itemize} 

El ejemplo con {\tt reduce} es interesante. En principio, 
contrario a una subrutina, no puedes pasar argumentos a un
bloque de código tan fácilmente (porque no tiene una signatura).
Pero el uso de los parámetros auto-declarados de posición (
o parámetros marcadores) con el twigil \verb|$^| hace posible
usar los parámetros dentro del bloque. 
\index{placeholder}
\index{placeholder!parameter}
\index{twigil}
\index{self-declared parameter}

Debido a esta posibilidad, el bloque de código anónimo se convierte
en lo que usualmente se conoce como una \emph{lambda} en la 
ciencia de la computación (y en matemáticas), i.e., un tipo de
función sin nombre. El cálculo lambda, una teoría matemática 
inventada en la década del 1930 por Alonzo Church, es la esencia
de la mayoría de los lenguajes de programación funcionales de
hoy en día.
\index{lambda}
\index{lambda calculus}
\index{Church, Alonzo}

Actualmente, los dos otros ejemplos más arriba que usan
la variable tópico \verb|$_| son también lambdas. Aunque 
no lo mencionamos en aquellas ocasiones, otras construcciones
que vimos anteriormente son también lambdas. En particular,
considera la sintaxis del ``bloque puntiagudo`` usada dos
en el siguiente bucle {\tt for} que muestra la tabla de 
multiplicación:
\index{pointy block}
\index{for loop}
\index{multiplication tables}

\begin{verbatim}
for 1..9 -> $mult {
    say "Tabla de multiplicación del $mult";
    for 1..9 -> $val {
        say "$mult * $val = ", $mult * $val;
    }
}
\end{verbatim}

Esta es otra forma de lambda donde el parámetro de la ``función``
es definido por la variable de bucle de un bloque puntiagudo.

El ejemplo de ordenamiento presentado en la Subsección~(\ref{fco-refresher})
más arriba puede también escribirse con un bloque de código anónimo
(tomando así ventaja de la sintaxis de {\tt sort}
y usando un bloque de código con solo argumento descrito en la 
Sección~\ref{advanced_sort}):
\index{sort}
\begin{verbatim}
my @in = <id1;13-05-2015 id2;17-04-2015 id3;21-02-2016>;
.say for sort { join ",", reverse split /<[;-]>/, $_ }, @in;
\end{verbatim}

Aquí, el bloque de código algo largo pasado como un argumento
a la función {\tt sort} es una lambda.

\index{lambda}

\subsection{Clausuras}
\index{closure}

En la programación de computadoras, una \emph{clausura} 
(o \emph{clausura lexical}) es una función que puede acceder
el contenido de variables que están lexicalmente disponibles
donde la función es definida, aún si aquellas variables ya no
se encuentran en ámbito en la sección de código donde la
función es llamada.
\index{counter}

Considera el siguiente ejemplo:

\begin{verbatim}
sub crear-contador(Int $cuenta) {
    my $contador = $cuenta;
    sub incrementar-cuenta {
        return $contador++
    }
    return &incrementar-cuenta;
}
my &contar-desde-cinco = crear-contador(5);
say &contar-desde-cinco() for 1..6; # imprime números del 5 al 10
\end{verbatim}

La subrutina {\tt crear-contador} inicializa la variable
\verb|$contador| al valor del parámetro recibido, 
define la subrutina {\tt incrementar-cuenta}, y 
devuelve esta subrutina. El código principal llama a
{\tt crear-contador} para dinámicamente crear la 
referencia de código {\tt \&contar-desde-cinco} (
y podría llamarla muchas veces para crear otros 
contadores que cuentan desde 6, 7, etc.). Después,
\verb|&contar-desde-cinco| es llamada seis veces
e imprime los números entre 5 y 10, cada uno en su 
propia línea.
\index{lexical scope}
\index{scope!lexical}

Lo mágico de esto es que la variable \verb|$contador| está fuera 
de ámbito cuando la función \verb|&contar-desde-cinco| es llamada, 
pero \verb|&contar-desde-cinco| puede aún acceder a su valor,
devolver su valor, e incrementarlo porque \verb|$contador| estaba
dentro del ámbito lexical al tiempo que {\tt incrementar-cuenta}
fue definida. Se dices entonces que {\tt incrementar-cuenta} 
contiene a la variable \verb|$contador|. Por lo tanto, 
la subrutina {\tt incrementar-cuenta} es una clausura.

El ejemplo anterior es un poco artificial y su sintaxis algo rara porque
yo quería mostrar un ejemplo de una clausura nombrada ({\tt incrementar-cuenta}
es una subrutina nombrada). Es usualmente más simple e idiomático
usar clausuras anónimas y escribir nuevamente el ejemplo
de la siguiente manera:

\index{idiomatic}

\begin{verbatim}
sub crear-contador(Int $cuenta) {
    my $contador = $cuenta;
    return sub {
        return $contador++
    }
}
my &contar-desde-cinco = crear-contador(5);
say &contar-desde-cinco() for 1..6; # imprime números del 5 al 10
\end{verbatim}

Podríamos simplificar {\tt crear-contador} aún más con el uso
de sentencias {\tt return} implícitas:

\begin{verbatim}
sub crear-contador(Int $cuenta) {
	my $contador = $cuenta;
	sub { $contador++ }
}
\end{verbatim}

pero esto es menos claro porque la intención del código es meno
explícita.

El último ejemplo {\tt crear-fifo} en la solución al ejercicio sobre
la cola FIFO (Subsección~\ref{functional_queue}) es otro ejemplo
que utiliza el mismo mecanismo:

\begin{verbatim}
sub crear-fifo {
    my @cola;
    return (
        sub {return shift @cola;}, 
        sub ($elem) {push @cola, $elem;}
        ) ;
}
my ($fifo-consigue, $fifo-pone) = crear-fifo();
$fifo-pone($_) for 1..10;
print " ", $fifo-consigue() for 1..5; # ->  1 2 3 4 5
\end{verbatim}
%
\index{closure}
\index{FIFO}
\index{anonymous subroutine}

En Perl~6, todas las subrutinas son clausuras, lo cual significa que 
todas las subrutinas tienen acceso a las variables lexicales que 
existían en el entorno al momento de la definición de las subrutinas.
Sin embargo, ellas no actúan necesariamente como clausuras.

De hecho, todos los objetos de código, incluyendo simple bloques de código anónimos,
pueden actuar como clausuras, lo cual significa que pueden 
hacer referencia a variables lexicales desde el ámbito externo,
y en efecto esto es lo que pasa con la variable de bucle de un
bloque puntiagudo o en el siguiente bloque {\tt map}:

\begin{verbatim}
my $multiplicador = 7;
say  map {$multiplicador * $_}, 3..6; # -> (21 28 35 42)
\end{verbatim}
\index{map}

En este ejemplo, el bloque pasado a \verb|map| hace referencia a
la variable \verb|$multiplicador| desde el ámbito externo, convirtiendo
al bloque en una clausura.


Los lenguajes sin clausuras no pueden proveer fácilmente
funciones de orden superior que sean tan poderosas y fáciles de usar
como {\tt map}.
\index{map}
\index{function!map}

Aquí presentamos otro ejemplo de un bloque que actúa como
una clausura para la implementación de un contador:

\begin{verbatim}
my &cuenta;
{
    my $contador = 10;
    &cuenta = { say $contador++ };
}
&cuenta() for 1..5;  
\end{verbatim}

Esta clausura guarda una referencia a la variable \verb|$contador|
cuando la clausura es creada. La llamada al bloque de código 
\verb|&cuenta| muestra y actualiza a \verb|$contador| exitosamente.
Esto sucede aún cuando la variable ya no se encuentra en el ámbito lexical
al momento que el bloque es ejecutado.
\index{scope!lexical}
\index{lexical scope}
\index{closure}

\section{Procesamiento de Listas y Programación de Tuberías}
\index{pipe-line programming}


A menudo, una computación puede expresarse como una
serie de transformaciones de una lista de valores. Perl
provee funciones capaces de trabajar con los artículos de una
lista y aplicar simples acciones, funciones retrollamadas, 
o bloques de código a estos artículos. Ya hemos visto y usado
abundantemente varias funciones de este tipo:

\begin{itemize}
\item {\tt map} aplica una transformación a cada artículo de una lista.
\index{map} \index{function!map}
\item {\tt grep} es un filtro que mantiene aquellos elementos para los cuales
la función o el bloque de código asociado con {\tt grep} evalúa a verdadero.
\index{grep} \index{function!grep}
\item {\tt reduce} usa cada artículo de una lista para calcular un único 
valor escalar.
\index{reduce} \index{function!reduce}
\item {\tt sort} ordena los elementos de una lista de acuerdo a reglas
definidas en el bloque de código o la subrutina que se pasa.
\index{sort} \index{function!sort}
\end{itemize}

Hemos discutidos varios ejemplos donde estas funciones pueden
usarse conjuntamente en un tipo de tubería de datos en la cual
los datos producidos en cada paso de la tubería son suministrados
al siguiente paso. Por ejemplo, anteriormente en este capítulo 
(Subsección~\ref{fco-refresher}), usamos esto:

\begin{verbatim}
    my $cmp1 = join ",", reverse split /<[;-]>/, $reg1;
\end{verbatim}
\index{reverse} \index{function!reverse}
\index{split} \index{function!split}
\index{join} \index{function!join}

Como destacamos anteriormente, este tipo de código debería leerse
de derecha a izquierda (y de abajo hacia arriba si está escrito 
en varias líneas de código): \verb|$reg1| es suministrada a {\tt split}, 
la cual divide el dato en cuatro piezas; las piezas son después 
invertidas (por {\tt reverse}) y suministrada a {\tt join} para 
crear un solo dato donde las piezas están ahora en orden inverso.

Similarmente, podríamos producir una lista de mascotas que pertenecen
a mujeres solteras que viven en Kansas con el siguiente código que encadena
varios métodos:

\begin{verbatim}
my @mascotas-de-mujeres-solteras-de-kansas =
    map  {  .mascotas },
    grep { !.esta-casada },
    grep {  .genero eq "Femenino" },
    grep {  .estado eq "Kansas" },
         @ciudadanos;
\end{verbatim}
\index{map} \index{function!map}
\index{grep} \index{function!grep}

Esta debería debería leerse de abajo hacia arriba. Toma una
lista de todos los ciudadanos, filtra aquellos de Kansas que 
son mujeres, filtra aquellas que no están casadas, y finalmente
genera la lista de mascotas de dichas personas. Nota que \verb|.mascotas|
puede devolver un animal, una lista de animales, o una lista 
vacía. La función \verb|map| ``aplana`` las listas hasta ese entonces
producidas, así que el resultado final que termina en el array es una
lista plana de animales (y no una lista anidada de listas).

Estas tuberías son muy poderosas y expresivas, y pueden hacer muchas
cosas en pocas líneas de código.

\subsection{Los Operadores Feed y Backward Feed}

En los ejemplos anteriores, los pasos de la tubería se encontraban en 
orden inverso; puedes considerar este hecho algo inconveniente, aunque
es fácil de acostumbrarse.

Perl~6 provee el operador \emph{feed} (\emph{suministro}) \verb|==>|
(algunas veces llamado \emph{pipe} en otros lenguajes)
que hace posible escribir los múltiples pasos de la tubería en un orden
``más natural``, de izquierda a derecha y de arriba hacia abajo.
\index{feed operator}
\index{operator!feed}
% TODO: get these entries working in plastex
\ifplastex \else
\index{==> feed operator@\texttt{==>} feed operator}
\fi

Por ejemplo, al reutilizar el ejemplo sobre el ordenamiento de
los registros por fechas que vimos al inicio de este capítulo,
podrías escribirlo nuevamente así:
\index{sort}

\begin{verbatim}
"id1;13-05-2015" 
    ==> split(/<[;-]>/) 
    ==> reverse() 
    ==> join(",") 
    ==> my @out; # @out es una array que contiene una artículo: una cadena de texto.
say @out.perl;   # ["2015,05,13,id1"]
\end{verbatim}

Por cierto, si estás usando tales operaciones de tubería con una
entrada bien grande, dependiendo de la arquitectura de tu plataforma, 
Perl~6 puede ser capaz de ejecuta estas múltiples operaciones en paralelo
en diferentes CPUs o núcleos (cores), mejorando significativamente
el rendimiento del proceso en su totalidad.

Existe también un operador backward feed (\emph{suministro inverso}), \verb|<==|,
que posibilita la escritura de la tubería en orden inverso:
\index{backward feed operator}
\index{operator!backward feed}
% TODO: get these entries working in plastex
\ifplastex \else
\index{<== backward feed operator@\texttt{<==} backward feed operator}
\fi

\begin{verbatim}
my $out <== join(",") 
        <== reverse() 
        <== split(/<[;-]>/) 
        <== "id1;13-05-2015";
\end{verbatim}


\subsection{El Metaoperador de Reducción}

Ya conocimos este metaoperador en la Sección~\ref{map_filter}. Un metaoperador
actúa sobre otros operadores. Dada una lista y un operador, el {\bf operador de reducción}
[...] aplica el operador iterativamente a todos los valores de la lista
para producir un valor único.
\index{metaoperator}
\index{reduction}
\index{reduction!metaoperator}

Por ejemplo, lo siguiente imprime la suma de todos los elementos
de una lista o un rango:

\begin{verbatim}
say [+] 1..10;      # -> 55
\end{verbatim}

Similarmente, podemos escribir una función factorial así:
\index{factorial!using the reduction meta-operator}

\begin{verbatim}
sub fact (Int $n where $n >= 0) {
    return [*] 1..$n;
}
say fact 20;        # -> 2432902008176640000
say fact 0;         # -> 1
\end{verbatim}

(Nota que esto produce el resultado correcto hasta para el caso 
del factorial de 1, el cual es definido matemáticamente como 1.)

\subsection{El Hiperoperador}

\index{hyperoperator}
Un hiperoperador aplica el operador especificado a cada
artículo de una lista (o dos listas en paralelo) y devuelve
una lista modificada (similar a la función {\tt map}).
Dicho hiperoperador usa las comillas francesas o alemanas, 
\verb|« »| (Unicode codepoints U+00AB y U+00BB), pero puedes
usar las comillas angulares, \verb|<< >>|, si así lo deseas
(o si no sabes como entrar estos caracteres Unicode con tu 
editor).
\index{map}
\index{French quote marks}
\index{German quote marks}

Nuestro primer ejemplo multiplicará cada elemento de una lista
por un número dado (5):

\begin{verbatim}
my @b = 6..10;
my @c = 5 <<*>> @b;
say @c;             # imprime 30 35 ... 50 
					# resultado (5*6, 5*7, ...)
\end{verbatim}

Podemos también combinar dos listas y, por ejemplo, agregar
valores respectivos:

\begin{verbatim}
my @a = 1..5;
my @b = 6..10;
my @d = @a >>+<< @b;
say @d;             # -> [7 9 11 13 15]
\end{verbatim}

Puedes también usar hiperoperadores con un operador unario:

\begin{verbatim}
my @a = 2, 4, 6;
say -<< @a;          # imprime:  -2 -4 -6
\end{verbatim}

Los hiperoperadores con operadores unarios siempre devuelven
una lista con el mismo tamaño que la lista de entrada. 
Los hiperoperadores infijos tienen un comportamiento diferente
dependiendo del tamaño de sus operandos:

\begin{verbatim}
@a >>+<< @b;   # @a y @b deben tener el mismo tamaño
@a <<+<< @b;   # @a puede ser más pequeño
@a >>+>> @b;   # @b pueden ser más pequeño
@a <<+>> @b;   # Cualquiera puede ser más pequeño, 
               # Perl probablemente hará lo que quieres 
               # (principio DWIM)
\end{verbatim}

Los hiperoperadores también funcionan con operadores 
modificadores de asignación:

\begin{verbatim}
@x >>+=<< @y   # Lo mismo que: @x = @x >>+<< @y
\end{verbatim}
\index{hyperoperator}

\subsection{Los Operadores Cruz (X) y Zip (Z)}
\index{cross operator X}
\index{operator!cross}
\index{operator!X (cross)}
\index{zip operator}
\index{operator!zip}
\index{operator!Z (zip)}

El operador cruz usa la letra mayúscula \verb|X|. Dicho operador
toma dos o  más listas como argumentos y devuelve una lista de
todas las listas que puede construirse al combinar los elementos
de cada lista (una forma de ``producto cartesiano''):
\index{cross operator}
\index{operator!cross}
\index{X cross operator}

\begin{verbatim}
my @a = 1, 2;
my @b = 3, 4;
my @c = @a X @b;       # -> [(1,3), (1,4), (2,3), (2,4)]
\end{verbatim}

El operador cruz puede también usarse como un metaoperador y 
aplicar el operador que modifica a cada combinación de artículos
derivada de sus operandos:
\index{metaoperator}

\begin{verbatim}
my @a = 3, 4;
my @b = 6, 8;
say @a X* @b;    # -> 18 24 24 32
\end{verbatim}

Si no se provee un operador adicional (como en el primer ejemplo), 
\verb"X" actúa como si la coma es proveída como el operador adicional
por defecto.

El operador zip \verb|Z| intercala las listas como una cremallera:
\index{zip operator}
\index{operator!zip}
\index{Z zip operator}

\begin{verbatim}
say 1, 2 Z <a b c > Z 9, 8;   # -> ((1 a 9) (2 b 8))
\end{verbatim}

El operador \verb|Z| también existe como un metaoperador, y 
en este caso, en lugar de producir listas anidadas internas
como en el ejemplo más arriba, el operador zip aplicará el 
operador adicional y reemplazará estas listas anidadas con
los valores generados. En el siguiente ejemplo, el operador
de concatenación ~ es usado para fusionar las listas internas 
producidas por el operador zip y crear una cadenas de texto:
\index{metaoperator}

\begin{verbatim}
say 1, 2, 3 Z~ <a b c > Z~ 9, 8, 7; # -> (1a9 2b8 3c7)
\end{verbatim}

\subsection{Un Resumen de Los Operadores de Listas}

Los operadores de listas anteriores son poderosos y pueden 
combinarse para producir construcciones increíblemente expresivas.

Como un ejercicio, intenta resolver las siguientes pequeñas pruebas 
(por favor no prosigas con la lectura hasta que lo haya intentado):

\begin{itemize}
\item Dado que la función integrada {\tt lcm} devuelve el mínimo
común múltiplo entre dos números, escribe un programa que muestre
el número positivo más pequeño divisible por todos los números
entre 1 y 20.
\index{lcm function}

\item Escribe un programa que calcule la suma de todos los dígitos del factorial
de 100.

\item Encuentra la diferencia entre el cuadrado de la suma de los
primeros 100 números enteros y la suma de los cuadrados de los
primeros 100 números enteros.
\end{itemize}

Nuevamente, no prosigas con la lectura hasta que hayas 
tratado de resolver estos pequeños problemas (y con suerte
lo hayas logrado).

El operación de reducción facilita la aplicación de un 
operador a todos los elementos de una lista. Así que si lo
usamos con la función {\tt lcm} nos dará el mínimo común 
múltiplo (MCM) entre 1 y 20:
\index{reduction!metaoperator}

\begin{verbatim}
say [lcm] 1..20;                           # -> 232792560
\end{verbatim}

Para la suma de los dígitos del factorial de 100, usamos 
el metaoperador de reducción \verb|[]| dos veces, una con el
operador de multiplicación para calcular el factorial de 100,
y otra con el operador de adición para añadir los dígitos del 
resultado:
\index{reduction!metaoperator}

% say [+] (1..100).comb;
\begin{verbatim}
say [+] split '', [*] 2..100;              # -> 648
\end{verbatim}

Para el cuadrado de la suma menos la suma de los cuadrados, 
es fácil calcular la suma de los 100 primeros números enteros
con el operador de reducción. El hiperoperador \verb|<<...>>|
fácilmente suministra una lista de los cuadrados de estos 
enteros, y otra aplicación del operador de reducción reduce esta 
lista a una suma:
\index{hyperoperator}

\begin{verbatim}
say ([+] 1..100)**2 - [+] (1..100) «**» 2; # -> 25164150
\end{verbatim}

\subsection{Creando Nuevos Operadores}

Hemos visto brevemente (Sección~\ref{operator_construction})
que puedes construir nuevos operadores o redefinir aquellos existente
con nuevos tipos.
\index{creating new operators}
\index{new operators!creating}

El ejemplo que proveímos fue definir el signo menos
como un operador infijo entre dos hashes para realizar un tipo
de sustracción matemática de conjuntos, i.e., para encontrar las llaves
del primer hash que no se encuentran en el segundo hash.
\index{operator type!infix}

\index{operator type!prefix}
En el párrafo anterior, la palabra \emph{infijo} (\emph{infix} en inglés)
significa que este es un operador binario (dos operandos) que 
se colocará entre los colocará entre los dos operandos.

Existen otros sabores de operadores:
\begin{itemize}
\item Prefijo (\emph{prefix}): un operador unario colocado antes del
operando, por ejemplo el signo de menos en la expresión $-1$
\index{operator type!prefix}

\item Sufijo (\emph{postfix}): un operador unario colocado después del 
operando, por ejemplo el signo de exclamación usado como un símbolo
matemático para el factorial: $5!$
\index{operator type!postfix}

\item Circunfijo (\emph{circumfix}): un operador compuesto de dos símbolos
alrededor del/de los operando(s), por ejemplo los paréntesis $(...)$ o los 
paréntesis angulares $<...>$
\index{operator type!circumfix}

\item Poscircunfijo (\emph{postcircumfix}): un operador compuesto de dos símbolos
colocados después de un operando y alrededor de otro operando, por ejemplo
los corchetes en \verb'@a[1]'
\index{operator type!postcircumfix}
\end{itemize}

Para declarar un nuevo operador, usualmente necesitas especificar los siguientes
elementos en el orden especificado:
\begin{enumerate}
\item El tipo (prefijo, sufijo, etc.) del operador
\item Dos puntos (:)
\item El símbolo o nombre del operador entre paréntesis angulares 
\item La signatura y el cuerpo de la función que define el operador
\end{enumerate}

Por ejemplo, podríamos definir un operador prefijo \% de la siguiente 
manera:

\begin{verbatim}
multi sub prefix:<%> (Int $x) {   # operador doble
    2 *  $x;
}
say % 21;   # -> 42
\end{verbatim}

Esto es solamente un ejemplo para mostrar cómo la construcción de un 
operador funciona; \% no es probablemente un buen nombre para un operador
doble. Lo interesante aquí es que hemos reutilizado un operador existente
(el operador de módulo), pero el compilador no se confunde porque el 
módulo es un operador infijo y nuestro nuevo operador es definido como
un operador prefijo.
\index{modulo operator}
\index{new operators!creating}
\index{creating new operators}

Un ejemplo con un mejor nombre sería usar un signo de exclamación(!) como
un operador sufijo para el factorial de un número, al igual que en 
notación matemática:
\index{factorial!operator}

\begin{verbatim}
multi sub postfix:<!> (Int $n where $n >= 0) {
    [*] 2..$n;
}
say 5!;                     # -> 120
\end{verbatim}

Observa que el signo de exclamación usado como un operador prefijo
(i.e., colocado al frente de su operando) es el operador de negación,
pero usualmente no es posible confundir los dos operadores
porque uno es un operador prefijo y nuestro operador es un operador
sufijo (aunque debes ser cuidadoso con la posición donde pones los 
espacios en blanco si tu expresión es algo complicada). La palabra 
{\tt multi} no es estrictamente requerida en este ejemplo, pero es 
probablemente buena práctica colocarla, solo para cubrir los casos
cuando es necesaria.

Al igual que el otro ejemplo, podríamos definir el operador $\Sigma$ (suma) 
de la siguiente manera:

\begin{verbatim}
multi sub prefix:<Σ> (@*lista-num) {
    [+] @lista-num;
}
say Σ (10, 20, 12);         # -> 42
\end{verbatim}

El beneficio de usar el operador $\Sigma$ sobre \verb|+| puede no 
ser directamente obvio, pero es algunas veces útil crear un ``lenguaje
de dominio específico``(DSL), i.e., un sublenguaje específicamente adaptado
para un contexto o área de estudio específica (por ejemplo, matemáticas o química),
lo cual permite que un problema o solución en particular sea expresada más
claramente que lo que el lenguaje principal existente permitiría. En Perl~6, 
las gramáticas y la facilidad de crear nuevos operadores hace la 
creación de un DSL una tarea sencilla.
\index{domain-specific language (DSL)}
\index{DSL (domain-specific language)}
\index{new operators!creating}
\index{creating new operators}


El nuevo operador no tiene que ser declarado entre los paréntesis angulares.
Las siguientes declaraciones podrían usarse para definir un operador de
adición:

\begin{verbatim}
infix:<+>
infix:<<+>>
infix:«+»
infix:('+')
infix:("+")
\end{verbatim}

También puedes especificar la precedencia de tus nuevos operadores
(relativa a aquellos existentes). Por ejemplo:
\index{precedence}
\index{operator!precedence}

\begin{verbatim}
multi sub infix:<mult> is equiv(&infix:<*>) { ... }
multi sub infix:<plus> is equiv(&infix:<+>) { ... }
mutli sub infix:<zz> is tighter(&infix:<+>) { ... }
mutli sub infix:<yy> is  looser(&infix:<+>) { ... }
\end{verbatim}

En uno de sus artículos (``Structured Programming with 
go to statements'', Diciembre 1974), Donald Knuth, un científico
de la computación muy famoso, usa el símbolo \verb|:=:| como un 
operador de pseudocódigo para expresar el intercambio de dos valores
en una variable, i.e, la siguiente operación:
\index{Knuth, Donald}
\index{swap}
\index{variable interchange}

\begin{verbatim}
# Advertencia: esto es pseudocódigo, no código funcional, por el momento
my $a = 1; my $b = 2;
$a :=: $b; 
say "$a $b";  # -> 2 1 
\end{verbatim}
\index{pseudo-code}

En el artículo de Knuth, esto es solo un atajo falso para 
discutir más fácilmente el algoritmo de ordenamiento rápido
(\emph{quicksort} en inglés) (descrito en el ejercicio~\index{quicksort}),
pero podemos implementa fácilmente ese símbolo:
\index{swap operator}
\index{sort!quick sort}
\index{Hoare, Charles Antony Richard}

\begin{verbatim}
multi sub infix:<:=:> ($a is rw, $b is rw) {
    ($a, $b) = $b, $a;
}
\end{verbatim}

Nota que esto puede también escribirse de esta manera:

\begin{verbatim}
multi sub infix:<:=:> ($a is rw, $b is rw) {
    ($a, $b) .= reverse;   # equiavalente a: ($a, $b) = ($a, $b).reverse 
}
\end{verbatim}

Podemos ahora probarlo con los siguientes ejemplos:

\begin{verbatim}
my ($c, $d) = 2, 5;
say $c :=: $d;             # -> (5 2)
# usándolo para intercambiar dos elementos del array
my @e = 1, 2, 4, 3, 5;
@e[2] :=: @e[3];
say @e;                    # -> [1 2 3 4 5]
\end{verbatim}

Ahora, el pseudocódigo de más arriba funciona tan bien
como código real. Un algoritmo de ordenamiento como el presentado
más abajo(Sección~\ref{combsort}) puede típicamente tener líneas
como estas para intercambiar dos elementos en un array:
\index{swap}

\begin{verbatim}
if $alguna-condicion {
    my ($x, $y) = @array[$i], @array[$i + gap];
    @array[$i], @array[$i + gap] = $y, $x;
}
\end{verbatim}

Si el operador \verb|:=:| es definido, podríamos escribir
estas líneas de la siguiente manera:

\begin{verbatim}
@array[$i] :=: @array[$i + gap] if $alguna-condicion;
\end{verbatim}

Un punto final muy interesante. Supón que queremos usar
el operador $\oplus$ para la unión matemática de conjuntos
entre dos hashes. Esto podría escribirse fácilmente 
como sigue:
\index{$\oplus$ operator}
\index{operator!$\oplus$}
\index{hash merge operator}

\begin{verbatim}
multi sub infix:<⊕> (%a, %b) {
    my %c = %a;
    %c{$_} = %b{$_} for keys %b;
    return %c
}
\end{verbatim}

Esto funciona bien:

\begin{verbatim}
my %q1 = jan => 1, feb => 2, mar => 3;
my %q2 = apr => 4, may => 5, jun => 6;
my %first_half = %q1 ⊕ %q2;
say %first_half;
# {apr => 4, feb => 2, jan => 1, jun => 6, mar => 3, may => 5}
\end{verbatim}

Hasta ahora, todo bien, nada realmente nuevo. Pero el 
nuevo operador infijo $\oplus$ se ha vuelto casi similar a
un operador integrado de Perl, y por lo tanto podemos
usarlo conjuntamente con el metaoperador de reducción:
\index{reduction!metaoperator}

\begin{verbatim}
my %q1 = jan => 1, feb => 2, mar => 3;
my %q2 = apr => 4, may => 5, jun => 6;
my %q3 = jul => 7, aug => 8, sep => 9;
my %q4 = oct => 10, nov => 11, dec => 12;
my %year = [⊕] %q1, %q2, %q3, %q4;
say %year;
# {apr => 4, aug => 8, dec => 12, feb => 2, jan => 1, 
# jul => 7, jun => 6, mar => 3, may => 5, nov => 11, 
# oct => 10, sep => 9}
\end{verbatim}

Todo funciona como si este nuevo operador fuera parte
de la gramática de Perl~6. Y eso es lo que, en efecto, ha pasado
aquí: hemos extendido el lenguaje con un nuevo operador.
Esta posibilidad de extender el lenguaje es una parte clave de
la habilidad de Perl~6 para afrontar futuras necesidades que
ni siquiera pensamos en el momento presente.
\index{extending the language}
\index{new operators!creating}
\index{creating new operators}

\section{Creando Tus Funciones Similares a Map}
\index{map}

Hemos visto en este capítulo y en la Sección~\ref{map_filter} (p.~\pageref{map_filter}) 
cómo la funciones de orden superior tales como las funciones
{\tt reduce}, {\tt map}, {\tt grep}, y {\tt sort} pueden ser 
poderosos y expresivas. También existen otros tipos de funciones
integradas en Perl, pero nos gustaría también crear nuestras 
propias funciones.


\subsection{Versiones Personalizadas de map, grep, etc.}

Veamos cómo podríamos escribir nuestra propias versiones
personalizadas de tales funciones.

\subsubsection{mapa, Una Versión Pura de Perl de map}
\index{my-map}

Primeramente, intentemos escribir en puro Perl la función {\tt map}.
Necesitamos tomar una subrutina o un bloque de código
como su primer argumento, aplicarlo a un array o una lista,
y devolver la lista modificada.

\begin{verbatim}
sub mapa (&codigo, @valores) { 
    my @temp;
    push @temp, &codigo($_) for @valores;
    return @temp;
}
my @result = mapa { $_ * 2 }, 1..5; 
say @result;                   # -> [2 4 6 8 10]
\end{verbatim}

En el primer intento, esto funciona exactamente como se espera.
(He intentado el mismo experimento con otros lenguajes de 
programación en el pasado, incluyendo Perl 5; tomó
varios intentos antes de conseguir el programa correcto, 
especialmente lo concerniente a la sintaxis de la llamada.
Aquí, todo cae en su lugar naturalmente.) Para ser honesto, 
las condiciones en este ejemplo son muy limitadas y podrían
haber casos donde {\tt mapa} no funciona de la misma forma
exactamente que {\tt map}; la lección a tomar es que es bien
fácil construir una subrutina de orden superior que se comporta 
esencialmente en la misma manera que {\tt map}.
\index{higher-order function}
\index{map}

\subsubsection{my-grep}
\index{my-grep}
\index{grep}

Writing our pure-Perl version of {\tt grep} is just about as 
easy:
\begin{verbatim}
sub my-grep (&code, @values) { 
    my @temp;
    for @values -> $val {
        push @temp, $val if &code($val);
    }
    return @temp;
}
my @even = my-grep { $_ %% 2 }, 1..10; 
say @even;                   # -> [2 4 6 8 10]
\end{verbatim}

\subsection{Our Own Version of a Sort Function}
\label{combsort}
\index{sort!comb sort}
\index{comb sort}

We can similarly write our own version of the sort 
function. 

\index{sort!merge sort}
\index{sort!quick sort}
\index{merge sort}
\index{quick sort}

The Perl {\tt sort} function implements a sort 
algorithm known as \emph{merge sort}\footnote{Merge 
sort is presented in some details in 
section~\ref{mergesort}.}.  Some 
previous versions of the Perl language (prior to 
version~5.8) implemented another algorithm known 
as \emph{quick sort}\footnote{Quick sort is presented 
in \ref{quicksort}}. The main reason for this 
change was that, although quick sort is on average  
a bit faster than merge sort, there are specific 
cases where quick sort is much less efficient than 
merge sort (notably when the data is almost sorted). 
These cases are very rare with random data, but not 
in real situations: it is quite common that you have 
to sort a previously sorted list in which only a 
few elements have been added or modified.
\index{quick sort}

In computing theory, it is frequently said that, for 
sorting \emph{n} items, both merge sort and quick 
sort have an \emph{average complexity} of $O(n \log n)$, 
which essentially means that the number of operations 
to be performed is proportional to $n \log n$ if 
the number of items to be sorted is $n$, 
with quick sort being usually slightly faster; but 
quick sort has a \emph{worst-case complexity} of 
$O(n^{2})$, whereas merge sort has a \emph{worst-case 
complexity} of $O(n \log n)$. When the number $n$ of 
items to be sorted grows large, $n^{2}$ becomes 
very significantly larger than $n \log n$. In other 
words, merge sort is deemed to be better because it 
remains efficient in all cases.
\index{algorithmic complexity}
\index{sort!merge sort}
\index{sort!quick sort}

\index{sort!comb sort}
Suppose now that we want to implement another sorting 
algorithm whose performance is alleged to be better. 
For this example, we will use a somewhat exotic sorting 
algorithm known as \emph{comb sort} (a.k.a. Dobosiewicz's 
sort), which is described on this page of Wikipedia:
\url{https://en.wikipedia.org/wiki/Comb_sort}. This 
algorithm is said to be \emph{in place}, which means that 
it does not need to copy the items into auxiliary data 
structures, and has generally good performance (often better 
than merge sort), but is not very commonly used in 
practice because its theoretical analysis is very difficult 
(in particular, it seems that it has a good worst-case 
performance, but no one has been able to prove this 
formally so far). In fact, we don't really care about 
the real performance of this sort algorithm; it is 
very unlikely that a pure Perl implementation 
of the comb sort will outperform the built in 
{\tt sort} function implemented in C and probably very 
carefully optimized by its authors. We only want to 
show how a sort subroutine could be implemented.

To work the same way as the internal {\tt sort}, a sort 
function must receive as parameters a comparison function 
or code block and the array to be sorted, and the 
comparison routine should use placeholder parameters (\verb'$^a' 
and  \verb'$^b' in the code below). This is a possible 
basic implementation:
\index{placeholder!parameter}

\begin{verbatim}
sub comb_sort (&code, @array) {
    my $max = @array.elems;
    my $gap = $max;
    loop {
        my $swapped = False;
        $gap = Int($gap / 1.3);    # 1.3: optimal shrink factor
        $gap = 1 if $gap < 1;
        my $lmax = $max - $gap - 1;
        for (0..$lmax) -> $i {
            my ($x, $y) = (@array[$i], @array[$i+$gap]);
            (@array[$i], @array[$i+$gap], $swapped) = ($y, $x, True)
                if &code($x, $y) ~~ More;  # or: if &code($x, $y) > 0
        }
        last if $gap == 1 and ! $swapped;
    }
}
\end{verbatim}
\index{sort!comb sort}

This can be tested with the following code:

\begin{verbatim}
my @v;
my $max = 500;
@v[$_] = Int(20000.rand) for (0..$max);

comb_sort {$^a <=> $^b}, @v;
.say for @v[0..10], @v[493..499]; # prints array start and end
# prints (for example):
# (14 22 77 114 119 206 264 293 298 375 391)
# (19672 19733 19736 19873 19916 19947 19967)
\end{verbatim}

The inner loop compares items that are distant from each 
other by \verb'$gap' values, and swaps them if they are 
not in the proper order. At the beginning, \verb'$gap' 
is large, and it is divided by a shrink factor at each 
iteration of the outer loop. Performance heavily depends
on the value of the shrink factor. At the end, the gap 
is 1 and the comb sort becomes equivalent to a bubble 
sort. The optimal shrink factor lies somewhere between 1.25 
and 1.33; I have used a shrink factor of 1.3, the value 
suggested by the authors of the original publications 
presenting the algorithm.
\index{sort!bubble sort}

\subsection{An Iterator Version of map}
\index{iter!map}
\index{map}

These {\tt my-map}, {\tt my-grep}, and {\tt comb\_sort} 
functions are pedagogically interesting, but they aren't 
very useful if they do the same thing as their built-in 
counterparts (and are probably slower). However, now 
that we have seen how to build them, we can create our 
own versions that do things differently.

Say we want to create a function that acts like 
{\tt map} in the sense that it applies a 
transformation on the items of the input list, but does 
that on the items one by one, on demand from a consumer 
process, and pauses when and as long as the consumer process 
does not need anything. This could be described as an 
{\bf iterator} returning modified elements on demand from the 
source list. You might think that this does not have much to 
do with {\tt map}, but it might also be considered as 
a form of {\tt map} with delayed evaluation, which 
processes only the elements of the input lists that are 
necessary for the program, not more than that. 
\index{iterator}
\index{delayed evaluation}

\index{laziness}
\index{lazy!list processing}
The idea 
of processing only what is strictly required is often called 
\emph{laziness}, and this is a very useful idea. Lazy 
list processing can be very useful not only because it 
avoids processing data that is not needed, and therefore 
can contribute to better resource usage and better 
performance, but also because it makes it possible to consider 
\emph{infinite} lists: so long as you can guarantee that 
you are only going to use a limited number of elements, 
you don't have any problem considering lists that are 
potentially unlimited. Perl~6 provides the concepts and 
tools to do this.
\index{infinite list}

To reflect these considerations, we will call our subroutine 
{\tt iter-map}. Since we might want to also write a 
{\tt iter-grep} subroutine and possibly others, we will 
write separately an iterator and a data transformer.
\index{iterator}

We can use a closure to manufacture an iterator:
\index{closure}

\begin{verbatim}
sub create-iter(@array) {
    my $index = 0;
    return sub { @array[$index++];}
}
my $iterator = create-iter(1..200);
say $iterator() for 1..5; # -> 1, 2, 3, 4, 5
\end{verbatim} 

Now that the iterator returns one value at a time, we 
can write the {\tt iter-map} subroutine:
\index{iter!map}

\begin{verbatim}
sub iter-map (&code-ref, $iter) {
    return &code-ref($iter);
}
my $iterator = create-iter(1..200);
say iter-map { $_ * 2 }, $iterator() for 1..5; # -> 2, 4, 6, 8, 10
\end{verbatim}

Since we have called the {\tt iter-map} function only 5~times, 
it has done the work of multiplying values by 2 only 5~times, 
instead of doing it 200 times, 195 of which are for nothing. 
Of course, multiplying a number by 2 isn't an expensive 
operation and the array isn't very large, but this shows 
how laziness can prevent useless computations. We will come 
back to this idea, since Perl~6 offers native support to lazy 
lists and lazy processing.
\index{laziness}

As already noted, an additional advantage of using a function 
such as {\tt iter-map} is that it is possible to use 
virtually infinite lists. This implementation using an 
infinite list works just as before:

\begin{verbatim}
my $iterator = create-iter(1..*);
say iter-map { $_ * 2 }, $iterator() for 1..5;
     # prints 2, 4, 6, 8, 10
\end{verbatim}

\subsection{An Iterator Version of grep}
\index{iter!grep}

If we try to write a {\tt iter-grep} subroutine on the same 
model:

\begin{verbatim}
my $iterator = create-iter(reverse 1..10);
sub iter-grep (&code_ref, $iter) {
    my $val = $iter();
    return $val if &code_ref($val);
}
# simulating ten calls
say iter-grep { $_ % 2 }, $iterator for 1..10;
\end{verbatim}

it doesn't quite work as desired, because this will print 
alternatively odd values (9, 7, 5, etc.) and undefined 
values (for the even values of the array). Although we 
haven't specified it yet, we would prefer {\tt iter-grep} 
to supply the next value for which the \verb'$code-ref' 
returns true. This implies that {\tt iter-grep} has to 
loop over the values returned by the iterator until it 
receives a proper value.

That might look like this:
\index{iter!grep}

\begin{verbatim}
my $iterator = create-iter(reverse 1..10);
sub iter-grep (&code_ref, $iter) {
    loop {
        my $val = $iter();
        return unless defined $val;  # avoid infinite loop
        return $val if &code_ref($val);
	}
}
# simulating ten calls
for 1..10 {
    my $val = iter-grep { $_ % 2 }, $iterator;
    say "Input array exhausted!" and last unless defined $val;
    say $val;
}
\end{verbatim}

This now works as expected:

\begin{verbatim}
9
7
5
3
1
Input array exhausted!
\end{verbatim}

However, we still have a problem if the array 
contains some undefined values (or ``empty slots''). This 
would be interpreted as the end of the input array, whereas 
there might be some additional useful values in the array. 
This is sometimes known in computer science as the 
``semi-predicate'' problem. Here, {\tt iter-grep} has no 
way to tell the difference between an empty slot in the array 
and the end of the array. A more robust implementation 
therefore needs a better version of {\tt create-iter}  
returning something different for an undefined array item 
and array exhaustion. For example, the iterator might return 
a false value when done with the array, and a pair with the 
array item as a value otherwise. A pair will be considered 
to be true, even if its value isn't defined:
\index{semi-predicate problem}
\index{pair}

\begin{verbatim}
sub create-iter(@array) {
    my $index = 0;
    my $max-index = @array.end;
    return sub { 
        return False if $index >= $max-index; 
        return ("a_pair" => @array[$index++]);
    }
}
my @array = 1..5;
@array[7] = 15;
@array[9] = 17;
push @array, $_ for 20..22;
.say for '@array is now: ', @array;
my $iterator = create-iter(@array);
sub iter-grep (&code_ref, $iter) {
    loop {
        my $returned-pair = $iter();
        return unless $returned-pair;  # avoid infinite loop
        my $val = $returned-pair.value;
        return $val if defined $val and &code_ref($val);
	}
}
for 1..10 {
    my $val = iter-grep { $_ % 2 }, $iterator;
    say "Input array exhausted!" and last unless defined $val;
    say $val;
}
\end{verbatim}
\index{iter!grep}

Running this script displays the following:
\begin{verbatim}
@array is now:
[1 2 3 4 5 (Any) (Any) 15 (Any) 17 20 21 22]
1
3
5
15
17
21
Input array exhausted!
\end{verbatim}

This now works fully as desired.

Although {\tt iter-map} did not suffer from the same problem, 
you might want as an exercise to modify {\tt iter-map}
to use our new version of {\tt create-iter}.
\index{iter-map}

The advantage of the iterator functions seen above is that they 
process only the items that are requested by the user code, so 
that they perform only the computations strictly required and 
don't waste CPU cycles and time doing unnecessary work. We have 
gone through these iterating versions of the {\tt map} 
and {\tt grep} functions as a form of case study for 
pedagogical purposes, in order to explain in practical terms 
the idea of laziness. 
\index{map}
\index{laziness}
\index{iterator}

This is what would have been necessary to implement lazy 
iterators in earlier versions of Perl (e.g., Perl~5), but 
much of this is not required with Perl~6 which has built-in 
support for lazy lists and lazy operators, as we will see soon.

\section{The gather and take Construct}
\index{gather function}
\index{gather and take construct}
\index{take function}

A useful construct for creating (possibly lazy) lists 
is  \verb'gather { take }'. A \verb'gather' block 
acts more or less like a loop and runs until 
\verb'take' supplies a value. This construct is also 
a form of iterator.

For example, the following code returns a list of 
numbers equal to three times each of the even numbers 
between 1 and 10:

\begin{verbatim}
my @list = gather { 
    for 1..10 {
        take 3 * $_ if $_ %% 2
    } 
};
say @list;                 # -> [6 12 18 24 30]
\end{verbatim}

Here, \verb'gather' loops on the values of the range 
and {\tt take} ``returns'' the wanted values.

If you think about it, the code above seems to 
be doing a form of combination of a {\tt map} and a 
{\tt grep}.
\index{map}
\index{grep}

We can indeed simulate a \verb'map'. For example:

\begin{verbatim}
my @evens = map { $_ * 2 }, 1..5;
\end{verbatim}

could be rewritten with a \verb'gather { take }' 
block :

\begin{verbatim}
my @evens = gather {take $_ * 2 for 1.. 5}; # [2 4 6 8 10]
\end{verbatim}

And we could simulate a {\tt grep} similarly:

\begin{verbatim}
my @evens = gather {take $_ if $_ %% 2 for 1..10};
\end{verbatim}

Since {\tt take} also admits a method syntax, this could 
be rewritten as:

\begin{verbatim}
my @evens = gather {.take if $_ %% 2 for 1..10};
\end{verbatim}

\index{map}
These code examples don't bring any obvious advantage 
over their \verb'map' or {\tt grep} counterparts and 
are not very useful in themselves, but they illustrate 
how a \verb'gather { take }' block can be thought 
of as a generalization of the \verb'map' and 
{\tt grep} functions. And, as already mentioned, 
the first example in this section actually does combine 
the actions of a {\tt map} and a {\tt grep}.

In fact, we can write a new version of {\tt my-map}:
\index{my-map}

\begin{verbatim}
sub my-map (&coderef, @values) {
   return gather {
      take &coderef($_) for @values;
   };
}
say join " ", my-map {$_ * 2}, 1..10;
# prints: 2 4 6 8 10 12 14 16 18 20
\end{verbatim}

Writing a new version of {\tt my-grep} is just 
about as easy and left as an exercise to the reader.
\index{my-grep}

\index{take function}
Calling the {\tt take} function only makes sense 
within the context of a \verb'gather' block, but 
it does not have to be within the block itself 
(or within the lexical scope of the \verb'gather' 
block); it can be within the \emph{dynamic scope} of the 
\verb'gather' block

\index{dynamic variable}
\index{variable!dynamic}
\index{dynamic scope}
\index{lexical scope}
Although we haven't covered this concept before, 
Perl has the notion of \emph{dynamic scope}: contrary 
to lexical scope, dynamic scope encloses not only 
the current block, but also the subroutines called 
from within the current block. Dynamic scope variables 
use the ``*'' twigil. Here is an example:
%
\begin{verbatim}
sub write-result () { say $*value; }
sub caller (Int $val) { 
    my $*value = $val * 2; 
    write-result();
}
caller 5;               # -> 10
\end{verbatim}
%
In the code above, the \verb'$*value' dynamic variable 
is declared and defined in the \verb'caller' subroutine 
and used in the \verb'write-result' subroutine. This would not 
work with a lexical variable, but it works with a dynamic 
variable such as \verb'$*value', because the scope of 
\verb'$*value' extends to the \verb'write-result' subroutine 
called by \verb'caller'. 

\index{take function}
Similarly, the {\tt take} function 
can work within the dynamic scope of the \verb'gather' 
block, which essentially means that the {\tt take} 
function can be called within a subroutine called from 
the \verb'gather' block. For example:
\index{lexical scope}
\index{dynamic scope}
\index{scope!dynamic}

\begin{verbatim}
my @list = gather {
    compute-val($_) for 1..10; 
}
sub compute-val(Numeric $x) {
    take $x * $x + 2 * $x - 6;
}
say @list[0..5];        # -> (-3 2 9 18 29 42)
\end{verbatim}

As you can see, the {\tt take} function is not called 
within the {\tt gather} block, but it works fine because 
it is within the dynamic scope of the gather block, i.e., 
within the {\tt compute-val} subroutine, which is itself 
called in the {\tt gather} block.

One last example will show how powerful the 
\verb'gather { take }' construct can be.

Let's consider this problem posted on the Rosetta Code 
site (\url{http://rosettacode.org/wiki/Same_Fringe}): 
write a routine that will compare the leaves (``fringe'') 
of two binary trees to determine whether they are the 
same list of leaves when visited left-to-right. The 
structure or balance of the trees does not matter; 
only the number, order, and value of the leaves is 
important. 
\index{rosettacode}
\index{binary tree}
\index{tree!binary}

The solution in Perl~6 uses a \verb'gather { take }' 
block and consists of just six~code lines:

\begin{verbatim}
sub fringe ($tree) {
    multi sub fringey (Pair $node) {fringey $_ for $node.kv;}
    multi sub fringey ( Any $leaf) {take $leaf;}
    gather fringey $tree;
}
sub samefringe ($a, $b) { fringe($a) eqv fringe($b) }
\end{verbatim}

Perl~6 is the clear winner in terms of the shortest code to 
solve the problem.

As a comparison, the Ada example is almost 300 lines long, 
the C and Java programs slightly over 100 lines. By the way, 
the shortest solutions besides Perl~6 (Clojure, Picolisp, 
Racket) run in about 20~lines and are all functional 
programming languages, or (for Perl~5 for example) are 
written using functional programming concepts. 
Although the number of code lines is only one of many 
criteria to compare programs and languages, this is 
in my humble opinion a testimony in favor of the functional 
programming paradigm and its inherent expressiveness.
\index{functional programming}


\section{Lazy Lists and the Sequence Operator}
\index{lazy!list}

Let's come back now to the idea of lazy lists and study 
how Perl~6 can handle and use them.

\subsection{The Sequence Operator}
\index{sequence operator}
\index{operator!sequence}

Perl provides the \verb'...' sequence operator to build 
lazy lists. For example, this:

\begin{verbatim}
my $lazylist := (0, 1 ... 200);
say $lazylist[42];                  # -> 42
\end{verbatim}

produces a lazy list of successive integers between 0 and 200. 
The Perl~6 compiler may or may not allocate some of the numbers
(depending on the implementation), but it is not required to 
produce the full list immediately. The numbers that have not 
been generated yet may be created and supplied later, if and when 
the program tries to use these values. 

As explained below, if you want to generate consecutive 
integers, you can actually simplify the lazy list definition:

\begin{verbatim}
my $lazylist := (0 ... 200);
\end{verbatim}


\index{laziness}
If you assign a sequence to an array, it will generate 
all the values of the sequence immediately, since 
assignment to an array is eager (nonlazy).  However, 
you can force laziness with the  {\tt lazy} built-in 
when assigning to an array:

\begin{verbatim}
my @lazyarray = lazy 1 ... 200;     # -> [...]
say @lazyarray.elems;               # -> Cannot .elems a lazy list
say @lazyarray[199];                # -> 200
say @lazyarray[200];                # -> (Any)
say @lazyarray.elems;               # -> 200
\end{verbatim}

Here, \verb'@lazyarray' is originally lazy. 
Evaluating one item past the last element of the array 
forces Perl to actually generate the full array (and the 
array is no longer lazy). After that, no further elements 
can be generated, and {\tt .elems} stays at 200 (unless 
you actually assign values to elements past the 200th 
element).

When given two integers, one for the first and the last items of 
a list, the sequence operator will generate a list of consecutive 
integers between the two supplied integers. If you supply two 
initial items defining implicitly a step, this will generate 
an arithmetic sequence:
\index{arithmetic sequence}

\begin{verbatim}
my $odds = (1, 3 ... 15);           # (1 3 5 7 9 11 13 15)
my $evens = (0, 2 ... 42);          # (0 2 4 6 8 ... 40 42)
\end{verbatim}

You may remember that, in Section~\ref{sequence} of the chapter 
on arrays and lists, we said that parentheses are usually not 
necessary for constructing a list, unless needed for 
precedence reasons. The above code is one such example: try 
to run that code without parentheses and observe the content 
of the \verb'$odds' and \verb'$evens' variables.

When three initial numbers in geometric progression are supplied, the 
sequence operator will produce a geometric sequence, as in 
this example producing the powers of two:
\index{geometric sequence}

\begin{verbatim}
say (1, 2, 4 ... 32);              # -> (1 2 4 8 16 32)
\end{verbatim}

The sequence operator may also be used to produce noninteger 
numbers, as shown in this example under the REPL:

\begin{verbatim}
> say (1, 1.1 ... 2);
(1 1.1 1.2 1.3 1.4 1.5 1.6 1.7 1.8 1.9 2)
\end{verbatim}

Contrary to the \verb'..' range operator, the sequence 
operator can also count down:
\index{range operator}

\begin{verbatim}
say (10 ... 1);                    #  (10 9 8 7 6 5 4 3 2 1)
\end{verbatim}

\subsection{Infinite Lists}
\index{infinite list}

One of the great things about lazy lists is that, since 
item evaluation is postponed, they can be infinite without 
consuming infinite resources from the computer:
\index{infinite list}

\begin{verbatim}
my $evens = (0, 2 ... Inf);        # (...)
say $evens[18..21];                # -> (36 38 40 42)
\end{verbatim}

The {\tt Inf} operand is just the so-called ``Texas'' 
or ASCII equivalent of the $\infty$ infinity symbol. 
The above could have been written:
\index{infinity symbol}

\begin{verbatim}
my $evens = (0, 2 ... ∞); 
say $evens[21];                    # -> 42
\end{verbatim} 

The most common way to indicate an infinite lazy list, though, 
is to use the \verb'*' whatever argument:
\index{whatever!operator}
\index{operator!whatever}

\begin{verbatim}
my $evens = (0, 2 ... *); 
say $evens[21];                    # -> 42
\end{verbatim} 

\subsection{Using an Explicit Generator}
\index{sequence operator!generator}

The sequence operator \verb'...' is a very powerful tool 
for generating lazy lists. Given one number, it just 
starts counting up from that number (unless the 
end of the sequence is a lower number, 
in which case it counts down). Given two numbers 
to start a sequence, it will treat it as an arithmetic 
sequence, adding the difference between those first 
two numbers to the last number generated to generate 
the next one. Given three numbers, it checks to see 
if they represent the start of an arithmetic or a 
geometric sequence, and will continue it.
\index{arithmetic sequence}
\index{geometric sequence}

However, many interesting sequences are neither arithmetic 
nor geometric.  They can still be generated with the 
sequence operator provided one term can be deduced from 
the previous one (or ones). For this, you need to explicitly 
provide the code block to generate the next number in 
the sequence. For example, the list of odd integers 
could also be generated with a generator as follows:

\begin{verbatim}
say (1, { $_ + 2 } ... 11);        # -> (1 3 5 7 9 11)
\end{verbatim}

We now have yet another way of defining the factorial 
function:
\index{factorial!with a lazy infinite list}

\begin{verbatim}
my $a;
my @fact = $a = 1, {$_ * $a++} ... *;
say @fact[0..8];          # -> (1 1 2 6 24 120 720 5040 40320)
\end{verbatim}

or, possibly more readable:

\begin{verbatim}
my @fact = 1, { state $a = 1; $_ * $a++} ... *;
say @fact[0..8];          # -> (1 1 2 6 24 120 720 5040 40320)
\end{verbatim}


This approach is much more efficient than those we have 
seen before for repeated use, since it automatically 
caches the previously computed values in the lazy array. 
As you might remember from Section~\ref{memoize} 
(p.~\pageref{memoize}), \emph{caching} is the idea of 
storing a value in memory in order to avoid having to 
recompute it, with the aim of saving time and CPU cycles.
\index{cache}

And we can similarly construct a lazy infinite list of 
Fibonacci numbers:
\index{Fibonacci!numbers}

\begin{verbatim}
my @fibo = 0, 1, -> $a, $b { $a + $b } ... *;
say @fibo[0..10];   # -> (0 1 1 2 3 5 8 13 21 34 55)
\end{verbatim}

This can be rewritten in a more concise (albeit possibly 
less explicit and less clear) way using the \verb'*' 
whatever placeholder parameter:
\index{whatever!operator}
\index{whatever!placeholder parameter}

\begin{verbatim}
my @fibo = 0, 1, * + * ... *;
say @fibo[^10];      # -> (0 1 1 2 3 5 8 13 21 34)
\end{verbatim}

Just as for factorial, this is more efficient than the 
implementations we've seen previously, because the 
computed values are cached in the lazy array.
\index{cache}

Similarly the sequence of odd integers seen at the 
beginning of this section could be generated in a 
slightly more concise form with the whatever "\verb'*'" 
parameter:
\index{sequence operator}

\begin{verbatim}
say (1, * + 2  ... 11);        # -> (1 3 5 7 9 11)
\end{verbatim}

This syntax with an asterisk is called a 
\emph{whatever closure}; we will come back to 
it below.
\index{whatever!closure}

There is, however, a small caveat in using the sequence operator with 
an explicit generator: the end value (the upper bound) has 
to be one of the generated numbers for the list to stop at 
it. Otherwise, it will build an infinite list:

\begin{verbatim}
my $nums = (0, { $_ + 4 } ... 10);
say $nums[0..5];     # -> (0 4 8 12 16 20)
\end{verbatim}

As you can see in this example, the generator ``jumps over the end 
point'' (it goes beyond 10), and the list is 
in fact infinite. This is usually not a problem 
in terms of the computer resources, since it is 
a lazy infinite list, but it is probably a bug if 
you expected the list not to run above 10. In this 
specific case, it is very easy to compute an end 
point that will be matched (e.g., 8 or 12), but it may be 
more complicated to find a valid end point. For example, 
it is not obvious to figure out what the largest 
Fibonacci number less than 10,000 might be without 
computing first the series of such numbers until the 
first one beyond 10,000.

In such cases where it is difficult to predict what the end 
point should be, we can define another code block to test 
whether the sequence should stop or continue. The sequence 
will stop if the block returns a true value. For example, 
to compute Fibonacci numbers until 100, we could use this 
under the REPL:

\begin{verbatim}
> my @fibo = 0, 1, -> $a, $b { $a + $b } ... -> $c { $c > 100}
[0 1 1 2 3 5 8 13 21 34 55 89 144]
\end{verbatim}

This is better, as it does stop the series of numbers, but 
not quite where we wanted: we really wanted it to stop at the last 
Fibonacci under 100, and we're getting one more. It would be 
quite easy to remove or filter out the last generated Fibonacci 
number, but it's even better not to generate it at all. A slight 
change in the syntax will do that for us:

\begin{verbatim}
> my @fibo = 0, 1, -> $a, $b { $a + $b } ...^ -> $c { $c > 100}
[0 1 1 2 3 5 8 13 21 34 55 89]
\end{verbatim}

Switching from \verb'...' to \verb'...^' means the 
resulting list does not include the first element 
for which the termination test returned true.

Similarly, we can limit the \emph{whatever closure} 
syntax seen above as follows:
\index{whatever!closure}

\begin{verbatim}
> say 0, 1, * + * ...^ * > 100;
(0 1 1 2 3 5 8 13 21 34 55 89)
\end{verbatim}

\section{Currying and the Whatever Operator}
\index{curry}

{\bf Currying} (or partial application) is a basic technique 
of functional programming, especially in pure functional 
programming languages such as Haskell. The ``curry'' name comes 
from the American mathematician Haskell Curry, one of the 
founders (with Alonzo Church) of logical mathematical 
theories, including lambda-calculus and others. (And, as 
you might have guessed, the Haskell programming language 
derived its name from Curry's first name.)
\index{Curry, Haskell}
\index{Church, Alonzo}

To curry a function having several arguments means replacing 
it with a function having only one argument and returning 
another function (often a closure) whose role is to 
process the other arguments.

In some pure functional programming languages, a function 
can only take one argument and return one result. Currying 
is a technique aimed at coping with this apparent limitation. 
Perl does not have such a limitation, but currying can still 
be very useful to reduce and simplify the arguments lists 
in subroutine calls, notably in cases of repeated recursive 
calls.


\subsection{Creating a Curried Subroutine}
\index{curry}

The standard example is an \emph{add} function. Suppose 
we have an add mathematical function, \verb'add(x, y)', 
taking two arguments and returning their sum. 

In Perl, defining the {\tt add} subroutine is very simple:
\index{add}

\begin{verbatim}
sub add (Numeric $x, Numeric $y) {return $x + $y}
\end{verbatim}

A curried version of it would be another function 
\verb'add_y(x)' returning a function adding $y$ to 
its argument.

This could be done with a closure looking like this:
\index{curry}

\begin{verbatim}
sub make-add (Numeric $added-val) {
    return sub ($param) {$param + $added-val;}    
    # or: return sub {$^a + $added-val;}
}
my &add_2 = make-add 2;
say add_2(3);           # -> 5
say add_2(4.5);         # -> 6.5
\end{verbatim}

The \verb'&add_2' code reference is a curried version 
of our mathematical {\tt add} function. It takes only 
one argument and returns a value equal to the argument 
plus two.

We can of course create other curried subroutines using 
{\tt make-add} with other arguments:

\begin{verbatim}
my &add_3 = make-add 3;
say &add_3(6);           # -> 9
\end{verbatim}

There is not much new here: the \verb'&add_2' and 
\verb'&add_3' are just closures that memorize the 
increment value passed to the {\tt make-add} 
subroutine. This can be useful when some functions 
are called many times (or recursively) with many  
arguments, some of which are always the same: 
currying them makes it possible to simplify the 
subroutine calls.

\subsection{Currying an Existing Subroutine with the {\tt assuming} Method}
\index{curry}
\index{method!assuming}

If a subroutine already exists, there is often no need 
to create a new closure with the help of a ``function 
factory'' (such as {\tt make-add}) as we've done just above. 
It is possible to curry the existing function, using 
the {\tt assuming} method on it:
\index{assuming method}
\index{function factory}

\begin{verbatim}
sub add (Numeric $x, Numeric $y) {return $x + $y}   
my &add_2 = &add.assuming(2);                       
add_2(5);              # -> 7                                     
\end{verbatim}

The {\tt assuming} method returns a callable object 
that implements the same behavior as the original 
subroutine, but has the values passed to {\tt assuming}
already bound to the corresponding parameters.

It is also possible to curry built-in functions. For example, 
the {\tt substr} built-in takes normally three arguments:
the string on which to operate, the start position, and the 
length of the substring to be extracted. You might need 
to make a number of extractions on the same very long 
string. You can create a curried version of {\tt substr} 
always working on the same string:
\index{substr function}

\begin{verbatim}
my $str = "Cogito, ergo sum";                     
my &string-start-chars = &substr.assuming($str, 0);
say &string-start-chars($_) for 6, 13, 16; 
\end{verbatim}

This will print:

\begin{verbatim}
Cogito
Cogito, ergo
Cogito, ergo sum
\end{verbatim}

Note that we have ``assumed'' two parameters here, so 
that the curried subroutine ``remembers'' the first 
two arguments and only the third argument needs be 
passed to \verb'&string-start-chars'.

You can even curry Perl~6 operators (or your own) if 
you wish:
\index{assuming method}

\begin{verbatim}
my &add_2 = &infix:<+>.assuming(2);
\end{verbatim}

\subsection{Currying with the Whatever Star Parameter}
\index{curry}
\index{whatever!term}
\index{whatever!star parameter}
\label{whatever star parameter}

A more flexible way to curry a subroutine or an expression 
is to use the \emph{whatever star} (*) argument:

\begin{verbatim}
my &third = * / 3; 
say third(126);          # -> 42
\end{verbatim}

The \emph{whatever star} (*) is a placeholder for 
an argument, so that the expression returns 
a closure.

It can be used in a way somewhat similar to the \verb'$_' 
topical variable (except that it does not have to exist 
when the declaration is made):

\begin{verbatim}
> say map 'foo' x * , (1, 3, 2);
(foo foofoofoo foofoo)
\end{verbatim}
\index{map}

It is also possible to use multiple whatever terms 
in the same expression. For example, the {\tt add} 
subroutine could be rewritten as a whatever 
expression with two parameters:

\begin{verbatim}
my $add = * + *;
say $add(4, 5);          # -> 9
\end{verbatim}

or:

\begin{verbatim}
my &add = * + *;
say add(4, 5);           # -> 9
\end{verbatim}

You might even do the same with the multiplication operator:

\begin{verbatim}
my $mult = * * *;
say $mult(6, 7);         # -> 42
\end{verbatim}

The compiler won't get confused and will figure out 
correctly that the first and third asterisks are 
whatever terms and that the second asterisk is 
the multiplication operator; in other words that this is 
more or less equivalent to:
\index{whatever!term}

\begin{verbatim}
my $mult = { $^a * $^b };
say $mult(6, 7);         # -> 42
\end{verbatim}

or to:

\begin{verbatim}
my $mult = -> $a, $b { $a * $b }
say $mult(6, 7);         # -> 42  
\end{verbatim}

To tell the truth, the compiler doesn't get confused, 
but the user might, unless she or he has been previously 
exposed to some functional programming languages that 
commonly use this type of syntactic construct. 

These ideas are powerful, but you are advised to pay 
attention so you don't to fall into the trap of code 
obfuscation.

That being said, the functional programming paradigm 
is extremely expressive and can make your code much 
shorter. And, overall, shorter code, provided it remains 
clear and easy to understand, is very likely to have 
fewer bugs than longer code.

\section{Using a Functional Programming Style}
\index{functional programming!style}
\label{funcstyle}

In this chapter, we have seen how to use techniques derived 
from functional programming to make our code simpler and 
more expressive. In a certain way, though, we haven't fully 
applied functional programming. All of the techniques we have 
seen stem from functional programming and are a crucial 
part of it, but the true essence of functional programming 
isn't really about using higher-order functions, list 
processing and pipeline programming, anonymous subroutines 
and closures, lazy lists and currying, and so on. 
The true essence of functional programming is a specific 
mindset that treats computation as the evaluation of mathematical 
functions and avoids changing-state and mutable data.
\index{functional programming!style}
\index{pipe-line programming}

Instead of simply using techniques derived from functional 
programming, we can go one step further and actually 
write code in functional programming style. If we are going 
to avoid changing-state and mutable data, this means that 
we will no longer use variables (or at least not change them, 
and treat them as immutable data) and do things differently.

\subsection{The Merge Sort Algorithm}
\label{mergesort}
\index{merge sort}
\index{sort!merge sort}

Consider the example of a classical and efficient sorting 
technique called the merge sort, invented by John von Neumann 
in 1945. It is based on the fact that if you have two sorted 
arrays, it is significantly faster to merge the two arrays 
into a single sorted array, by reading each array in parallel and 
picking up the appropriate item from either of the arrays, than 
it would be to blindly sort the data of the two arrays.
\index{von Neumann, John}
\index{merging arrays or lists}

Merge sort is a ``divide and conquer'' algorithm which 
consists of recursively splitting the input unsorted array into 
smaller and smaller sublists, until each sublist contains only 
one item (at which point the sublist is sorted, by definition), 
and then merging the sublists back into a sorted array.
\index{divide and conquer algorithm}

To avoid adding unnecessary complexity, we will discuss here 
implementations that simply sort numbers in ascending numeric 
order.


\subsection{A Non-Functional Implementation of Merge Sort}
\index{merge sort! non functional implementation}

Here's how we could try to implement a merge sort algorithm using 
purely imperative/procedural programming:

\begin{verbatim}
# ATTENTION: buggy code
sub merge-sort (@out, @to-be-sorted, $start = 0, $end = @to-be-sorted.end) {
    return if $end - $start  < 2;
    my $middle = ($end + $start) div 2;
    my @first = merge-sort(@to-be-sorted, @out, $start, $middle);
    my $second = merge-sort(@to-be-sorted, @out, $middle, $end);
    merge-lists(@out, @to-be-sorted, $start, $middle, $end);
}
sub merge-lists (@in, @out, $start, $middle, $end) {
    my $i = $start;
    my $j = $middle;
    for $start..$end  -> $k {
        if $i < $middle and ($j >= $end or @in[$i] <= @in[$j]) {
            @out[$k] = @in[$i];
            $i++;
        } else {
            @out[$k] = @in[$j];
            $j++;
        } 
    }
}
my @array = reverse 1..10;
my @output = @array;
merge-sort2  @output, @array;
say @output;
\end{verbatim}

This program always works on the full array (and its copy) and 
the sublists are not extracted; the extraction is simulated by 
the use of subscript ranges.

This code is not very long, but nonetheless fairly complicated.
If you try to run it, you'll find that there is a bug: 
the last item of the original array is improperly sorted. For 
example, if you try to run it on the list of 10 consecutive 
integers in reverse order (i.e., ordered from 10 to 1) used in 
the test at the end of the above code, you'll get the following 
output array:

\begin{verbatim}
[2 3 4 5 6 7 8 9 10 1]
\end{verbatim}


As an exercise, try fixing the bug before reading any further. 
(The fix is explained next.)

It is likely that you'll find that identifying and correcting 
the bug is quite difficult, although this bug is actually 
relatively simple (when I initially wrote this code, I 
encountered some more complicated bugs before arriving at this one). 
It is quite hard to properly use the array subscripts and 
insert the data items in the right place, avoiding off-by-one 
and other errors.
\index{off-by-one error}

Here's a corrected version:

\begin{verbatim}
sub merge-sort (@out, @to-be-sorted, $start = 0, $end = @to-be-sorted.elems) {
    return if $end - $start < 2;
    my $middle = ($end + $start) div 2;
    merge-sort(@to-be-sorted, @out, $start, $middle);
    merge-sort(@to-be-sorted, @out, $middle, $end);
    merge-lists(@out, @to-be-sorted, $start, $middle, $end);
}
sub merge-lists (@in, @out, $start, $middle, $end) {
    my $i = $start;
    my $j = $middle;
    for $start..$end - 1  -> $k {
        if $i < $middle and ($j >= $end or @in[$i] <= @in[$j]) {
            @out[$k] = @in[$i];
            $i++;
        } else {
            @out[$k] = @in[$j];
            $j++;
        } 
    }
}
my @array = pick 20, 1..100;
my @output = @array;
merge-sort2  @output, @array;
say @output;
\end{verbatim}

The main change is in the signature of the \verb'merge-sort' 
subroutine: the default value for the \verb'$end' parameter 
is the size (number of items) of the array, and no 
longer the subscript of the last elements of the array (so, 
the bug was an off-by-one error). Making this correction 
also makes it necessary to change the pointy block 
(\verb'for $start..$end - 1 -> ...') in the 
\verb'merge-lists' subroutine.
\index{off-by-one error}

For 20~randoms integers between 1 and 100, this prints 
out something like the following:

\begin{verbatim}
[11 13 14 15 19 24 25 29 39 46 52 57 62 68 81 83 89 92 94 99]
\end{verbatim}

The point is that it is difficult to understand the detailed 
implementation of the algorithm, and fairly hard to debug, even 
using the Perl debugger presented in section~\ref{perl-debugger}.
\index{debugger}
\index{merge sort! non functional implementation}

\subsection{A Functional Implementation of Merge Sort}
\index{merge sort!functional implementation}
Rather than modifying the entire array at each step through 
the process (and being confused in the management of subscripts), 
we can split recursively the data into actual sublists and work 
on these sublists.

This can lead to the following implementation:

\begin{verbatim}
sub merge-sort (@to-be-sorted) {
    return @to-be-sorted if @to-be-sorted < 2;
    my $middle = @to-be-sorted.elems div 2;
    my @first = merge-sort(@to-be-sorted[0 .. $middle - 1]);
    my @second = merge-sort(@to-be-sorted[$middle .. @to-be-sorted.end]);
    return merge-lists(@first, @second);
}
sub merge-lists (@one, @two) {
    my @result;
    loop {
        return @result.append(@two) unless @one;
        return @result.append(@one) unless @two;
        push @result, @one[0] < @two[0] ?? shift @one !! shift @two;
    }
} 
\end{verbatim}

The code is shorter than the previous implementation, but 
that's not the main point.

The {\tt merge-sort} subroutine is somewhat similar to 
the previous implementation, except that it recursively 
creates the sublists and then merge the sublists.

It is the {\tt merge-lists} subroutine (which does the bulk 
of the work in both implementations) that is now much 
simpler: it receives two sublists and merges them. Most of this 
work is done in the last code line; the two lines before it 
are only taking care of returning the merged list when one 
of the input sublists ends up being empty.
\index{merge sort!functional implementation}

This functional version of the program captures  
the essence of the merge sort algorithm:
\begin{itemize}
\item If the  array has less than two items, it is already 
sorted, so return it immediately (this is the base case 
stopping the recursion).
\index{base case}
\index{recursion!base case}
\item Else, pick the middle position of the array to divide 
it into two sublists, and call \verb'merge-sort' recursively 
on them;
\item Merge the sorted sublists thus generated.
\item Return the merged list to the caller.
\end{itemize}

\index{functional programming!style}
I hope that you can see how much clearer and simpler the 
functional style implementation is. To give you an idea, 
writing and debugging this latter program took me about 
15~minutes, i.e., about 10~times less than the 
nonfunctional version. If you don't believe me, try to 
implement these two versions for yourself. (It's a good 
exercise even if you \emph{do} believe me.)

The exercise section of this chapter (section~\ref{quicksort}) 
will provide another (and probably even more telling) example 
of the simplicity of functional programming compared to more 
imperative or procedural approaches.


\section{Debugging}
\label{test_module}
\index{testing!automated tests}

This time, we will not really talk about debugging proper, 
but about a quite closely related activity, testing.

Testing code is an integral part of software development. In 
Perl~6, the standard {\tt Test} module (shipped and installed 
together with Rakudo) provides a testing framework which enables 
automated, repeatable verification of code behavior.
\index{test module}
\index{testing!module}

The testing functions emit output conforming to the \emph{Test 
Anything Protocol} or TAP (\url{http://testanything.org/}), a 
standardized testing format which has implementations in Perl, 
C, C++, C\#, Ada, Lisp, Erlang, Python, Ruby, Lua, PHP, Java, 
Go, JavaScript, and other languages.

A typical test file looks something like this:

\begin{verbatim}
use v6;
use Test;      # a Standard module included with Rakudo
use lib 'lib';

# ...

plan $num-tests;

# .... tests

done-testing;  # optional with 'plan'
\end{verbatim}

We ensure that we're using Perl~6, via the use of the \verb'v6' 
pragma, then we load the \verb'Test' module and specify where 
our libraries are. We then specify how many tests we plan 
to run (such that the testing framework can tell us 
if more or fewer tests were run than we expected) 
and when finished with the tests, we use {\tt done-testing} 
to tell the framework we are done.

We have already seen a short example of the use of the 
\verb'Test' module in Section~\ref{sol_fact_operator} 
(solution to the exercise of 
Section~\ref{operator_construction}).

The \verb'Test' module exports various functions that 
check the return value of a given expression, and produce 
standardized test output accordingly.

In practice, the expression will often be a call to a function 
or method that you want to unit-test. You may want to check:

\begin{itemize}
\item Truthfulness: 
\begin{verbatim}
ok($value, $description?); 
nok($condition, $description?);
\end{verbatim}
\index{ok function (testing)}
\index{nok function (testing)}
\index{test module!ok function}
\index{test module!nok function}

The {\tt ok} function marks a test as passed if the given 
\verb'$value' evaluates to true in a Boolean context. 
Conversely, the {\tt nok} function marks a test as passed 
if the given value evaluates to false. Both functions 
accept an optional \verb'$description' of the test. 
For example:

\begin{verbatim}
ok  $response.success, 'HTTP response was successful';
nok $query.error,      'Query completed without error';
\end{verbatim}

\item String comparison:
\begin{verbatim}
is($value, $expected, $description?)
\end{verbatim}
\index{is function (testing)}
\index{test module!is function}

The {\tt is} function marks a test as passed if \verb'$value' 
and \verb'$expected' compare positively with the \verb'eq' 
operator. The function accepts an optional description 
of the test.
\index{string equality}

\item Approximate numeric comparison:

\begin{verbatim}
is-approx($value, $expected, $description?)
\end{verbatim}
\index{is-approx function (testing)}
\index{test module!is-approx function}

{\tt is-approx} marks a test as passed if the \verb'$value' and 
\verb'$expected' numerical values are approximately equal 
to each other. The subroutine can be called in numerous ways 
that let you test using relative or absolute tolerance 
of different values. (If no tolerance is set, it will default 
to an absolute tolerance of $10^{-5}$.)
\index{approximate numeric equality}

\item Regex:

\begin{verbatim}
like($value, $expected-regex, $description?)
unlike($value, $expected-regex, $description?)
\end{verbatim}
\index{like function (testing)}
\index{unlike function (testing)}
\index{test module!like function}
\index{test module!unlike function}

The {\tt like} function marks a test as passed if the 
\verb'$value' matches the \verb'$expected-regex'. Since 
we are speaking about regexes, ``matches'', in the 
previous sentence, really means ``smart matches''. The 
{\tt unlike} function marks a test as passed if the 
\verb'$value' does not match the \verb'$expected-regex'.

For example:
\begin{verbatim}
like 'foo', /fo/, 'foo looks like fo';
unlike 'foo', /bar/, 'foo does not look like bar';
\end{verbatim}

\item And many other functions which you can study in the 
following documentation: \url{https://docs.perl6.org/language/testing.html}.

\end{itemize}

In principle you could use {\tt ok} for every kind of 
comparison test, by including the comparison in the 
expression passed as a value:
\index{factorial}

\begin{verbatim}
ok factorial(4) == 24, 'Factorial - small integer';
\end{verbatim}

However, it is better (where possible) to use one of the 
specialized comparison test functions, because they can 
print more helpful diagnostics output in case the comparison 
fails.

If a test fails although it appears to be successful, and 
you don't understand why it fails, you may want to use the 
\verb'diag' function to get additional feed back. For example,
assume that the test:

\begin{verbatim}
ok $foo, 'simple test';
\end{verbatim} 

is failing and that you don't have enough feedback to 
understand why; you may try:
\index{diag function (test diagnostic}

\begin{verbatim}
diag "extensive feedback" unless
    ok $foo, 'simple test';
\end{verbatim}

This might give you the additional information you need.

Suppose we want to test a subroutine to determine whether a 
given string is a palindrome (as discussed in several chapters 
in this book, see for example Exercise~\ref{palindrome} and 
Subsection~\ref{palindrome_2}). You could perform that test 
by writing something like this:
\index{palindrome}

\begin{verbatim}
# file is-palindrome.p6
use v6;

sub is-palindrome($s) { $s eq $s.flip }

multi sub MAIN( $input ) {
    if is-palindrome( $input ) {
        say "'$input' is palindrome.";
    }
    else {
        say "'$input' is not palindrome.";
    }
}

multi sub MAIN(:$test!) {
    use Test;
    plan 4;
    ok is-palindrome(''), 'empty string';
    ok is-palindrome('aba'), 'odd-sized example';
    ok is-palindrome('abba'), 'even-sized example';
    nok is-palindrome('blabba'), 'counter example';
}
\end{verbatim}

Usually, tests are stored in different files placed in a ``t'' 
subdirectory. Here, for this short test, everything is in the 
same file, but two multi {\tt MAIN} subroutines are supplied 
to either test whether a passed parameter is a palindrome, or to 
run the test plan. See Section~\ref{MAIN} (p.~\pageref{MAIN} 
and Subsection~\ref{MAIN_sub} (p.~\pageref{MAIN_sub})
if you need a refresher on the {\tt MAIN} subroutine.
\index{MAIN}
\index{multi subroutines}

You can run these tests as follows:

\begin{verbatim}
$ perl6 is-palindrome.p6 abba
'abba' is palindrome.
$ perl6 is-palindrome.p6 abbaa
'abbaa' is not palindrome.
$
$ perl6 is-palindrome.p6 --test
1..4
ok 1 - empty string
ok 2 - odd-sized example
ok 3 - even-sized example
ok 4 - counter example
\end{verbatim}

Try this example, play with it, change some lines, add 
new tests, and see what happens.

Writing such unit tests may appear to be tedious work. 
The truth, though, is that it is manual testing that is 
somewhat tedious and, it you try, you'll find that 
writing and using such test scenarios makes the testing 
work much less cumbersome. You usually write the tests 
once, and run them very often. And you will be surprised 
how many bugs you will find even if you are sure your 
code is correct! Also, once you've written 
a test suite for something, you might still be using it 
years later, for example for nonregression testing after 
a software change. This can be not only a time saver, but 
also a guarantee that you're supplying good quality software.
\index{non-regression test}

Many organizations actually write their tests even before 
writing the programs. This process is called 
\emph{test-driven development} and there are many areas where 
it is quite successful. In fact, the Rakudo/Perl~6 compiler 
had a very large test suite (more than 40,000 tests) long 
before the compiler was ready. In a way, the test suite 
even became the true specification of the project, so that 
you could use the same test suite for verifying another 
implementation of Perl~6.
\index{test-driven development}

An additional advantage of 
such an approach is that measuring the ratio of tests that 
pass may often be a better metric of software completion than  
the usual ``wet finger in the wind'' estimates, such as, say, 
a ratio of the number of code lines written versus the estimate 
of the final number of code lines.
\index{software metric}

\section{Glosario}

\begin{description}

\item[Objeto de primera clase] Un objeto que puede
pasarse como un argumento a una subrutina o devolverse 
como un valor de retorno de la misma. En Perl, las subrutinas
son objetos de primera clase (también llamados ciudadanos de
primera clase).. 
\index{first-class object}
\index{first-class citizen}

\item[Función retrollamada] Una función o subrutina que se pasa
como un argumento a otra función.
\index{callback function}

\item[Función de orden superior] Una función o subrutina que toma
otra subrutina (o un simple bloque de código) como un 
argumento. Las funciones integradas 
{\tt map}, {\tt grep}, {\tt reduce}, y {\tt sort}
son ejemplos de funciones de orden superior.
\index{higher-order function}
\index{function!higher-order}
\index{map}
\index{grep}
\index{reduce}

\item[Subrutina anónima] Una subrutina que carece de nombre. 
También llamada comúnmente una \emph{lambda}. Aunque no
son exactamente lo mismo, bloques puntiagudos pueden también
ser asimilados en subrutinas anónimas.
\index{anonymous subroutine}
\index{lambda}

\item[Clausura] Una función que puede acceder variables que 
están lexicalmente disponible donde la función es definida, 
incluso aquellas variables que ya no se encuentran en el
ámbito donde la función es llamada.
\index{closure}

\item[Programación de tubería] Un modelo de programación en el
cual piezas de datos (usualmente listas) experimentan transformaciones
sucesivas como en una tubería o línea de ensamblaje.
\index{pipe-line programming}

\item[Reducción] Un proceso a través del cual una lista de valores
es reducida a un valor único. Por ejemplo, una lista de números
puede reducirse, por ejemplo, para encontrar el promedio, un valor
máximo, o una mediana. Algunos lenguajes conocen este proceso como
\emph{folding}.
\index{reduction}

\item[Metaoperador] Un operador que actúa sobre otro operador para
proveer nuevas funcionalidades.
\index{metaoperator}

\item[Complexidad algorítmica] Una medida aproximada del número de operaciones 
de computación (y tiempo) necesarias para realizar una computación sobre
un conjunto de datos relativamente grande, y, más precisamente, una 
medida de cómo un algoritmo escala cuando el conjunto de datos 
crece.
\index{algorithmic complexity}

\item[Evaluación peresoza] Un proceso de evaluación aplazada donde,
por ejemplo, uno puebla una lista o procesa los artículos de una 
lista solamente a petición, cuando es requerido, para evitar 
procesamiento innecesario.
\index{laziness}

\item[Iterador] Una pieza de código que devuelve valores a petición
y mantiene un récord de donde ha llegado, para de este modo ser
capaz de saber cuál valor debería ser proveído en la siguiente 
iteración.
\index{iterator}

\item[Cache] El proceso de caché se refiere al almacenamiento de un valor 
en la memoria (en una variable, un array, un hash, etc.) para de esta 
forma evitar la necesidad de calcularlo nuevamente, y como 
resultado ahorrar algún tiempo de computación.
\index{cache}

\item[Currying] La técnica de currying se refiere al hecho de 
tomar una función que acepta varios argumentos y transformarla en 
una serie de funciones que toman un menor número de argumentos.

{\bfseries Nota del traductor:} Introducida por 
\href{https://es.wikipedia.org/wiki/Gottlob_Frege}{Gottlob Frege} y desarrollada por 
\href{https://es.wikipedia.org/wiki/Moses_Sch%C3%B6nfinkels}{Moses Schönfinkel}, esta 
técnica de programación funcional fue nombrada en honor al matemático y lógico 
\href{https://es.wikipedia.org/wiki/Haskell_Curry}{Haskell Curry}, 
quien también ayudó en su desarrollo. Como resultado de esto, 
me he refrenado de traducir o proveer una alternativa en español de dicho término.
\index{curry}

\item[Desarrollo guiado por pruebas] Una metodología de desarrollo donde las
pruebas de software se escriben de las especificaciones antes que 
el programa actual. Esto se hace para que sea más fácil chequear que 
el programa cumple con las especificaciones.
\index{test-driven development}

\end{description}

\section{Ejercicio: Ordenamiento Rápido}
\label{quicksort}
\index{quick sort}
\index{sort!quick sort}

\begin{exercise}
Ordenamiento rápido es un algoritmo de ordenamiento ``divide y conquista``
inventado por Tony Hoare en 1959. Se basa en la división del array
a ser ordenado. Para dividir un array, un elemento conocido el pivote
es seleccionado. Todos los elementos más pequeños que el pivote son
en frente del mismo y todos aquellos que son más grandes son movidos
después del mismo. Las sublistas compuestas de los elementos más pequeños
y más grandes son entonces recursivamente ordenadas a través del mismo 
procedimiento y finalmente reagrupadas.
\index{Hoare, Charles Antony Richard}
\index{divide and conquer algorithm}

Una de las dificultades es seleccionar el pivote correcto. Idealmente
debería ser el valor mediano de los artículos del array, dado que
esto daría particiones de igual tamaño, por consiguiente haciendo el 
algoritmo óptimamente eficiente, pero encontrar el valor mediano 
para cada partición tomaría tiempo y últimamente penalizaría el
dicho rendimiento. Varias variantes de ordenamiento rápido se han 
intentado, con diferentes estrategias para seleccionar (usualmente de
forma arbitraria) un pivote. Aquí, seleccionamos un elemento en o
cerca del medio de la partición.
\index{pivot!quick sort algorithm}

La siguiente es una implementación típica del algoritmo de 
ordenamiento rápido que no usa la programación funcional.

\begin{verbatim}[fontshape=up]
sub ordrapido(@entrada) {
    sub intercambiar ($x, $y) {
        (@entrada[$x], @entrada[$y]) = @entrada[$y], @entrada[$x];
    }
    sub rap_ord ($izq, $der) {
        my $pivote = @entrada[($izq + $der) div 2];
        my $i = $izq;
        my $j = $der;
        while $i < $j {
            $i++ while @entrada[$i] < $pivote;
            $j-- while @entrada[$j] > $pivote;
            if $i <= $j {
                intercambiar $i, $j;
                $i++;
                $j--;
            }
        }
        rap_ord($izq, $j) if $izq < $j;
        rap_ord($i, $der) if $j < $der;
    }
    rap_ord(0, @entrada.end)
}
my @array = pick 20, 1..100;
ordrapido @array;
say @array;
\end{verbatim}

El array es modificado en lugar (lo cual tiene la ventaja de 
requerir memoria limitada), lo cual significa que el array 
original es modificado.

Para la programación funcional, los datos internos son inmutables, 
por lo tanto copias fragmentos de datos en nuevas listas, en lugar 
de modificarlos ``en el lugar``.

Con el mismo espíritu que lo hicimos en la sección\ref{funcstyle}
para el algoritmo de ordenamiento de mezcla, trata de escribir una
implementación del algoritmo de ordenamiento rápido. Pista: esto 
se puede hacer la mitad de una docena de líneas de código. 

Solución: \ref{sol_quicksort}.
\index{quick sort}

\end{exercise}

