
\chapter{Caso Práctico: Juego de Palabras}
\label{wordplay}

En lugar de introducir conceptos nuevos, el propósito de 
este capítulo es dejarte practicar y consolidar el conocimiento
que has adquirido hasta ahora. Para ayudarte a ganar experiencia
con la programación, discutiremos un caso práctico que involucra
la solución de sopas de palabras al buscar palabras que tienen
ciertas propiedades. Por ejemplo, encontraremos los palíndromos
más largos en inglés y buscaremos palabras cuyas letras aparecen
en orden alfabético. Y presentaré otro plan de desarrollo de programa:
reducción a un programa que se solucionó previamente.

\section{Lectura de y Escritura a un Archivo}

Para los ejercicios en este capítulos, necesitaremos que nuestros
programas lean texto desde los archivos. En muchos lenguajes 
de programación, usualmente esto significa que necesitamos una sentencia
que abra un archivo, después una sentencia o grupo de sentencias 
para leer el contenido del archivo, y finalmente una sentencia para
cerrar el archivo (aunque esta operación puede realizarse automáticamente
en algunas circunstancias).
\index{file!reading from}
\index{file!writing to}
\index{file!open statement}
\index{file!close statement}

Aquí estamos interesados en archivos de texto que están compuestos 
por líneas separadas por lógicos caracteres de nueva línea; 
dependiendo de tu sistema operativo, lógicos caracteres de nueva línea
consisten de uno (Linux, Mac) o dos (Windows) caracteres físicos (bytes).
\index{newline character}

La función integrada de Perl {\tt open} toma la ruta la cual indica donde el 
archivo se encuentra y el nombre del archivo. {\tt open} devuelve
un {\bf descriptor de archivo} ({\tt IO::Handle object}) que puedes usar
para leer (o para escribir en el mismo):
\index{open function}
\index{function!open}
\index{plain text}
\index{text!plain}
\index{object!file}
\index{file handle}
\index{function!slurp-rest}

\begin{verbatim}
my $fh = open("ruta/a/mi_archivo.txt", :r);
my $datos = $fh.slurp-rest;
$fh.close;
\end{verbatim}
%
La opción {\tt :r}  ({\tt read}) es el modo de escritura. {\tt \$fh} es
un nombre común para un descriptor de archivo ({\bf f}ile {\bf h}andle en inglés).
El \emph{objeto de archivo} provee métodos para leer, tales como {\tt slurp-rest}
el cual devuelve el contenido completo del archivo desde la posición actual hasta
el final (y el contenido completo del archivo lo hemos abierto por primera vez).

Esta es la manera tradicional de abrir y leer archivos en la mayoría de
los lenguajes.
\index{read mode}
\index{file mode!read}

No obstante, el rol {\tt IO} de Perl (en términos simples, un rol 
es una colección de métodos relacionados) ofrece métodos más simples
los cuales pueden abrir y leerlo en una sola instrucción (i.e., 
sin tener que primero abrir un descriptor de archivo y después cerrarlo):
\index{role}
\index{IO role}

\begin{verbatim}
my $texto = slurp "ruta/a/mi_archivo.txt";
# alternativamente:
my $texto = "ruta/a/mi_archivo.txt".IO.slurp;
\end{verbatim}
%

\index{slurp function}
{\tt slurp} toma la responsabilidad de abrir y cerrar el
archivo por ti.

Podemos también leer el archivo línea por línea, lo cual es 
mu práctico si cada línea contiene una entidad lógica tal como
un registro, y es especialmente útil para archivos muy grandes
que podrían no caber en la memoria:

\begin{verbatim}
for 'ruta/a/mi_archivo.txt'.IO.lines -> $línea {
    # Haz algo con la $línea
}
\end{verbatim}
%
Por defecto, el método {\tt .lines} removerá el caracteres de nueva línea
colgantes de cada línea, para que no tengas que preocuparte sobre ellos.
\index{IO.lines method}
\index{IO.slurp method}
\index{slurp function}
%

No hemos estudiados los arrays aún, pero puedes también leer
todas las líneas de un archivo en un array, con cada línea del
archivo volviéndose un artículo del array. Por ejemplo, puedes
cargar el archivo {\em mi\_archivo.txt} en el array \verb|@líneas|:
\index{array}

\begin{verbatim}
my @líneas = "mi_archivo.txt".IO.lines;
\end{verbatim}
%

El acceso a cada línea puede entonces hacerse con el operador bracket
y un índice. Por ejemplo, para imprimir la primera y séptima línea:

\begin{verbatim}
say @líneas[0];
say @líneas[6];
\end{verbatim}
%
Para escribir datos en un archivo, es posible abrirlo al igual 
que cuando queremos leer desde el mismo, excepto que la opción
{\tt :w} (write) (\emph{escritura}) debe usarse:
\index{write mode}
\index{file mode!write}

\begin{verbatim}
my $fh = open("ruta/a/mi_archivo.txt", :w);
$fh.say("escribir esta línea en el archivo");
$fh.close;
\end{verbatim}

Sé cuidadoso con esto. Si el archivo ya existía, cualquier
contenido será sobrescrito. Así que sé cauteloso al abrir un archivo
en el modo de escritura, debido a que podrías borrar todo el 
contenido de un archivo si no tienes cuidado. 

También es posible abrir el archivo en modo de concatenación, usando
la opción {\tt :a} (append). Con esta opción, los datos nuevos serán 
entonces agregados al contenido existente.
\index{append mode}
\index{file mode!append}

La escritura a un archivo puede simplificarse con el uso de la función
{\tt spurt}, la cual abre el archivo, escribe los datos en el mismo, y
consecuentemente cierra:
\index{spurt function}

\begin{verbatim}
spurt "ruta/a/mi_archivo.txt", "escribir esta línea en el archivo\n";
\end{verbatim}

Usado de esta manera, {\tt spurt} sobrescribirá cualquier contenido existente
en el archivo. También puede usarse en el modo de concatenación con la 
opción {\tt :append}:
\index{spurt function!append mode}

\begin{verbatim}
spurt "ruta/a/mi_archivo.txt", "escribir esta línea en el archivo\n ", :append;
\end{verbatim}


\section{Lectura de Listas de Palabras}
\label{wordlist}

Para los ejercicios en este capítulo necesitamos una lista de palabras
en español. Hay muchas listas de palabras en la web, pero una de la
más adecuada para nuestro propósito es una de las listas de palabras
coleccionada y cedida al dominio público por Grady Ward como parte
del proyecto léxico Moby (ver \url{http://wikipedia.org/wiki/Moby_Project}).
Es una lista de 113,809 palabras oficiales de crucigramas; es decir, 
palabras que son consideradas válidas en un crucigrama y otros juegos de
palabras. En la colección Moby, el nombre del archivo es \emph{spanish.txt};
puedes descargar una copia
desde \url{https://www.gutenberg.org/files/3206/files/spanish.txt}.
\index{Moby Project}
\index{crosswords}
\index{word list}

\begin{description}
\item {\bf Nota del traductor}: En lugar de utilizar la lista de palabras en \emph{spanish.txt}
del proyecto Moby, he decidido utilizar la lista en \emph{lemario.txt} (\url{https://gitlab.com/uzluisf/piensaperl6/tree/master/suplementos/lemario.txt}) 
que contiene aproximadamente 88,000 palabras\footnote{
También puedes encontrar el lemario y más información sobre el mismo aquí: (\url{http://olea.org/proyectos/lemarios/index.html})}. La decisión se debe a que la lista de 
palabras en español del proyecto Moby contiene muchas palabras que no están
formateadas apropiadamente. 
\end{description}

Este archivo contiene texto simple (con cada palabra de la lista
en su propia línea), así que puedes abrirlo con un editor de texto,
pero también puedes leerlo con Perl. Hagamos esto en el modo
interactivo (con el REPL):
\index{interactive mode}
\index{REPL}

\begin{verbatim}
> my $fh = open("lemario.txt", :r);
IO::Handle<lemario.txt>(opened, at octet 0)
> my $línea = get $fh;
aa
> say "<<$línea>>";
<<aa>>
\end{verbatim}

La función {\tt get} lee una línea del descriptor de archivo.
\index{get function}

La primera palabra en esta lista en particular es ``a``.

Imprimir la variable \verb|$línea| entre los paréntesis angulares
dentro de una cadena de texto nos muestra que la función {\tt get}
removió explícitamente los caracteres de nueva línea colgantes, en este
caso una combinación de los caracteres \verb|\r\n| (un retorno de carro
 y una nueva línea), dado que este archivo aparentemente fue preparado en Windows.

El descriptor de archivo mantiene un registro de lo que se ha leído
desde el archivo y lo que debe leerse después, así que se llamas
a {\tt get} otra vez, obtienes la siguiente línea:

\begin{verbatim}
> my $línea = get $fh;
a-
\end{verbatim}
%
La palabra siguiente es ``a-''. Si llamas a {\tt get} nuevamente,
obtienes la palabra ``aarónico``. 

Esto es grandioso si queremos explorar las primeras líneas del
archivo {\tt lemario.txt}, pero no vamos las 88-k líneas del 
archivo de esta forma.

Necesitamos que un bucle lo haga por nosotros. Podríamos insertar
la nueva instrucción {\tt get} en un bucle {\tt while}, pero
hemos visto una manera más fácil y más eficiente de hacerlo con
un bucle {\tt for} y el método {\tt IO.lines}, sin la inconveniencia
de abrir o cerrar el archivo:
\index{for loop}
\index{IO.lines method}

\begin{verbatim}
for 'lemario.txt'.IO.lines -> $línea {
    say $línea;
}
\end{verbatim}
%
Este código lee el archivo línea por línea, e imprime cada línea
en la pantalla. Pronto haremos cosas más interesantes que solo
mostrar las líneas en la pantalla.

\section{Ejercicios}

Este caso práctico consiste principalmente de ejercicios y soluciones
dentro del cuerpo del capítulo porque solucionar los ejercicios es la
manera principal de enseñar el material de este capítulo. Por lo tanto, 
las soluciones a estos ejercicios se encuentran en las siguientes 
secciones de este capítulo, no en el apéndice. Por lo menos 
deberías tratar de solucionar cada uno de ellos antes de leer las
soluciones.

\begin{exercise}
Escribe un programa que lee el archivo {\tt lemario.txt} e imprime
solo las palabras con más de 20 caracteres.
\index{whitespace}

\end{exercise}

\begin{exercise}

En 1939, Ernest Vincent Wright publicó una novela con aproximadamente
50,000 palabras llamada {\em Gadsby} que no contiene la letra ``e``. 
Dado que ``e`` es la letra más común en inglés, eso no es fácil
de hacer. De hecho, es difícil construir un solo pensamiento sin
usar la letra más común. Tal escritos donde se omite alguna letra
es algunas veces conocido como un lipograma.
\index{Wright, Ernest Vincent}
\index{Gadsby}
\index{lipogram}

Escribe una subrutina llamada \verb|sin-e| que devuelve {\tt True}
si la palabra dada no contiene la letra ``e``.
\label{no_e}

Modifica tu programa del ejercicio previo para imprimir solamente
palabras que no tienen ``e`` y calcular el porcentaje de palabras
en la lista que no tienen ``e``.

(El diccionario de palabras que estamos usando, \emph{lemario.txt}, 
tiene completamente en letras minúsculas, así que no tienes que preocuparte
acerca de la ``E`` mayúscula.)


\end{exercise}


\begin{exercise} 

Escribe una subrutina llamada {\tt evitar} que toma una palabra
y una cadena de texto de letras prohibidas, y devuelve {\tt True}
si la palabra no contiene ninguna de las letras prohibidas.

Después, modifica tu programa para encitar el usuario a entrar
una cadena de texto de letras prohibidas y después imprimir el número
de palabras que no contienen ningunas de ellas. ?`Puedes encontrar una
combinación de cinco letras prohibidas que excluya  el menor número
de palabras?
\end{exercise}



\begin{exercise}

Escribe una subrutina llamada \verb|usa-solamente| que toma una palabra
y una cadena de letras, y devuelve {\tt True} si la palabra contiene
solo las letras en la lista. ?`Puedes construir una oración usando solo
las letras {\tt acefhlo}?  
%Other than ``Hoe alfalfa?''

\end{exercise}


\begin{exercise} 

Escribe una subrutina llamada \verb|usa-todas| que toma una palabra
y una cadena de letras requeridas, y devuelve {\tt True} si la palabra
usa todas las letras requeridas por lo menos una vez. ?`Cuántas 
palabras se encuentran ahí que usan todas las vocales {\tt aeiou}? 
?`Y cuántas usan {\tt aeiouy}?

\end{exercise}


\begin{exercise}

Escribe una función llamada \verb|es_abecedaria| que devuelve
{\tt True} si las letras en una palabra aparecen en orden 
alfabético (letras dobles son aceptables). ?`Cuántas palabras
son abecedarias?

\index{abecedarian}

\end{exercise}



\section{Búsqueda}
\label{search}
\index{search!pattern}
\index{pattern!search}
\index{regex}

La mayoría de los ejercicios en la sección previa tiene algo en
común; pueden solucionarse con el patrón de búsqueda (y la función
{\tt index} que vimos en la Sección~\ref{find} 
(p.~\pageref{find}). Muchos de ellos también pueden solucionarse
con regexes.

\subsection{Palabras con Más de 20 Caracteres (Solución)}

La solución al ejercicio más simple ---imprimir todas las palabras
de {\tt lemario.txt} que tienen más de 20 caracteres--- es:

\begin{verbatim}
for 'lemario.txt'.IO.lines -> $línea {
    say $línea if $línea.chars > 20
}
\end{verbatim}
%

Dado que el código es tan simple, esto es un ejemplo típico
de un one-liner útil (como se describió en la 
Sección~\ref{one-liner mode}). Asumiendo que quieras saber las
palabras con más de 20 caracteres, no necesitas ni siquiera escribir
un script, guardarlo y ejecutarlo. Simplemente puedes teclear esto
en la prompt de tu sistema operativo:
\label{one-liner-example}
\index{one-liner mode}

\begin{verbatim}
$ perl6 -n -e '$_.say if $_.chars > 20;' lemario.txt
\end{verbatim} 
%
La opción ``-e`` le dice a Perl que el script a ser ejecutado
viene en la línea de comando entre comillas. La ``-n`` le indica
a Perl que lea el archivo línea por línea después de la línea
de comando, almacene cada línea en la variable tópico \verb|$_|,
y aplique el contenido del script a cada línea. Y el script en modo
one-liner solo imprime el contenido de \verb|$_| si su longitud
es mayor que 20.
\index{topical variable}

Para simplificarlo un poco, las dos opciones \verb|-n| y \verb|-e|
pueden agruparse como \verb|perl6 -ne|. En adición, la variable tópico
\verb|$_| puede omitirse en las llamadas de método (en otras palabras,
si un método no tiene invocante, el invocante será \verb|$_| por defecto).
Finalmente, el punto y coma colgante puede igualmente removerse.
El script en modo one-liner más arriba puede entonces simplificarse
aún más:
\index{invocant}

\begin{verbatim}
$ perl6 -ne '.say if .chars > 20' lemario.txt
\end{verbatim} 
%
Recuerda que, si está intentando esto en Windows, necesita reemplazar las 
comillas simples con las comillas dobles (y viceversa si el script
contiene comillas dobles):

\begin{verbatim}
C:\Users\Laurent>perl6 -ne ".say if .chars > 20" lemario.txt
\end{verbatim} 
%

\subsection{Palabras con No ``e'' (Solución)}

Una subrutina que devuelve {\tt True} para palabras que no tienen ``e''
(Ejercicio~\ref{no_e}) es también simple:

\begin{verbatim}
sub sin-e (Str $palabra) {
    return True unless defined index $palabra, "e";
    return False;
}
\end{verbatim}
%

La subrutina simplemente devuelve {\tt True} si la función {\tt index} 
no encontró una ``e`` en la palabra recibida como un parámetro y
por lo contrario, devuelve {\tt False}.

Nota que esto funciona correctamente porque la lista de palabras usada
(\emph{lemario.txt}) está completamente en letras minúsculas. La subrutina
más arriba necesitaría ser modificada si fuera a ser llamada con palabras
que contienen letras mayúsculas.

Dado que la función {\tt defined} devuelve un valor Booleano, podríamos 
simplificar nuestra subrutina:
\begin{verbatim}
sub sin-e (Str $palabra) {
    not defined index $palabra, "e";
}
\end{verbatim}
%

Podríamos también haber usado un regex para comprobar la 
presencia de una ``e`` en la segunda línea de esta subrutina:

\begin{verbatim}
    return True unless $palabra ~~ /e/;
\end{verbatim}
%

Esta sintaxis bastante concisa es atractiva, pero cuando se
busca por una coincidencia literal exacta, la función {\tt index}
es probablemente más eficiente (más rápida) que un regex.

Buscar y contar las palabras sin ``e`` en nuestra lista de palabras 
no es muy difícil:

\begin{verbatim}
sub sin-e (Str $palabra) {
    not defined index $palabra, "e";
}

my $cuenta-total = 0;
my $cuenta-sin-e = 0;
for 'lemario.txt'.IO.lines -> $línea { 
    $cuenta-total++;
    if sin-e $línea {
        $cuenta-sin-e++;
        say $línea;
    }
}
say "=" x 24;
say "Cuenta total de palabras: $cuenta-total";
say "Palabras sin 'e': $cuenta-sin-e";
printf "Porcentaje de palabras sin 'e': %.2f %%\n", 
        100 * $cuenta-sin-e / $cuenta-total;
\end{verbatim}

El programa anterior mostrará lo siguiente al final de 
su salida:

\begin{verbatim}
========================
Cuenta total de palabras: 87899
Palabras sin 'e': 36671
Porcentaje de palabras sin 'e': 41.72 %
\end{verbatim}

Así que más de un tercio de las palabras en nuestra lista no
tienen ``e``.

\subsection{Evitar Otras Letras (Solución)}

\index{generalization}
La subrutina {\tt evitar} es una versión más general de 
\verb|sin-e|, pero tiene la misma estructura:

\begin{verbatim}
sub evitar (Str $palabra, Str $prohibida) {
    for 0..$prohibida.chars - 1 -> $idx {
        my $letra = substr $prohibida, $idx, 1;
        return False if defined index $palabra, $letra;
    }
    True;
}

\end{verbatim}
%
Podemos devolver {\tt False} tan pronto encontremos una letra
prohibida; si llegamos al final del bucle, devolvemos {\tt True}.
Debido a que una subrutina devuelve la última expresión evaluada,
no necesitamos usa la sentencia {\tt return True} explícitamente
en la última línea de código más arriba. He usado esta característica
como un ejemplo; podrías encontrarlo más claro {\tt devolver} los valores 
explícitamente, excepto quizás con subrutinas de una línea muy simples.

Nota que tenemos dos bucles anidados implícitamente. Podríamos 


Note that we have here implicitly two nested loops. We could 
reverse the outer and the inner loops:

\begin{verbatim}
sub avoids (Str $word, Str $forbidden) {
    for 0..$words.chars - 1 -> $idx {
        my $letter = substr $words, $idx, 1;
        return False if defined index $forbidden, $letter;
    }
    True;
}
\end{verbatim}
%

The main code calling the above subroutine is similar to 
the code calling {\tt has-no-e} and might look like 
this:

\begin{verbatim}
my $total-count = 0;
my $count-no-forbidden = 0;

for 'words.txt'.IO.lines -> $line { 
    $total-count++;
    $count-no-forbidden++ if avoids $line, "eiou";
}

say "=" x 24;
say "Total word count: $total-count";
say "Words without forbidden: $count-no-forbidden";
printf "Percentage of words with no forbidden letter: %.2f %%\n", 
        100 * $count-no-forbidden / $total-count;
\end{verbatim}    
%
  

\subsection{Using Only Some Letters (Solución)}

\verb"uses-only" is similar to our {\tt avoids} subroutine 
except that the sense of the condition is reversed:

\begin{verbatim}
sub uses-only (Str $word, Str $available) {
    for 0..$word.chars - 1 -> $idx {
        my $letter = substr $word, $idx, 1;
        return False unless defined index $available, $letter;
    }
    True;
}
\end{verbatim}
%
Instead of a list of forbidden letters, we have a list 
of available letters.  If we find a letter in {\tt word} that 
is not in {\tt available}, we can return {\tt False}. And 
we return {\tt True} if we reach the loop end.

\subsection{Using All Letters of a List (Solución)}

\verb"uses-all" is similar to the previous two subroutines, 
except that we reverse the role of the word and the string 
of letters:

\begin{verbatim}
sub uses-all(Str $word, Str $required) {
    for 0..$required.chars - 1 -> $idx {
        my $letter = substr $word, $idx, 1;
        return False unless defined index $word, $letter;
    }
    return True;
}
\end{verbatim}
%
Instead of traversing the letters in {\tt \$word}, the loop
traverses the required letters.  If any of the required letters
do not appear in the word, we can return {\tt False}.
\index{traversal}

If you were really thinking like a computer scientist, you would
have recognized that \verb"uses-all" was an instance of a
previously solved problem in reverse: if word~A uses all 
letters of word~B, then word~B uses only letters of word~A. So, 
we can call the {\tt uses-only} subroutine and write:

\begin{verbatim}
sub uses-all ($word, $required) {
    return uses-only $required, $word;
}
\end{verbatim}
%
This is an example of a program development plan called 
{\bf reduction to a previously solved problem}, which means that you
recognize the problem you are working on as an instance of a solved
problem and apply an existing solution. 
\index{reduction to a previously solved problem} 
\index{development plan!reduction}

\subsection{Alphabetic Order (Solución)}

For \verb"is_abecedarian" we have to compare adjacent letters. 
Each time in our {\tt for} loop, we define one letter as our 
current letter and compare it with the previous one:

\begin{verbatim}
sub is_abecedarian ($word) {
    for 1..$word.chars - 1 -> $idx {    
        my $curr-letter = substr $word, $idx, 1;
        return False if $curr-letter lt substr $word, $idx - 1, 1;  
    }     
    return True
}
\end{verbatim}

An alternative is to use recursion:
\index{recursion}

\begin{verbatim}
sub is-abecedarian (Str $word) {
    return True if $word.chars <= 1;
    return False if substr($word, 0, 1) gt substr($word, 1, 1);
    return is-abecedarian substr $word, 1;
}
\end{verbatim}

Another option is to use a {\tt while} loop:

\begin{verbatim}
sub is_abecedarian (Str $word):
    my $i = 0;
    while $i < $word.chars -1 {
        if substr($word, $i, 1) gt substr ($word, $i+1, 1) {
            return False;
        }
        $i++;
    }
    return True;
}
\end{verbatim}
%
The loop starts at {\tt \$i=0} and ends when {\tt i=\$word.chars -1}.  Each
time through the loop, it compares the $i$th character (which you can
think of as the current character) to the $i+1$th character (which you
can think of as the next).

If the next character is less than (alphabetically before) the current
one, then we have discovered a break in the abecedarian trend, and
we return {\tt False}.

If we get to the end of the loop without finding a fault, then the
word passes the test.  To convince yourself that the loop ends
correctly, consider an example like \verb"'flossy'".  The
length of the word is 6, so
the last time the loop runs is when \verb|$i| is 4, which is the
index of the second-to-last character.  On the last iteration,
it compares the second-to-last character (the second ``s'') to 
the last (the ``y''), which is what we want.

\subsection{Otro Ejemplo de Reducción a un Problema Previamente Solucionado}

\index{palindrome}
\label{palindrome_2}

Aquí tenemos una versión de \verb|es_palindromo|(ver Ejercicio~\ref{palindrome})
que usa dos índices; uno comienza al inicio e incrementa, mientra que el
otro comienza al final y disminuye:

\begin{verbatim}
sub un-car (Str $cadena, $idx) {
    return substr $cadena, $idx, 1;
}
sub es-palindromo (Str $palabra) {
    my $i = 0;
    my $j = $palabra.chars - 1;

    while $i < $j {
        return False if un-car($palabra, $i) ne 
                        un-car($palabra, $j);
        $i++;
        $j--;
    }
    return True;
}
\end{verbatim}

O podríamos reducirlo a un problema previamente solucionado
y escribir:
\index{reduction to a previously solved problem}
\index{development plan!reduction}

\begin{verbatim}
sub es-palindromo (Str $palabra) {
    return es-inversa($palabra, $palabra)
}
\end{verbatim}
%
usando \verb|es-inversa| de la Sección~\ref{isreverse} (
pero deberías probablemente elegir la versión corregida de la
subrutina \verb|es-inversa| dada en el apéndice: 
ver Subsección~\ref{sol_isreverse}).


\section{Depuración de Programas}
\index{debugging}
\index{testing!is hard}
\index{program!testing}

Hacer pruebas manuales de los programas es difícil. Las funciones en este
capítulo son relativamente fáciles para probar porque puedes chequear 
el resultado manualmente. Aún así, es algo entre difícil e imposible elegir
un conjunto de palabras que prueben para todos los errores posibles.

Por ejemplo, tomemos el caso de la subrutina \verb|sin-e|. En esta función, hay dos casos
que debemos chequera: palabras que tienen una ``e`` debería devolver {\tt False},
y palabras que no tienen ninguna ``e`` deberían devolver {\tt True}. Esto
no representa problema alguno debido a que es fácil encontrar palabras
que representan cada caso.

Dentro de cada caso, existen casos que son menos obvios. Entre las palabras
que tienen una ```e``, deberías probar palabras con una ``e`` al inicio, 
al final, y en el medio. Deberías probar con palabras largas, palabras cortas
y con palabras muy cortas, como una cadena de texto vacía. La cadena de texto 
vacía es un ejemplo de un {\bf caso especial}, el cual es uno de los 
casos que no son obvios donde los errores usualmente merodean.
\index{special case}

Además de las pruebas de casos que generes, puedes también someter
tu programa a una prueba con una lista de palabras como \emph{lemario.txt}.
Al escanear la salida, podrías atrapar errores, pero debes ser cuidadoso:
podrías atrapar un tipo de error (palabras que no deberían ser incluidas,
pero son incluidas) y dejar pasar otros errores (palabras que deberían ser
incluidas pero no son incluidas).

En general, las pruebas pueden ayudar a encontrar errores, pero no es fácil
generar un buen conjunto de pruebas de casos, y aún si lo haces, no puedes
estar seguro que tu programa es correcto.

Según un legendario científico de la computación:
\index{testing!and absence of bugs}

\begin{quote}
	
La depuración de programas puede usarse para mostrar la presencia
de errores, pero jamás para mostrar la ausencia de los mismos\footnote{
"Program testing can be used to show the presence of bugs, 
but never to show their absence!" --- Edsger W. Dijkstra
}.

--- Edsger W. Dijkstra
\end{quote}
\index{Dijkstra, Edsger}


\section{Glosario}

\begin{description}

\item[Objeto de archivo] Un valor que representa un archivo abierto.
\index{file object}
\index{object!file}

\item[Reducción a un problema previamente solucionado] Una manera de 
solucionar un problema al expresarlo como una instancia de un problema
que fue solucionado previamente.
\index{reduction to a previously solved problem}
\index{development plan!reduction}

\item[Caso especial] Un caso de prueba que es atípico o que no es evidente
(y con pocas posibilidades de ser manejado correctamente). Los términos
\emph{edge case} and \emph{corner case} expresan más o menos la misma
idea.
\index{special case}
\index{edge case}
\index{corner case}

\end{description}


\section{Exercises}

\begin{exercise}
\index{Car Talk}
\index{Puzzler}
\index{double letters}
\label{cartalk}

This question is based on a Puzzler that was broadcast on the radio
program {\em Car Talk} 
(\url{http://www.cartalk.com/content/puzzlers}):

\begin{quote}
Give me a word with three consecutive double letters. I'll give you a
couple of words that almost qualify, but don't. For example, the word
committee, c-o-m-m-i-t-t-e-e. It would be great except for the ``i'' that
sneaks in there. Or Mississippi: M-i-s-s-i-s-s-i-p-p-i. If you could
take out those i's it would work. But there is a word that has three
consecutive pairs of letters and to the best of my knowledge this may
be the only word. Of course there are probably 500 more but I can only
think of one. What is the word?
\end{quote}

Write a program to find it.

Solution: \ref{sol_cartalk}.

\end{exercise}


\begin{exercise}
Here's another {\em Car Talk}
Puzzler (\url{http://www.cartalk.com/content/puzzlers}):
\label{cartalk2}
\index{Car Talk}
\index{Puzzler}
\index{odometer}
\index{palindrome}

\begin{quote}
``I was driving on the highway the other day and I happened to
notice my odometer. Like most odometers, it shows six digits,
in whole miles only. So, if my car had 300,000
miles, for example, I'd see 3-0-0-0-0-0.

``Now, what I saw that day was very interesting. I noticed that the
last 4 digits were palindromic; that is, they read the same forward as
backward. For example, 5-4-4-5 is a palindrome, so my odometer
could have read 3-1-5-4-4-5.

``One mile later, the last 5 numbers were palindromic. For example, it
could have read 3-6-5-4-5-6.  One mile after that, the middle 4 out of
6 numbers were palindromic.  And you ready for this? One mile later,
all 6 were palindromic!

``The question is, what was on the odometer when I first looked?''
\end{quote}

Write a program that tests all the six-digit numbers and prints
any numbers that satisfy these requirements.
  
Solution: \ref{sol_cartalk2}.

\end{exercise}


\begin{exercise}
Here's another {\em Car Talk} Puzzler you can solve with a
search (\url{http://www.cartalk.com/content/puzzlers}):
\index{Car Talk}
\index{Puzzler}
\index{palindrome}
\label{cartalk3}

\begin{quote}
Recently I had a visit with my mom and we realized that
the two digits that make up my age when reversed resulted in her
age. For example, if she's 73, I'm 37. We wondered how often this has
happened over the years but we got sidetracked with other topics and
we never came up with an answer.

When I got home I figured out that the digits of our ages have been
reversible six times so far. I also figured out that if we're lucky it
would happen again in a few years, and if we're really lucky it would
happen one more time after that. In other words, it would have
happened 8 times over all. So the question is, how old am I now?

\end{quote}

Write a Perl program that searches for solutions to this Puzzler.
Hint: you might find the string formatting method 
{\tt sprintf} useful.
\index{sprintf function}

Solution: \ref{sol_cartalk3}.

\end{exercise}
