% intro_part_2.tex -- Introduction to second part
%
% \chapter{Introduction to the Second part of this book}

Ahora que has alcanzado el final de la primera parte 
de este libro, ya no deberías sentirte un principiante.
Ahora, deberías ser capaz de leer (y escudriñar) la 
documentación oficial de Perl~6 (\url{https://docs.perl6.org})
y encontrar tu camino.

Hay muchas cosas que decir acerca de la programación.
Los siguientes tres capítulos se dedicaran a conceptos
más avanzados y a nuevos paradigmas de programación,
incluyendo:
\begin{description}

\item[Programación Orientada a Objetos] Describiremos 
cómo construir nuestros propios tipos y métodos, lo cual
es una manera de extender el lenguaje.

\item[Uso de gramáticas] Esta es una forma de programación
declarativa en la cual defines axiomas y reglas y derivas
conocimientos de estos; las gramáticas son una forma muy
poderosa de analizar contenido textual y se usan para 
transformar el código fuente de un programa en sentencias
que se pueden ejecutar.

\item[Programación funcional] Este es aún otro tipo de paradigma 
de programación en el cual una computación se expresa como
la evaluación de funciones matemáticas.
\end{description}

Cada uno de estos capítulos probablemente merece un
libro completo (y podrían tener uno en el futuro) 
pero esperamos decirte lo suficiente acerca
de ellos para ponerte en marcha. En mi opinión,
cada programador debería saber sobre estos conceptos 
poderosos para ser capaz de seleccionar la mejor forma
de resolver un problema en particular.

Perl~6 es un lenguaje multiparadigmático, así que 
podemos cubrir estas disciplinas usando el lenguaje Perl~6.
Un número de temas que introducimos en los
capítulos anteriores deberían facilitar la fácil
adquisición de estas nuevas ideas, y esta es la 
razón por la cual pienso que es posible cubrirlas
apropiadamente solo con un capítulo para cada una de
estas disciplinas.

Habrán menos ejercicios en la segunda parte, 
porque esperamos por ahora que ya eres capaz de idear 
tus propios ejercicios y hacer tus propios experimentos.
Y habrán solo algunas soluciones sugeridas, porque 
nos estamos acercando a un nivel donde no existe una 
única solución correcta, sino muchas maneras de 
enfrentar y resolver un problema.

Con respecto al lenguaje Perl, hemos cubierto muchísimo 
material, pero, como te advertí al principio, estos
estos dista de ser exhaustivo. Los siguientes son los temas
que no cubrimos (y no cubriremos); podrías desear explorar
la documentación por ti mismo:

\begin{description}
\item[Programación concurrente] Las computadoras de hoy en día
poseen procesadores múltiples o procesadores con múltiples
núcleos; Perl~6 ofrece varias formas de tomar ventaja de estos
para ejecutar procesos de computación en paralelo para así 
mejorar el rendimiento y reducir el tiempo de ejecución; ver
\url{https://docs.perl6.org/language/concurrency} 
for more details.

\item[Manejo de excepciones] El manejo de situaciones
cuando algo no funciona correctamente es una parte 
importante de la programación. Perl~6 ofrece varios
mecanismos para el manejo de tales situaciones.
Ver \url{https://docs.perl6.org/language/exceptions} 
para más detalles.

\item[Comunicación entre procesos:] Programas usualmente tienen
que ejecutar otros programas y comunicarse con los mismos. Ver 
\url{https://docs.perl6.org/language/ipc}.

\item[Módulos] Cómo crear, usar, y distribuir módulos de
Perl~6. Ver \url{https://docs.perl6.org/language/modules}.

\item[Intefaz para llamadas nativas] Cómo llamar librerías que están
escritas en otros lenguajes de programación y seguir las
convenciones de llamadas del lenguaje C.
Ver \url{https://docs.perl6.org/language/nativecall}

\end{description}