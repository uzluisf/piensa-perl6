% intro_part_1.tex -- Introduction to first part of the book
%

Este libro está dividido en dos partes. La razón principal 
por esta decisión es que yo quería hacer una distinción entre 
las nociones relativamente básicas que cualquier programador necesita 
al usar Perl~6 y los conceptos avanzados que un buen programador
necesita saber y que pueden ser necesarios en el día a día de un programador.

Los primeros once capítulos (aproximadamente 200 páginas) 
que constituyen la primera parte están diseñados para enseñar los
conceptos básicos que cada programador debe conocer: variables,
expresiones, condiciones, recursión, precedencia de operadores, bucles,
etc., y también las estructuras de datos básicas usadas comúnmente, 
y los algoritmos más útiles. Creo que estos capítulos pueden
ser la base de un curso introductorio a la programación
de un semestre.
 
Por supuesto, el profesor o instructor que desee utilizar este material
es libre de saltar algunos detalles en la Parte~1 (y también incluir 
de la Parte~2). Por lo menos, he incluido algunas recomendaciones
sobre como este libro podría ser usado para enseñar programación
usando el lenguaje Perl~6.

La segunda parte se enfocan en paradigmas diferentes y algunas
técnicas avanzadas de programación que son en mi opinión de gran
importancia, pero que deberían ser estudiados en el contexto
de un segundo semestre más avanzado.

Por ahora, comencemos con lo básico. Espero que disfrute la travesía.

