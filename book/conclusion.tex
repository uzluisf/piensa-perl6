\chapter{Algunos Consejos Finales}

\begin{quote}
\raggedleft 
\emph{Todo el mundo sabe que la depuración es dos veces más difícil\\
que escribir un programa en primer lugar. Así que si eres tan astuto\\
como puedas cuando lo escribes, cómo podrás depurarlo? }\\
--- Brian Kernighan, "The Elements of Programming Style".	
\end{quote}
% Everyone knows that debugging is twice as hard as writing \\ 
% a program in the first place. So if you're as clever as you \\
% can be when you write it, how will you ever debug it? }\\
% --- Brian Kernighan, "The Elements of Programming Style".
\index{Kernighan, Brian}

\section{Hazlo Claro, Mantenlo Simple}

Escribir un programa en la vida real no es lo misma cosa que 
aprender el arte de la programación o aprender un nuevo lenguaje.

Debido a que la meta de este libro es conducirte a aprender
más conceptos avanzados o nueva sintaxis, usualmente he tratado
de hacer las cosas de diferentes maneras. Pero esto no significa 
que deberías tratar de empaquetar tu conocimiento más avanzado
en cada uno de tus programas. Es todo lo opuesto.

La regla de oro es ``KISS``: Keep It Simple, Stupid (Mantenlo Simple, Estúpido).
El principio de ingeniería KISS (originado en el US NAVY alrededor de 1960)
dice que la mayoría de sistemas funcionan mejor si se mantienen
simples en lugar de complicados; por lo tanto la simplicidad
debería ser una meta en el diseño y la complexidad innecesaria 
debería evitarse. No obstante, esto no significa que deberías escribir 
código simplista.
\index{KISS: keep it simple, stupid}

Como un ejemplo, si estás buscando un pedazo de cadena de texto
literal dentro de una cadena, usa la función integrada {\tt index},
en lugar de utilizar el armamento completo del motor de regex para eso.
Similarmente, si sabes la posición y longitud del pedazo de 
cadena de texto, usa la función {\tt substr}. Pero si necesitas una
coincidencia más ``parcial`` (``fuzzy``) con quizás algunas alternativas o 
una clase de caracteres, entonces un regex podría ser la herramienta
indicada para eso.
\index{index}
\index{substr}
\index{regex}

Otro principio relacionado es ``YAGNI``: you aren't gonna need 
it (no lo vas a necesitar). Este acrónimo viene de la escuela 
de programación conocida como ``extreme programming`` (XP).
Aunque no sigas todos los principios de XP, esta idea es
válida y justificada: no agregues una funcionalidad hasta que 
realmente la necesites.
\index{YAGNI: you aren't gonna need it}

Trata de hacer tus programas tan claros como sea posible,
y tan simples como puedas. Usa conceptos más avanzados solo
si es realmente necesario, pero no lo hagas solo con el propósito
de presumir.  No trates de ser astuto o, por lo menos,
no tan \emph{astuto}.

Recuerda que el código no solo es usado por el compilador,
sino que también por humanos. Piensa en ellos.

Contempla sobre la persona que mantendrá tu código. 
Como a algunas personas le gusta decir: ``Siempre escribe
código como si la persona que termina manteniendo tu código
es un psicópata violento que sabe donde vives.`` Y, si esto no te 
convence, recuerda que la persona que mantenga tu código dentro de
un año podría set tú. Para ese entonces, puede ser que no recuerdes cómo ese
truco astuto que usaste realmente funciona.

Una cita final de Edsger Dijkstra sobre el tema:
``La simplicidad es un prerequisito para la fiabilidad.``
\index{Dijkstra, Edsger}
\index{simplicity}

\section{Qué Hacer y Qué No Hacer}

\begin{description}

\item[No te repitas:] Evita la duplicación de código.
Si tienes el mismo código en diferentes lugares de tu programa,
entonces algo malo puede suceder. Quizás el código repetido
puede ir dentro de un bucle o una subrutina separada,
o a lo mejor en un módulo o una librería. Recuerda que copiar
y pegar es una fuente de maldad.
\index{DRY: don't repeat yourself}

\item[No invente la rueda de nuevo:] Usa librerías y módulos
existentes cuando puedas; es posible que hayan sido probadas
y funcionarán mejor que el pequeño arreglo que escribirás.
El ecosistema de Perl~6 tiene una gran colección de módulos
de software (ver \url{modules.perl6.org})
que crece día a día. Estos módulos  puedes utilizarlos
en tus programas. 
\index{reinventing the wheel}

\item[Usa identificadores significativos] Si tus variables, métodos,
clases, gramáticas, módulos, y programas tienen nombres sensibles
que expresan claramente lo que son o lo que hacen,
entonces tu código será más claro y podría necesitar menos
comentarios. Nombres muy cortos como \verb|$i| o \verb|$n|
son usualmente buenos para las variables del bucle, pero 
todo lo demás necesita un nombre que explique el contenido 
o el propósito. Nombres como \verb|$array| o \verb|%hash|
se han usados a lo largo de este libro para indicar 
con más claridad la naturaleza de la estructura de datos,
pero se recomienda usarlos en programas en la vida real.
Si tu hash contiene una colección de palabras, llámalo 
\verb|%palabras| o \verb|%lista-palabras|, no \verb|%hash|.
De cualquier modo, el sigilo \% ya indica que es un hash.
\index{meaningful identifier}

\item[Escribe comentarios útiles y evita los inútiles] Un 
comentario como este:
\begin{verbatim}
my $count = 1;     # asigna 1 a $count
\end{verbatim}
es completamente inútil. En general, tus comentarios no deberían
explica lo que tu código está haciendo o cómo lo está haciendo
(esto debería ser obvio si tu código es claro), pero
\emph{la razón por la cual estás haciendo eso}: quizás deberías
hacer referencia a un teorema matemático, un ley de física, una
decisión de diseño, o una regla de negocio.
\index{comment}

\item[Remueve código muerto y código de andamiaje]
Al escribir código nuevo, puedes crear variables que no 
usas en la versión final de tu código. Si esto sucede, 
remuévelas; no las dejes distraer la atención de tu lector. 
Si modificas un programa existente, limpia el lugar después
de modificarlo. Recuerda la regla de los boy scouts: 
deja el lugar mejor y más limpio de como lo encontraste.
\index{dead code}
\index{scaffolding}

\item[Prueba agresivamente] Nadie puede escribir una pieza de
código significante sin tener un número inicial de errores.
Se cita a Edsger Dijkstra diciendo: ``Si la depuración 
es el proceso de remover errores de software, entonces
la programación debe ser el proceso de crearlos.``
Lamentablemente es cierto. Aunque Dijkstra también dijo
que ``las pruebas de software muestran la presencia, no la ausencia de
errores,`` las pruebas de software son una parte esencial del
desarrollo de software. Escribe planes de pruebas extensos, úsalos
a menudo, y actualízalos a medida que la funcionalidad evoluciona.
Ve la Sección~\ref{test_module} (p.~\pageref{test_module}) para algunas herramientas 
de pruebas automatizadas.
\index{test}
\index{Dijkstra, Edsger}

\item[Evita la optimización prematura] En las palabras de
Donald Knuth: ``La optimización prematura es la fuente de todo lo 
malvado (o por lo menos la mayoría) en la programación.``
\index{premature optimization}
\index{Knuth, Donald}

\item[No uses números mágicos:] Considera esto:
\begin{verbatim}
my $tiempo-restante = 31536000;
\end{verbatim}

¿Qué es este número 31,536,000 que aparece de la nada?
No hay manera de saber con solo mirar esta línea de código.
Compáralo con esto:

\begin{verbatim}
my $segundosEnUnAño = 365 * 24 * 60 * 60;
# ...
my $tiempo-restante = $segundosEnUnAño;
\end{verbatim}

¿No es la segunda versión más clara?\footnote{Esta calculación se refiere al año \emph{común} (i.e. un calendario que no es bisiesto o año civil). No es la misma definición de otros
anos astronómicos, los cuales más de un cuarto de día más largo.} Bueno, para
ser sincero, sería mejor usar una constante en tal caso:
\begin{verbatim}
constant SEGUNDOS-POR-AÑO = 365 * 24 * 60 * 60;
\end{verbatim}
\index{magical number}
\index{constant}

\item[Evita valores hardcoded:] Los valores hard-coded 
son malos. Si tienes que usar alguno, defínelos como 
variables o constantes al comienzo de tu programa, y usa esas
variables o constantes en vez. Los paths hard-coded de archivos
son especialmente malos. Si tienes que usar algunos, usa
algunas variables con paths relativos:
\begin{verbatim}
my $dir-base = '/path/a/datos/aplicacion';
my $dir-entrada = "$dir-base/ENTRADA";
my $dir-result = "$dir-base/RESULT";
my $dir-temp = "$dir-base/TEMP";
my $dir-log = "$dir-base/LOG";
\end{verbatim}
Por lo menos, si el path debe cambiar, solo tienes que cambiar
la línea de código superior.
\index{hard-coded value}

\item[No ignore los errores devueltos por las subrutinas 
o funciones integradas:] No todos los valores de retorno son
útiles; por ejemplo, usualmente nos chequeamos el valor de retorno
de una sentencia de impresión, pero eso es usualmente bueno porque
no estamos interesados en el efecto secundario, el hecho de imprimir
algo a la pantalla o a un archivo, en lugar de un valor de retorno.
En la mayoría de los casos, necesites saber si algo malo pasó 
para tomar los pasos necesarios para recuperarse de la 
condición de error si es posible o abortar el programa con gracia
(e.g., con un mensaje de error informativo) si el error 
es muy serio para que el programa continúe.
\index{error!ignoring}

\item[Formatea tu código clara y consistentemente] 
El compilador puede no importarle la indentación de código,
pero eso es importante para los lectores humanos. El formato 
de tu código debería ayudar a clarificar la estructura y 
flujo de control de tus programas.
\index{indentation}

\item[Sé bueno y diviértete.]
\index{fun}

\end{description}

\section{Usa Expresiones Idiomáticas}
\index{idiom}

Cualquier lenguaje tiene sus métodos de ``buenas prácticas``.
Estas son las expresiones idiomáticas que los programadores con
experiencia usan y se han convertido en las maneras preferidas 
de hacer las cosas. Estos modismos son importantes. Ellos
te protegen de inventar la rueda de nuevo. También son lo que
los programadores con experiencia esperan leer; son familiares
y te permiten enfocarte en el diseño de código en lugar de 
atascarte en preocupaciones detalladas de código. Usualmente,
formalizan patrones que evitan equivocaciones o errores comunes.

Aunque Perl~6 es un lenguaje relativamente nuevo, un número de
tales modismos se han perfeccionados a lo largo del tiempo. 
Las siguientes son algunas de estas construcciones idiomáticas
\footnote{Cuando se sugiere dos soluciones,
	la segunda es usualmente la más idiomática.}.
\index{idiomatic Perl~6}


\begin{description}
\item[Crear un hash de una lista de llaves y una lista de valores:] 
\index{hash slice}

Usando rebanadas (\emph{slices})
\begin{verbatim}
my %hash; %hash{@llaves} = @valores;
\end{verbatim}

Usando el operador \emph{zip} y un metaoperador con el constructor pareja:
\index{zip operator}
\index{metaoperator}
\index{operator!zip}

\begin{verbatim}
my %hash = @llaves Z=> @valores;
\end{verbatim}

Para las pruebas existente, los valores de los hash solo necesitan ser 
verdaderos. Esta es una buena manera de crear un hash de una lista de
llaves:
\begin{verbatim}
my %existentes = @llaves X=> True;
\end{verbatim}

O, mejor aún, usa un set:
\begin{verbatim}
my $existentes = @llaves.Set;
say "existe" if $existentes{$objeto};
\end{verbatim}

\item[Hacer los atributos mandatorios (o parámetros de la subrutina):]
Esto es una forma buena de hacer un atributo mandatorio en una clase:
\index{mandatory attribute}
\begin{verbatim}
has $.atr = die "El atributo 'atr' es mandatorio";
\end{verbatim}
Este código usa el mecanismo de valor por defecto: si un valor es 
provisto, entonces el código para el valor por defecto no se ejecuta.
Si no valor es provisto, entonces el código muere con el mensaje 
de error apropiado. El mismo mecanismo puede usarse para los
parámetros de las subrutinas.

O puedes usar el rasgo \verb|is required|:

\begin{verbatim}
> class A { has $.a is required }; 
> A.new;
The attribute '$!a' is required, 
but you did not provide a value for it.
\end{verbatim}

Hasta puedes pasar un mensaje con una explicación:

\begin{verbatim}
> class A { has $.a is required("Lo necesitamos") }; 
> A.new;
The attribute '$!a' is required because Lo necesitamos,
but you did not provide a value for it.
\end{verbatim}

\item[Iterar sobre los subíndices de un array] La primera
solución que viene a la mente podría ser:
\index{end method}

\begin{verbatim}
for 0 .. @array.end -> $i {...}
\end{verbatim}

Eso está bien, pero esto es probablemente mejor:
\index{keys function or method}

\begin{verbatim}
for @array.keys -> $i {...}
\end{verbatim}

\item[Iterar sobre los subíndices y valored de un array] 
El método \verb|.kv|, en combinación con un bloque puntiagudo
que toma dos parámetros, te permite fácilmente iterar sobre un
array:
\index{kv function or method}

\begin{verbatim}
for @array.kv -> $i, $valor {...}
\end{verbatim}


\item[Imprimir el número de artículos en un array] Dos posibles
soluciones:
\index{elems function or method}

\begin{verbatim}
say +@array; 
# o:
say @array.elems; 
\end{verbatim}

\item[Hacer algo cada tercera vez] Usa el operador de
visibilidad \verb|%%| en la variable del bucle:
\index{divisibility}
\begin{verbatim}
if $i %% 3 {...}  
\end{verbatim}

\item[Hacer algo cada \emph{n} vez:] Usa el operador de
rango de derecha exclusiva:
\index{range operator}

\begin{verbatim}
for 0 ..^ $n {...}
# o, más simple:
for ^$n {...} 
\end{verbatim}

\item[Divide una cadena de texto en palabras (dividiendo en espacio):]
Un invocación de método sin un invocante explícito siempre usa 
la variable tópico \verb|$_| como un invocante implícito. 
Así que, asumiendo que la cadena de texto ha sido asignada
a \verb|$_|:
\index{split function or method}
\index{words function or method}
\index{topical variable}
\index{invocant}

\begin{verbatim}
@palabras = .split(/\s+/);
# o, más simple:
@palabras = .words;
\end{verbatim}

\item[Un bucle infinito] Una sentencia de {\tt bucle} 
sin paréntesis y sin argumentos se ejecuta indefinidamente:
\index{infinite loop}
\index{loop!infinite}

\begin{verbatim}
while True {...}
# o, más idiomático:
loop {...}   
\end{verbatim}

Por supuesto, el cuerpo de la sentencia {\tt bucle} 
debe tener algún tipo de sentencia de flujo de control 
para salir del bucle en un momento determinado.

\item[Returnar los elementos únicos de una lista] El método
{\tt unique} remueve los duplicados de la lista de entrada:
\index{unique function}

\begin{verbatim}
return @array.unique;
\end{verbatim}

O, si sabes que la lista está ordenada, puedes usar la
función {\tt squish} (la cual remueve los duplicados adyacentes).
\index{squish function}

\item[Añadir los elementos de una lista] Usa la función 
{\tt reduce} o el metaoperador de reducción:
\index{reduce function}
\index{reduction operator}
\index{sum function or method}

\begin{verbatim}
my $suma = @a.reduce(* + *);
# o, más simple:
my $suma = [+] @a;
# o hasta más simple, usando la suma integrada:
my $suma = @a.sum;  
\end{verbatim}

\item[Intercambiar dos variables] Usa la invocación
mutante del método \verb|.=| con la función \verb|reverse|:
\begin{verbatim}
( $x, $y ) =   $y, $x;
# o:
( $x, $y ) .= reverse; # equivalente a: ($x, $y) = ($x, $y).reverse
\end{verbatim}
\index{swapping variables}

\item[Generar enteros aleatorios entre 2 y 6] Usa el operador
de rango \verb|..| y el método {\tt pick}:
\index{pick function or method}
\index{random number}
\begin{verbatim}
$z = 2 + Int(5.rand);
# o, mejor:
$z = (2..6).pick;
\end{verbatim}

\item[Contar progresivamente de 3 en 3 en un bucle infinito] Usa 
el operador de secuencia con el operador whatever star ``*``
y el operador puntiagudo:
\index{whatever operator}
\index{sequence operator}

\begin{verbatim}
for 3, * + 3 ... * -> $n {...}
# o:
for 3, 6, 9 ... * -> $n {...}    
\end{verbatim}

\item[Recorrer un rango de valores, ignorando los límites del rango:] Usa
el operador de rango exclusivo:
\index{range operator}

\begin{verbatim}
for ($comienzo+1) .. ($final-1) -> $i {...}
# o, mejor:
for $comienzo ^..^ $final -> $i {...}
\end{verbatim}
\end{description}

\section{¿Qué sigue?}

En un libro como este no se te puede decir todo sobre la
programación ni sobre Perl~6. A esta altura, deberías saber cómo
escribir un programa para resolver un problema de dificultad promedio,
pero mucho se ha hecho en la última década para resolver problemas
más complejos. Así que, hacia dónde deberíamos dirigirnos desde aquí?

Lee libros sobre algoritmos. Muchos libros excelentes tratan el tema,
pero especialmente recomiendo los siguientes dos (aunque deberías saber que no
son fáciles):
\begin{itemize}
\item Thomas H. Cormen, Charles E. Leiserson, Ronald L. Rivest, 
and Clifford Stein, \emph{Introduction to 
Algorithms}, The MIT Press
\item Donald Knuth, \emph{The Art of Computer Programming}, Addison Wesley 
(many volumes, many editions).
\end{itemize}

Lee otros libros sobre programación, aún sus objetivos sean
otros lenguajes de programación o no lenguaje en específico.
Es probable que tengan una perspectiva diferente sobre varios
temas; esto te ofrecerá una perspectiva diferente y quizás una
mejor entendimiento, y complementará lo que leíste aquí.
Lee tutoriales, artículos, blogs, y foros sobre programación.
Participa cuando puedas. Lee la \emph{Introduction to Perl~6}
la cual existe en ocho lenguajes diferentes al momento de
escribir este libro (\url{http://perl6intro.com/}). Lee la
documentación oficial de Perl~6 (\url{https://docs.perl6.org}).
\index{Perl~6 documentation}

Este libro tiene más de mil ejemplos de código, lo cual es mucho,
pero podría ser no suficiente si quieres aprender mucho más.
Debería leer muestras de código escritas por otros. Mira las 
librerías o módulos de código abierto e intenta entender lo que 
hace y cómo lo hacen. Intenta usarlas.

Habiendo dicho eso, debería enfatizar que puedes leer tantos libros
como quieras sobre la teoría de la natación, pero nunca aprenderás
a nadar hasta que lo intentes. Lo mismo es cierto sobre aprender 
a programar y aprender un lenguaje de programación. Escribe nuevo
código. Modifica los ejemplos existentes y observa lo que pasa. 
Intenta cosas nuevas. Continúa, sé valiente, sumérgete en la piscina
y nada. El resultado es:  aprenderás al hacer.

¡Aprender el arte de la programación es divertido. Disfrútalo!
\index{fun}

