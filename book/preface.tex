
\chapter{Introducción}

Bienvenido al arte de programación de computadoras y 
al nuevo lenguaje Perl~6. Este será probablemente el primer
libro publicado que usa Perl~6 (o uno de los primeros), 
un lenguaje de programación poderoso, expresivo, maleable
y altamente extensible. Pero este libro es menos acerca de
Perl~6, y más sobre cómo aprender a escribir programas 
de computadora. 

Este libro está destinado a principiantes y no requiere ningún conocimiento
previo de programación, aunque espera que aquellos con experiencia de 
programación puedan aún así beneficiarse.

\section*{El Objetivo de este Libro}

El objetivo de este libro no es primariamente enseñar Perl~6,
sino enseñar el arte de programación, usando el lenguaje Perl~6. 
Después de haber completado el libro, con suerte deberías de ser capaz
de escribir programas para resolver problemas relativamente difíciles en 
Perl~6, pero mi objetivo principal es enseñar ciencia de la computación,
programación de software, y resolución de problemas más que solamente
enseñar el lenguaje Perl~6 por sí mismo.

Esto significa que no abarcaré cada aspecto de Perl~6, pero
solo un subconjunto (relativamente largo pero incompleto) del
lenguaje. Este libro no pretende ser una referencia del lenguaje.

No es posible aprender programación o aprender un nuevo
lenguaje de programación solo con leer un libro;
la práctica es esencial. Este libro contiene
muchos ejercicios. Se te anima a que haga un esfuerzo real
para hacerlos. Independiente a si eres capaz de resolver los 
ejercicios, deberías mirar las soluciones en el Apéndice, dado
que, muy a menudo, se sugieren varias soluciones con una extensa
discusión en la materia y los asuntos relacionados. Algunas
veces, la sección de solución del Apéndice introduce ejemplos
de temas que serán cubiertos en el siguiente capítulo---y algunas 
veces cosas que no se discutirá en otras partes del libro. Así que,
para sacarle provecho a este libro, te sugiero a que trates de solucionar
los ejercicios y revisar las soluciones e intentarlas.

Hay más de mil ejemplos de código en este libro;
estúdialos, asegúrate de entenderlos, y ejecútalos. Si es posible,
trata de cambiarlos y observa que pasa. Es probable que aprendas mucho
de este proceso.


\section*{La Historia de este Libro}

En el transcurso de los últimos tres a cuatro años,
he traducido o adaptado al francés un número de tutoriales
y artículos sobre Perl~6, y también he escrito algunos totalmente
nuevos en francés.~\footnote{Ve por ejemplo 
\url{http://perl.developpez.com/cours/\#TutorielsPerl6}.}
Juntos, estos documentos representaban para el final del 2015
entre 250 y 300 páginas de material sobre Perl~6. Para ese entonces,
probablemente había hecho público más material en francés que todos
los otros escritores juntos.

A fines del 2015, comencé a sentir un documento de Perl~6 para principiantes
era algo que faltaba que yo estaba dispuesto a llevar a cabo. Busqué alrededor
y encontré que aparentemente no existía nada igual en inglés. 
Me vino la idea que, después de todo, sería más útil escribir tal documento inicialmente
en inglés, para dárselo a una audiencia más amplia. Así fue que comencé a 
contemplar escribir una introducción para la programación de Perl~6 destinada
a principiantes. En aquel entonces, mi idea era sobre un tutorial de 50- a 70-paǵinas
y comencé a recopilar material e ideas en esta dirección.

Entonces, algo pasó que cambió mis planes.

En Diciembre del 2015, algunos amigos mío estaban contemplando
traducir al francés \emph{Think Python, Second Edition} de Allen B. Downey\footnote{Ve \url{http://greenteapress.com/wp/think-python-2e/}.}. 
Había leído una versión previa de ese libro y totalmente apoyé la idea
de traducirlo\footnote{Lo sé, es acerca de Python, no Perl.
Sin embargo, no creo en las ``guerras de lenguajes`` y pienso que todos
debemos aprender de otros lenguajes; para mí, el lema de Perl, ``hay más de 
una forma para hacer algo,`` también significa que hacerlo en Python (o cualquier
otro lenguaje) es realmente una posibilidad aceptable.}. Como resultado,
terminé como un co-traductor y el editor técnico de la traducción al francés de ese
libro\footnote{Ve
	\url{http://allen-downey.developpez.com/livres/python/pensez-python/}.}.

Mientras trabaja en la traducción al francés del libro sobre Python
de Allen, la idea se me ocurrió, en vez de escribir un tutorial
sobre Perl~6, sería más útil hacer un ``traducción a Perl~6`` de 
\emph{Think Python}. Debido a que estaba en contacto con Allen en 
el contexto de la traducción al francés, yo le sugerí la idea a Allen, 
quién acogió favorablemente la idea. Así fue como comencé a escribir este 
libro a final de Enero del 2016, poco tiempo después de haber completado
la traducción al francés de su libro sobre Python.

De tal manera, este libro es mayormente un derivado de \emph{Think Python}
de Allen, pero adaptado a Perl~6. Como sucedió, es también algo más que 
una ``traducción a Perl~6`` del libro de Allen: con suficiente nuevo material,
se ha convertido en un libro totalmente nuevo, con una gran deuda al 
libro de Allen, pero aún nuevo libro por el cual tomo toda la responsabilidad.
Cualquier error es mío, no de Allen.

Mi esperanza es que esto será útil para la comunidad de Perl~6, y
generalmente para la comunidad de código abierto (\emph{open source})
y la comunidad de la programación de computadoras. En una entrevista,
con \emph{LinuxVoice} (July 2015), Larry Wall, el creador de Perl~6, dijo:
``Nosotros pensamos que Perl 6 se podrá aprender como un primer lenguaje.``
¡Esperemos que este libro contribuya a  lograr este cometido!


\section*{Reconocimientos}

Realmente no sé cómo podría agradecerle a Larry Wall al nivel de gratitud
que él se merece por haber creado Perl en primer lugar, y 
Perl~6 más recientemente. Que seas bendecido por toda la eternidad, Larry.

Y gracias a todos ustedes que fueron parte de esta aventura (en ningún orden en 
particular), Tom, Damian, 
chromatic, Nathan, brian, Jan, Jarkko, John, Johan, Randall, 
Mark Jason, Ovid, Nick, Tim, Andy, Chip, Matt, Michael, Tatsuhiko, 
Dave, Rafael, Chris, Stevan, Saraty, Malcolm, Graham, Leon, 
Ricardo, Gurusamy, Scott y muchísimos otros.

Todas mis gracias también a aquellos quienes creyeron
en el proyecto de Perl~6 y que lo hicieron posible, incluyendo
a aquellos que abandonaron en algún momento pero que contribuyeron
por algún tiempo; sé que no fue siempre fácil.

Muchas gracias a Allen Downey, quien amablemente apoyó la idea
de adaptar su libro para Perl~6 y me ayudó en muchos aspectos,
pero también se abstuvo de interferir en lo que yo ponía en este
libro.

Le agradezco sinceramente a la gente de O'Reilly quienes 
aceptaron la idea de este libro y sugirieron muchas correcciones o 
mejoras. Quiero agradecer especialmente a Dawn Schanafelt, mi editor de 
O'Reilly, cuyos consejos han contribuido a hacer este un mejor libro.
Muchas gracias también a Charles Roumeliotis, el editor de copia, y 
Kristen Brown, la editora de producción, quien arregló muchos problemas
tipográficos y faltas ortográficas.

De antemano le doy gracias a todos los lectores quienes ofrecerán comentarios
o someterán sugerencias o correcciones, al igual que palabras de aliento.

Si ves algo que necesita ser corregido o que podría ser mejorado,
por favor amablemente envía tus comentarios a 
\url{think.perl6 (at) gmail.com}.
% ...


\section*{Lista de Contribuciones}

% ...
Me gustaría agradecer especialmente a Moritz Lenz y Elizabeth 
Mattijsen, quienes revisaron los detalles en los borradores de este 
libro y sugirieron un sin número de mejoras y correcciones.
Liz invirtió mucho tiempo en una revisión detallada del contenido
completo de este libro y le agradezco eternamente por sus numerosos
y muy útiles comentarios. Gracias también a Timo Paulssen y ryanschoppe quienes
también revisaron los primeros borradores y proveyeron algunos comentarios
útiles. Muchísimas gracias también a Uri Guttman, quien revisó este libro
y sugirió un gran número de pequeñas correcciones y mejoras un poco antes
de la publicación.  

Kamimura, James Lenz, y Jeff McClelland cada uno sometieron algunas correcciones
en la lista de errata en el sitio web de O'Reilly. zengargoyle señaló un carácter
falso en un regex y lo arregló en el repositorio Github del libro. zengargoyle
también sugirió una aclaración en el capítulo acerca de la programación
funcional. Otro James (segundo nombre que no conozco) sometió una errata
al sitio web de O'Reilly. Mikhail Korenev sugirió correcciones precisas
para tres muestras de código. Sébastien Dorey, Jean Forget, y Roland Schmitz 
enviaron algunos correos electrónicos sugiriendo algunas correcciones o mejoras útiles. 
Luis F. Uceta corrigió varios errores tipográficos en el repositorio
de Github. Gil Magno, zoffixznet y Joaquin Ferrero también sugirieron
varias correcciones en Github.
\clearemptydoublepage

% TABLE OF CONTENTS
\begin{latexonly}

\tableofcontents

\clearemptydoublepage

\end{latexonly}

